\chapter{The Kilonova, A Case For Multi-Messenger Astronomy}

%In physics, GW detection could provide information about strong-field gravitation, the untested domain of strongly curved space where Newtonian gravitation is no longer even a poor approximation. In astrophysics, the sources of GWs that LIGO might detect include binary NSs (like PSR 1913 + 16 but much later in their evolution); binary systems where a black hole (BH) replaces one or both of the NSs; a stellar core collapse which triggers a type II supernova; rapidly rotating, non-axisymmetric NSs; and possibly processes in the early universe that produce a stochastic background of GWs [3].

With LIGO we have just pierced a new window in the purely relativistic universe.

LIGO is sensitive to phenomena in the 10Hz to 7kHz frequency band.

Similarly to the optical case, once other frequencies were explored more science could be done, with other detection methods like PTA (Pulsar Timing Array) and LISA will shed new light to other physical phenomena and physics.

Two things brought about radical changes in Astronomy: Multi-Messenger Astronomy and Time Domain Astronomy.

Sky location is ligo is very poor because it's based mainly on triangulation of detectors.
	Sky error regions are very large (e.g. $\approx$ 850 deg2 for GW150914; Abbott et al. 2016).
	With Virgo and KAGRA and INDIGO will be of of 10-100 square degrees or less (e.g., Fairhurst 2011, Nissanke et al. 2013, Rodriguez et al. 2014). 
	It still greatly exceeds the fields of view of most radio, optical, and X-ray telescopes.

Identifying host galaxies of GW is important. We can know:
	Age of stellar population.
	Displacement due to SN birth kicks.
	Determine distance to GW source. This reduces degeneracies in the GW parameter estimation, especially of the binary inclination with respect to the line of sight.

Based on their derived peak luminosities being approximately one thousand times brighter than a nova, Metzger et al. (2010) first introduced the term `kilonova' to describe the EM counterparts of NS mergers powered by the decay of r-process nuclei

It is known that short GRBs are basically NS-NS mergers.
GRB emission is collimated and jetted, so the chances to detect one along a GW are very low.
NS-NS also emit EM radiation of lower energy or frequency by a different process that occurs in the jet but is nonetheless isotropic. 

Observational (e.g., Fong et al. 2013) and theoretical (e.g. Eichler et al. 1989, Narayan et al. 1992) evidence suggest a relation between merges with at least one NS and the ``short duration'' class of GRBs (Nakar 2007, Berger 2014). 


[literal] Short GRBs are commonly believed to be powered by the accretion of a massive remnant disk onto the compact BH or NS remnant following the merger. This is typically expected to occur within seconds of the GW chirp, making their temporal association with the GWs unambiguous (the gamma-ray sky is otherwise quiet).

For the majority of GW-detected mergers, the jetted GRB emission will be relativistically beamed out of our line of sight.
The off-axis afterglow probably does not provide a promising counterpart for most observers

critical four-way connection between kilonovae, short GRBs, GWs from NS-NS/BH-NS mergers, and the astrophysical origin of the r-process nuclei. Metzger et al. (2010)

NS-NS/BH-NS mergers are also predicted to be accompanied by a more isotropic counterpart, commonly known as a `kilonova'. Kilonovae are day to week-long thermal, supernova-like transients, which are powered by the radioactive decay of heavy, neutron-rich elements synthesized in the expanding merger ejecta (Li \& Paczynski 1998). They provide both a robust EM counterpart to the GW chirp, which is expected to accompany a fraction of BH-NS mergers and essentially all NS-NS mergers, as well as a direct probe of the un- known astrophysical origin of the heaviest elements (e.g., Metzger et al. 2010).

The most significant of those is no doubt, the Kilonova emission produced by rapid capturing of neutrons.
Neutron capture has to be faster than the beta decay rate of the neutron and that's why it has to be rapid.
This capture process is called r-process. R is for rapid.
The r-process physics is quite complicated and involves a bunch of stuff, much of which is modeled to certain confidence, but many other elements are not well known. Several ingredients to the model are not considered fully. 

Blinnikov et al. (1984) and Paczynski (1986) first suggested a connection be- tween NS-NS mergers and GRBs.

Even prior to the discovery of the first binary pulsar (Hulse \& Taylor 1975), Lattimer \& Schramm (1974, 1976) proposed that the merger of compact star binaries --in particular the collision of BH-NS systems-- could give rise to the r-process by the decompression of highly neutron-rich ejecta (e.g. Meyer 1989). 

As compared to the earlier predictions (e.g. Metzger et al. 2010), these higher opacities push the bolometric light curve to peak later in time (1 week instead of a 1 day timescale), and at a lower luminosity (Barnes \& Kasen, 2013). More importantly, the enormous optical opacity caused by line blanketing moved the spectral peak from optical/UV frequencies to the near-infrared (NIR).

BH-BH mergers have no EM counterpart, except perhaps in very specific situations. This is mainly due to lack of baryonic matter.

\section{Physics of the KN}


\subsection{Radioactive radiation in neutron-rich ejecta}

The radioactive decay rate is also largely insensitive to uncertainties in the assumed nuclear masses, cross sections, and fission fragment distribution (although the r-process abundance pattern will be e.g. Eichler et al. 2015; Wu et al. 2016; Mumpower et al. 2016).

Radioactive heating occurs through a combination of $\beta$-decays, $\alpha$-decays, and fission (Metzger et al. 2010, Barnes et al. 2016, Hotokezaka et al. 2016). 

\subsection{isotropic}

Consider the merger ejecta of total mass M, which is expanding at a constant velocity v, such that its radius is R = vt after a time t following the merger. 
We assume spherical symmetry, good first-order approximation

\subsection{r-process}

Burbidge et al. (1957) and Cameron (1957) realized that approximately half of the elements heavier than iron are synthesized via the capture of neutrons onto lighter seed nuclei (e.g., iron) in a dense neutron-rich environment in which the timescale for neutron capture is shorter than the $\beta$-decay timescale.

Freiburghaus et al. (1999) presented the first explicit calculations showing that the ejecta properties extracted from a hydrodynamical simulation of a NS-NS merger (Rosswog et al. 1999) indeed produces abundance patterns in basic accord with the solar system r-process.

In the late 50's Burbidge et al. (1957) and Cameron (1957) had already proposed that approximately half of the elements heavier than iron are synthesized via the capture of neutrons onto lighter seed nuclei (e.g., iron) in a dense neutron-rich environment in which the timescale for neutron capture is shorter than the $\beta$-decay timescale.
Rapid neutron capture process', or r-process for short, 
Despite this mechanism was known for long time, the astrophysical environments in which this happens remained a mystery.
They showed that the radioactive heating rate was relatively insensitive to the precise electron fraction of the ejecta, and they were the first to consider how efficiently the decay products thermalize their energy in the ejecta.

\subsection{ejecta (where? where from?) and opacity}

The ejecta from NS mergers are an astrophysical source of rapid neutron-capture (r-process) 

The type of radiation depends on the EOS for the ejecta and its thermodynamical properties. Opacity, nucleon and particle content, pressure, temperature, MHD state.

Li \& Paczynski (1998, LP98) first showed that the radioactive ejecta from a NS-NS or BH-NS merger provides a source for powering transient emission, in analogy with Type Ia SNe. Given the low mass and high velocity of the ejecta from a NS-NS/BH-NS merger, they concluded that the ejecta will become transparent to its own radiation quickly, producing emission which peaks on a timescale of about one day, much faster than for normal SNe (which instead peak on a timescale of weeks or longer).

Consider the merger ejecta of total mass M, which is expanding at a constant velocity v, such that its radius is R = vt after a time t following the merger. 
We assume spherical symmetry, good first-order approximation
The ejecta is hot immediately after the merger, especially if it originates from the shocked interface between the colliding NS-NS binary. 

This thermal energy cannot, however, initially escape as radiation because of its high optical depth at early times
and the correspondingly long photon diffusion timescale through the ejecta. 
As the ejecta expands, the diffusion time decreases inversely proportional with time, until eventually radiation can escape on the expansion timescale.
This condition determines the characteristic timescale at which the light curve peaks
For values of the opacity $\kappa \approx$ 1 - 100 cm2 g${}^{-1}$ which characterize the range from Lanthanide-free and Lanthanide-rich matter, respectively, the derivation predicts characteristic durations about 1 day to 1 week.

Opacity is crucial since it determines at what time and wavelength the ejecta becomes transparent and the light curve peaks. 

The temperature of matter freshly ejected at the radius of the merger (about 100 km) exceeds billions of degrees. 
However, absent a source of persistent heating, this matter will cool through adiabatic expansion, losing all but a fraction $\approx$ (R0/Rpeak) $\approx$ 1E-9 of its initial thermal energy before reaching the radius Rpeak = vtpeak at which the ejecta becomes transparent.
Such `adiabatic losses' would leave the ejecta so cold as to be effectively invisible.
In reality, the ejecta will be continuously heated by a combination of sources. 
At a minimum, this heating includes contributions from radioactivity due to r-process nuclei (and possibly free neutrons), while, more speculatively, the ejecta can be heated from within by a central engine, such as a long-lived magnetar or accreting BH.

Matter ejected either by tidal forces or due to compression-induced heating at the interface between merging bodies.
Unbound debris can have enough angular momentum to form a disk around the merge.
Outflows from this remnant disk, taking place on longer timescales of up to seconds, provide a second important source of ejecta

In the case of a NS-NS merger, the ejecta properties depend sensitively on the fate of the massive NS remnant which is created by the coalescence event.

\section{Final fate of NS mergers}

The end product of a NS-NS or BH-NS merger is a central compact remnant, either a BH or a massive NS. 
The last stages of the system will also effect the emission of the kilonova.

The final system depends sensitively on the total mass of the original NS- NS binary (e.g., Shibata \& Uryu 2000; Shibata \& Taniguchi 2006). Above a threshold mass of Mcrit about 2.6 to 3.9 solar masses the remnant collapses to a BH essentially immediately, on the dynamical time of milliseconds or less (Hotokezaka et al. 2011; Bauswein et al. 2013a).

The maximum mass of a NS, though primarily sensitive to the NS EOS, can be increased if the NS is rotating rapidly (e.g., Baumgarte et al. 2000, Ozel et al. 2010, Kaplan et al. 2014). This will result in either a HMSN or a SMNS

\subsection{Prompt BH (here goes all the NS-BH mergers)}

Stellar mass BH-BH binaries are not expected to produce luminous EM emission because there are no baryonic matter.

Besides NS-NS, a NS with a BH is also posible and it will, under some circumstances give raise to a Kilonova. The physics in this case is a bit more complicated and the parameters of the BH play a big role in determining the Kilonova shape or if there's one at all.

\subsection{HMNS -> BH (short duration: ms)}

The mass is supported exclusively by differential rotation it's called a hypermassive NS (HMNS).
This decays into a BH in a few ms.

\subsection{SMNS -> BH (minutes to much longer)}

The mass can be supported by solid body rotation is called a supramassive NS.
This can decay in a BH by a less effective mechanism and can remain stable for minutes or much longer periods.

\subsection{Indefinitely stable NS (see magnetar remnant)}

The merger of a binary with a total mass less than the maximum mass of a non-rotating NS (dependent on the particular EOS but around 2 solar masses), will produce an indefinitely stable remnant, from which a BH never forms (e.g., Metzger et al. 2008; Giacomazzo \& Perna 2013).

\section{Types of Kilonovas}

The variety of sources which contribute to heating the ejecta, particularly on timescales when the ejecta is first becoming transparent.
At a minimum, the ejecta receives heating from the radioactive decay of heavy nuclei synthesized in the ejecta by the r-process. 

\subsection{Red KN}

In the tidal tails in the equatorial plane, or in more spherical outflows from the accretion disk in cases when BH formation is prompt or the HMNS phase is short-lived, the highly neutron-rich matter (Ye < 0.29) will form heavy r-process nuclei.
This r-process will peak in the near infra-red (NIR) at J and K bands (1.2 and 2.2 μm, respectively) on a timescale of several days to a week.

\subsection{Blue KN}

In addition to the highly neutron-rich ejecta (Ye < 0.29), growing evidence suggests that some of the matter which is unbound from a NS-NS merger is less neutron rich (Ye > 0.29; e.g. Wanajo et al. 2014a; Goriely et al. 2015) and thus will be free of Lanthanide group elements (Metzger \& Fernandez 2014). This low-opacity ejecta can reside either in the polar regions, due to dynamical ejection from the NS-NS merger interface, or in more isotropic outflows from the accretion disk in cases when BH formation is significantly delayed.
By assuming a lower opacity appropriate to Lanthanide-free ejecta, the emission now peaks at the visual bands R and I, on a timescale of about 1 day at a level 2-3 magnitudes brighter than the Lanthanide-rich case.

In general, the total kilonova emission from a NS-NS merger will be a combination of `blue' and `red' components, as both high- and low-Ye ejecta components could be visible for viewing angles close to the binary rotation axis (Fig. 4). For equatorial viewing angles, the blue emission is likely to be blocked by the higher opacity of the lanthanide-rich equatorial matter (Kasen et al. 2015). Thus, although the week-long NIR transient is fairly generic, an early blue kilonova will be observed in only a fraction of mergers.

\subsection{Magnetar remnant KN}

As described in §3.1, the type of compact remnant produced by a NS-NS merger (e.g. prompt BH formation, hypermassive NS, supramassive NS, or indefinitely stable NS) depends sensitively on the total mass of the binary relative to the maximum mass of a non-rotating NS, Mmax($\Omega$ = 0). The value of Mmax($\Omega$ = 0) exceeds about 2solar masses (Demorest et al. 2010, Antoniadis et al. 2013) but is otherwise unconstrained13 by observations or theory up to the maximum value about 3solar masses set by the causality limit on the EOS. A `typical' merger of two about 1.3-1.4sm NS results in a remnant mass of about 2.3-2.4sm after accounting for neutrino losses and mass ejection (e.g., Belczynski et al. 2008). If the value of Mmax($\Omega$ = 0) is well below this value (e.g. 2.1-2.2sm), then most mergers will undergo prompt collapse or form hypermassive NSs with very short lifetimes. On the other hand, if the value of Mmax($\Omega$ = 0) is close to or exceeds 2.3-2.4sm, then a order unity fraction of NS-NS mergers could result in long-lived supramassive or indefinitely stable remnants.

If the rotational energy could be extracted in non-GW channels on timescales of hours to years after the merger (e.g., by magnetic dipole radiation), this could substantially enhance the EM emission from NS-NS mergers (e.g. Gao et al. 2013; Metzger \& Piro 2014; Gao et al. 2015; Siegel \& Ciolfi 2016a). However, for NSs of mass Mns   Mmax($\Omega$ = 0), only a fraction of the rotational energy is available to power EM emission, even in principle. This is because the loss of angular momentum that accompanies spin-down results in the NS collapsing into a BH before all of its rotational energy is released.

Nonetheless, there are several mechanisms to extract rotational energy from the indefinitely stable magnetar remnant.
There is plenty literature on the subject that suggests that rotational energy input from a stable magnetar could enhance kilonova emission. The emission is still red in color and peaks on a timescale of 1 to 2 weeks, but the luminosity is greatly enhanced compared to the radioactive case, with peak magnitudes of K $\approx$ 18- 20

\subsection{Enhancing from free neutrons}

In addition to the blue and red components, recent NS-NS merger simulations show that a small fraction of the dynamical ejecta (typically a few percent, or about 1E-4sm) expands sufficiently rapidly that the neutrons do not have time to be captured into nuclei (Bauswein et al., 2013a). This fast expanding matter, which reaches asymptotic velocities v about 0.4-0.5 c, originates from the shock- heated interface between the merging stars and resides on the outermost layers of the polar ejecta. This `neutron skin' can super-heat the outer layers of the ejecta, enhancing the early kilonova emission (Metzger et al. 2015; Lippuner \& Roberts 2015).

\section{Possibly observed KN}

Tanvir \& Metzger

Later that year, Tanvir et al. (2013) and Berger et al. (2013) presented evidence for excess infrared emission following the short GRB 130603B on a timescale of about one week using the Hubble Space Telescope. If confirmed by future observations, this discovery would be the first evidence directly relating NS mergers to short GRBs, and hence to the direct production of r-process nuclei.

\section{Rates}

Rate of GRBs from NS-NS mergers is low, less than once per year all-sky. (e.g. Metzger \& Berger 2012)
We should not expect the first --or even the first several dozen-- GW chirps from NS-NS/BH-NS mergers to be accompanied by a GRB.


Population synthesis models of field binaries predict GW detection rates of NS-NS/BH-NS mergers of about 0.2-300 per year, once Advanced LIGO/Virgo reach their full design sensitivities near the end of this decade (e.g. Abadie et al. 2010, Dominik et al. 2015).
Empirical rates based on observed binary pulsar systems in our galaxy predict a comparable range, with a best bet rate of about 8 NS-NS mergers per year (Kalogera et al. 2004; Kim et al. 2015).









