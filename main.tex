%% LyX 2.1.4 created this file.  For more info, see http://www.lyx.org/.
%% Do not edit unless you really know what you are doing.
\documentclass[12pt,english]{report}
\usepackage[T1]{fontenc}
\usepackage[latin9]{inputenc}
\usepackage{babel}
\usepackage{longtable}
\usepackage{float}
\usepackage{calc}
\usepackage{amsmath}
\usepackage{amsthm}
\usepackage{setspace}
\usepackage[unicode=true,pdfusetitle,
 bookmarks=true,bookmarksnumbered=false,bookmarksopen=false,
 breaklinks=false,pdfborder={0 0 1},backref=false,colorlinks=false]
 {hyperref}
 
\usepackage{comment}
\usepackage{graphicx}
\makeatletter

%%%%%%%%%%%%%%%%%%%%%%%%%%%%%% LyX specific LaTeX commands.
\providecommand{\LyX}{\texorpdfstring%
  {L\kern-.1667em\lower.25em\hbox{Y}\kern-.125emX\@}
  {LyX}}
%% Because html converters don't know tabularnewline
\providecommand{\tabularnewline}{\\}
\floatstyle{ruled}
\newfloat{algorithm}{tbp}{loa}[chapter]
\providecommand{\algorithmname}{Algorithm}
\floatname{algorithm}{\protect\algorithmname}

%%%%%%%%%%%%%%%%%%%%%%%%%%%%%% Textclass specific LaTeX commands.
\usepackage{UTSAthesis}      
\usepackage{times}            
\usepackage{latexsym}

%Bibliography packages
\usepackage[square]{natbib} % defines citet, citep, ...
\bibpunct{(}{)}{;}{a}{}{,} % to follow the A&A style - 
\newcommand{\aj}{AJ}
\newcommand{\apj}{ApJ}
\newcommand{\apjl}{ApJ}
\newcommand{\apjs}{ApJS}
\newcommand{\aap}{A\&A}
\newcommand{\aaps}{A\&AS}
\newcommand{\mnras}{MNRAS}
\newcommand{\nat}{Nature}
\newcommand{\araa}{ARAA}
\newcommand{\prd}{Phys. Rev. D}
\newcommand{\pasj}{PASJ}
\newcommand{\ETC}{et al.}
\newcommand{\physrep}{Physics Report}
\newcommand{\gca}{GCA}
\newcommand{\pasa}{PASA}
\newcommand{\pasp}{PASP}
\newcommand{\aapr}{A\&A~Rev.}
\newcommand{\apss}{Ap\&SS}
%End of bibliography packages 

%Added by me
\newcommand\numberthis{\addtocounter{equation}{1}\tag{\theequation}}
\newcommand{\order}{\ensuremath{\mathcal{O}}}
%End of added by me

\newenvironment{ruledcenter}{%
  \begin{center}
  \rule{\textwidth}{1mm} } {%
  \rule{\textwidth}{1mm} 
  \end{center}}%


  \theoremstyle{definition}
  \newtheorem{defn}{\protect\definitionname}
\theoremstyle{plain}
\newtheorem{thm}{\protect\theoremname}

\@ifundefined{showcaptionsetup}{}{%
 \PassOptionsToPackage{caption=false}{subfig}}
\usepackage{subfig}
\makeatother

\providecommand{\definitionname}{Definition}
\providecommand{\theoremname}{Theorem}

\begin{document}

% Committee Members
\supervisor{Mario Diaz, Ph.D.}
\committeeB{Lucas Macri, Ph.D.}
\committeeC{Matthew Benacquista, Ph.D.}
\committeeD{Eric Schlegel, Ph.D.}
\committeeE{Ricardo Lopez Mobilia, Ph.D.}

\informationitems{Doctor of Philosophy in Physics}{Ph.D.}{M.Sc.}{Department of Physics And Astronomy}{College of Sciences}{May}{ 2017 }

\thesiscopyright{Copyright 2017 Martin Beroiz \\
All rights reserved. }

\dedication{\emph{I would like to dedicate this thesis to ??????.}}


\title{\textbf{OPTICAL COUNTERPARTS TO GRAVITATIONAL WAVES}}


\author{Martin Beroiz}
\maketitle
\begin{acknowledgements}
Thanks y'all.

(Notice: If any part of the thesis/dissertation has been published
before, the following two paragraphs should be included without alteration).

\begin{singlespace}
\emph{This Masters Thesis/Recital Document or Doctoral Dissertation
was produced in accordance with guidelines which permit the inclusion
as part of the Masters Thesis/Recital Document or Doctoral Dissertation
the text of an original paper, or papers, submitted for publication.
The Masters Thesis/Recital Document or Doctoral Dissertation must
still conform to all other requirements explained in the Guide for
the Preparation of a Masters Thesis/Recital Document or Doctoral Dissertation
at The University of Texas at San Antonio. It must include a comprehensive
abstract, a full introduction and literature review, and a final overall
conclusion. Additional material (procedural and design data as well
as descriptions of equipment) must be provided in sufficient detail
to allow a clear and precise judgment to be made of the importance
and originality of the research reported. }

\emph{It is acceptable for this Masters Thesis/Recital Document or
Doctoral Dissertation to include as chapters authentic copies of papers
already published, provided these meet type size, margin, and legibility
requirements. In such cases, connecting texts, which provide logical
bridges between different manuscripts, are mandatory. Where the student
is not the sole author of a manuscript, the student is required to
make an explicit statement in the introductory material to that manuscript
describing the students contribution to the work and acknowledging
the contribution of the other author(s). The signatures of the Supervising
Committee which precede all other material in the Masters Thesis/Recital
Document or Doctoral Dissertation attest to the accuracy of this statement.}\end{singlespace}
\end{acknowledgements}
\begin{abstract}
The upcoming research in the Gravitational Wave domain has introduced new challenges to Astronomy and has opened a new window to the universe.

\textcolor{red}{Write abstract}

In the first chapter, I introduce the concepts and frame on which my thesis developed.

In the second chapter, I discuss the two main elements of a modern transient search. That is, Image Difference and Machine Learning classification.

\end{abstract}

\pageone{}

\chapter{Introduction}

In February 2016, celebrating the centenary anniversary of Einstein's first paper on gravitational waves(``Approximate integration of the field equations of gravitation''), the LVC collaboration announced the first ever direct detection of a gravitational wave, labeled GW150914.
With this, a new window of the universe for the purely relativistic astronomical phenomena was opened. 

The history of GW was not without controversies. From the discussion of whether GW would carry energy to the experiments of Joseph Weber, Gravitational Waves made its way from the theoretical realm into a stronghold position in Astrophysics.

The detection of GW150914 firmly established the base for a new kind of Astronomy.
Years to come will turn GW detection, into a full field GW Astronomy.

There is a new messenger in the universe that is purely relativistic in nature. The graviton and its wave counterpart will bring us information about high gravitational fields, compact massive objects, relativistic speed phenomena and more.
It will let us probe into the physics of the extreme gravitational field astronomical bodies.

The GW information will complete, refine and expand the understanding of our universe.
It will complement our telescopes in the radio, visible and every other frequency band. 
Optical Astronomy has become an even more important partner in this search.
Together they will uncover even more details of the inner mechanisms of the celestial bodies and their interactions.

In the following sections, I offer a brief introduction of GWs and their relation to Optical Astronomy.

\section{Gravitational Waves}

\begin{comment}

%Quadrupole radiation is the lowest allowed form and is thus usually the dominant form. In this case, the GW field strength is proportional to the second time derivative of the quadrupole moment of the source, and it falls off in amplitude inversely with distance from the source.

%As with electromagnetic waves, GWs travel at the speed of light and are transverse in character, i.e. the strain oscillations occur in directions orthogonal to the direction in which the wave is propagating.
\end{comment}

General Relativity predicts that very massive or energetic events will create traveling perturbations of the underlying spacetime metric in the form of gravitational waves.
These are linear disturbances on a flat background metric, that are propagated outwards with the speed of light fading with the inverse of the distance.
These tensorial transversal waves will modify the local metric as it travels through space.

Putting it in more technical terms, in an asymptotically flat spacetime, removed from any strong source so that one can assume an almost flat local solution, one can study weak perturbations to the flat Minkowski metric.
In this regime, the linearized Einstein field equations admit wave solutions.
These wave solutions are called Gravitational Waves. \textcolor{red}{Add who coined the term for the first time, it's in the review.}

The GW tensor metric in the TT gauge coordinate frame, has the form: (we refer the reader to Appendix \ref{gwderivation} for a more complete derivation of the metric)

\begin{equation}
h_{\mu \nu} = 
\begin{bmatrix}
0 & 0 & 0 & 0 \\
0 & h_{+}e^{\pm i k_{\mu}x^{\mu}} & h_{\times} e^{\pm i k_{\mu}x^{\mu}} & 0 \\
0 & h_{\times} e^{\pm i k_{\mu}x^{\mu}} & -h_{+}e^{\pm i k_{\mu}x^{\mu}} & 0 \\
0 & 0 & 0 & 0 \\
\end{bmatrix}
\end{equation}

\textcolor{red}{Talk about the two polarizations}

Since a spacetime metric is a measure of distance between pair of events, the metric perturbation wave will modify the relative distance between points in space as it passes by. 
It is in fact an effective oscillating strain or tidal force between free test masses.
One could then devise an instrument set to detect these relative differences in distance for such test masses.

But because of the very `perturbative' nature of GW, these differences are very small. 
To have any chance of detection, one must look into the extreme side of gravity: very massive compact astrophysical objects with big gravitational fields, moving at relativistic velocities.
Even the most powerful astronomical events, like a black hole merger will produce disturbances of $10^{-23}$ m at a few hundreds Mpc of distance.

Perhaps the first experimental evidence for the existence of GWs came 50 years after their prediction with the work of Husel and Taylor. For decades, they studied the pulses we receive from one Neutron Star (NS) in a Neutron Star Binary (NSB).
They discovered that the slow rate of period decrease of the pulses precisely matched General Relativity predictions. \textcolor{red}{To Add: It was a Nobel Prize work. It settled the debate on whether GW carry energy or not. The decrease in period rate is proof that the energy is radiated away into GW.}

Nonetheless, a direct detection of GW wouldn't come for another 50 years.
In September 14, 2015, a GW detection labeled GW150914 was finally made by the LIGO Collaboration.

Direct detection {\em in situ} of GW requires a transducer of GW in some other form measurable by common instruments.

One pioneer work along this line was the work done by Joseph Weber in the late 1960s and early 70s. He proposed using a 2 meters in length and 1 meter in diameter aluminium cylinder, that would resonate with passing GW at 1660 Hz.
These tiny resonant oscillations could in principle be measured. Even though ingenious, the carefully devised experiment could not yield positive reproducible results that convinced the scientific community. \textcolor{red}{Maybe add that there were still attempts at cryogenic resonators into the 90s?}

A whole other category of these instruments are the ones based in the interferometry of lasers running on two long arms. Several of this kind are built on different parts of the planet. A few are planned to be built in the future. Most notably is the case of Lisa, a proposed GW interferometer to be set in space trailing the Earth's orbit.

Interferometer GW detectors transduce a GW warp in space to the shrinking and stretching of relative distances of masses put far apart in two or more --mostly perpendicular-- long interferometers. 
In the case of the Earth based observatories, each interferometer is one arm several kilometers long, of an L shape facility.
The masses are the end mirrors that reflect a laser beam pointed in each arm direction. 
On normal circumstances, the beams from two different arms can be set to be (`locked') on a dark or bright fringe of the interferometer diffraction pattern.
When a GW passes by, it will alter the relative distances of the mirror masses, thus changing the optical path of the laser beams. This in turn, will translate into a shift in the diffraction pattern consistent with the deformation of space.

In the following section I offer a more detailed account on the the operation of the first of these Earth based interferometers, the LIGO observatory.

\section{The Transient World and the Time-Domain Revolution in Astronomy}

In this so-called ``Information Age'', data is king.
As we move away from the traditional one person project and manual way to deal with the information, the amount of data that needs processing becomes unmanageable by humans. 
This is not to say that individual work has no place in science, 
but that the collective gathering of information becomes more ubiquitous in science.

Astronomy has not been immune to information overload. 
The sizes of sky surveys have been increasing steadily and dramatically over the years.
From 0.1 TB and about 10 to 100 events per night now, to about 30 TB and $10^5$ to $10^6$ events per night in the LSST era, the data load and the resources to process them become overwhelming if not done by automated agents.

The big amount of data posses yet another challenge.
The identification of interesting objects of study among countless other events on the sky.
This `mining' or harvesting problem have been tackled by other disciplines by different techniques of many kinds.
They are collectively called ``Data Mining'' or ``Machine Learning'' as well as other names depending on the specific field and purpose.
Classification problems done by automated agents trained for that purpose are usually dealt within the techniques called ``Machine Learning'', and we will devote a great deal of this thesis on such field.

Rare events interesting to a particular survey or telescope must then be identified against the stationary or slowly varying sky, and must be separated from other transients of a different kind.

Perhaps, one of the most spectacular of these transient events because of its  display of brightness and energy, are the Gamma Ray Bursts (GRBs).

The brightness of this flash of gamma-rays can temporarily overwhelm all other gamma-ray sources in the Universe. The burst can last from a fraction of a second to over a thousand seconds. 

GRBs are explosions as energetic as $10^?$ ergs, liberated Gamma Rays and subsequent afterglows across the whole spectrum up to Radio frequencies.

The afterglows for GRBs can be seen in the optical as a transient brightness excess, usually located near the galaxies that host them. GRBs can be at extremely distant galaxies, and therefore optical glows can sometimes be too faint to observe.

Since the detection of the first few of these, now we have two space telescopes dedicated to find them.

To detect the optical afterglow, GRB locations were communicated to other observatories that would promptly point their telescopes in that direction.
A multi-band observation and spectrography is very valuable for the understanding and classification of these events.

The value of multi-messenger astronomy of this kind was recognized right on the onset for the understanding and classification of these events.

With the new GW detections of LIGO, a new kind of multimessenger Astronomy is born.
LIGO will bring information of a completely new information carrier: the {\em graviton}.
With GW information we can pierce into the high energetic and ultra relativistic regions in space, something that could not be done before.

It is then very important to complement such information with the EM counterpart.
Both, the GW and the EM information can bring up a more complete picture of the phenomena at hand.
Together they can give us an understanding richer than the sum of their separate contributions.
Even more, EM counterparts can most notably help where GW information fails harder, in localization.
Localization by means of the EM counterpart, not only helps on this few parameters of the position in the sky, but will also help to separate the uncertainty correlation on the geometrical disposition of a merger, for example.
This way, improving other estimates of the merge.

Such classification of transient events is nowadays done with the aid of classification agents trained with Machine Learning techniques.


\textcolor{blue}{As the data and event discovery rates increase dramatically, from about 0.1 TB and about 10 to 100 events per night now, to about 30 TB and $10^5$ to $10^6$ events per night in the LSST [4] era, available follow-up facilities would be simply overwhelmed, and unable to react to all potentially interesting events.}

\begin{comment}
The new enabling technologies will be automated or robotized.

Rare events will be harvested, or mined in the jargon of big-data, from huge data sets.

The time periods on which variability occurs range from seconds to tens or hundreds of days, as for Super Novae (SN).

Gamma Ray Bursts (GRB) are transient events ...

An important consideration on information retrieval, is how will data be process to detect the target objects in the survey.

\end{comment}

\subsection{Related Efforts}

The Palomar Transient Factory (PTF), and its iterations: iPTF (the intermediate PTF) and the future Zwicky Transient Facility (ZTF).

LSST is an ambitious project...

Oggle and 

(Blue is from Hotwiring the transient universe)

\textcolor{blue}{Essentially every field of astrophysics is touched and enhanced with the exploration of the time domain, and many interesting phenomena can be only studied in the time domain.}


\textcolor{blue}{The essential enabling technologies will be automated, robust classification and decision making for the optimal use of follow-up facilities. Given the exponential growth of data rates, the traditional ``manual'' approach from the past will simply not scale to the next generation of surveys, especially if one is interested in the rarer transients.}

\textcolor{blue}{a broad diversity of the fundamentally different physical phenomena look the same at the instance of their event detection, e.g., a ``new'' or a much brighter point source relative to the baseline image.}

\textcolor{blue}{As we have conducted more comprehensive surveys, it has become clear that there are rapid changes in the distant Universe with durations measured in days or even seconds, and that those ephemeral changes provide important clues about the nature and evolution of our Universe}

Variability of the sky usually refers to changes in brightness, \textcolor{blue}{but the position or color of the object could vary instead or as well.}

\textcolor{blue}{Other objects display primarily astrometric variability rather than photometric variability - objects in our solar system display motions visible on timescales between seconds to hours}

\textcolor{blue}{There are vastly different time periods available for detecting different kinds of variable or transient objects. The optical counterparts of gamma ray bursts brighten quickly within minutes to hours then fade over hours to days, while supernovae brighten over a period of hours to days and then fade over a period of tens to hundreds of days; neither of these events repeat. M-dwarf flares occur and disappear within minutes but recur often although unpredictably, while periodic variables may vary on a timescale between hours to days but repeat on a definite schedule, making it possible to phase observations from different periods back together.}

\textcolor{blue}{Due to hardware limitations (smaller telescopes, smaller fields of view), most past surveys have focused on one primary science goal. Still, the data from many of these projects has been adopted to search for many other kinds of variable or transient objects. Three surveys which were originally designed to search for potentially hazardous NEOs - the Lowell Observatory Near Earth Object Survey (LONEOS)[6], the Near-Earth Asteroid Tracking (NEAT) survey [22] and the Catalina Sky Survey (CSS)[17] - have (respectively) been used to discover RR Lyrae [20], joined in collaboration with the Palomar-Quest survey [10] to provide supernova candidates for the Nearby Supernova Factory [2], and serve as the source of data searched for all kinds of variability in the Catalina Real-Time Transient Survey (CRTS) [11].}

\textcolor{blue}{However, there is yet another twist - how to identify the target objects of interest among the background of millions of other potential variable or transient objects? ... this results in a pool of candidates contaminated with other variables or transients. Reducing that contamination requires additional observational data either to determine the full shape of the transient or variable lightcurve, or to determine the velocity and acceleration vectors of a solar system object, or to determine the spectral energy distribution of the object through observations at multiple wavelengths or in multiple filters.
The process of winnowing out the contaminants and identifying the true members of the target sample is the heart of ``classification''}

\textcolor{blue}{A final consideration when planning a synoptic survey is how to process the data to detect the target objects in the survey. Briefly summarized, methods for detection of variables or transients boil down to generating a catalog of objects which are `different' (in brightness or position) in an image acquired from the survey telescope from what is expected for that region of sky. One common method is to create a `template image' from a series of images previously obtained of the same field, match this template to the image from the telescope being searched for variables or transients, and then to subtract the two to create a difference image. The matching process must include matching the point-spread function of the template and science image, as well as aligning their coordinate systems. Source detection algorithms are then run on the difference image to look for objects which varied in brightness or position between the template and science image. Another common method is to create calibrated catalogs from each image directly, match objects from the catalogs created from each image with the same objects from the other catalogs using spatial correlations, and again look for things which vary in brightness or position. Each method has its advantages and disadvantages, among which are: difference imaging is typically more computationally expensive but does well at removing extended non-variable objects such as galaxies, can detect variability in non-stellar profiles, and can perform better in crowded fields, while catalog searches require more accurate calibration but do not require template images and can be used to search for variability in data from different telescopes.}

\textcolor{blue}{It is interesting to see the progression that has occurred, from searching for a particular target population (such as Type Ia SNe or Near Earth Objects or Trans-Neptunian Objects) to searching for a wide variety of objects. Current or imminent synoptic surveys, such as the Palomar Transient Factory [18], [23], Skymapper [16] and PanStarrs1 (PS) [9], and significantly larger future synoptic surveys, such as PanStarrs-4 and the Large Synoptic Survey Telescope (LSST) [15], [19], will be searching for all kinds of known transients and variables, as well as previously unknown objects where the requirements for understanding these `unknown unknowns' are still developing.}


\textcolor{blue}{Balancing the requirements for detecting and identifying a wide range of transients and variables is challenging, but these surveys are planning to meet the challenge with multi-color observations (typically some or all of u, g, r, i, z and y filters) over large portions of the sky (approximately 20,000 square degrees). Very large fields of view (7-10 square degrees) will allow these surveys to cover the sky on a rapid observational cadence; while the base plan is to cover all of the visible sky every few days (for LSST) or subsets of the visible sky every few days (PS4), it is likely that an entire range of revisit times between a few hours to several days may be explored for greater sensitivity to variability at a wider range of timescales. Data processing will be intensive - difference imaging will be used to detect transients and variables, and then software pipelines will run to identify moving objects, linking their detections into orbits. Accurate calibration of photometry and astrometry on a global reference scale requires more processing. LSST will send out public VOEvents describing the millions of transient and variable detections each night. Huge databases (upwards of 12 PB for LSST at the end of its 10 year survey) of objects and all of their detections will be created and available for queries. Users will be able to use this database to identify and classify the target samples of interest to them, as well as access the VOEvent stream to get real-time updates on targets.}

\textcolor{blue}{More than ten years ago, the gamma-ray burst (GRB) community ignited the excitement over transient astronomical events. GRBs were an enigma until ultra-fast event dissemination allowed optical identification of afterglows, leading to rich data and rich science.}
\textcolor{red}{Talk about era of information}
\textcolor{blue}{Thus we expect the infrastructure for astronomical transients to allow end-users to select in advance what types of events they want, and we expect specialist ``event brokers'' that disseminate only certain kinds of transients to their target audiences. Such dissemination should happen quickly, so that the essential telescope resources can be directed for follow-up, and we expect these decisions to be taken automatically by machines.}

\textcolor{blue}{the IVOA has settled on a flavor of XML called VOEvent for the transmission of astronomical events, and processing those would need XML streaming and filtering technologies.}


* The transient world
	* The Time Domain revolution in Astronomy
	* What are transients and which ones are we interested in

	* How can we catch transients
		* Previous work: Catalog Matching: Marica's pipeline and Shuang's work on the transient identification and comparison to my OIS code.
		* Previous work: OIS + ML (the iPTF experience)

\section{The LIGO-Virgo Collaboration}

\begin{comment}
The following text is mine, but too convoluted

The test masses for LIGO are two massive mirrors at the end of two perpendicular long arms (4 km long.) Each mirror weighs about 40 kg and each one of the arms can be thought, for simplicity, as an independent laser interferometer.

When a GW passes at an angle to this set-up, the arms will stretch and shorten in opposite directions (one arm will stretch as the other shrinks), while the laser keeps traveling the arms unaffected. This causes minuscule differences in optical paths for the light and this in turn will result in the lasers being out of phase from each other. This difference in phase from the `lock' position indicates a change in the relative positions of the mirrors.

Many other mundane situations can also produce this same effect, the most simple one being seismic activity of almost any strength. Additionally, noise from the instrument itself, from the laser beam and electronics, as well as many intrinsic vibration modes of the system complicates the output signal that will be analyzed.
\end{comment}

LIGO --the Laser Interferometer Gravitational-Wave Observatory-- (Abramovici et al. 1992) is an American observatory set to detect the GW predicted by General Relativity.

It actually consists of two separate observatories, one in the state of Washington and another one in Louisiana.

Co-founded in 1992 by Kip Thorne and Ronald Drever of Caltech and Rainer Weiss of MIT, LIGO is a joint project between scientists at MIT, Caltech, and many other colleges and universities.

Virgo is a similar observatory in Europe. Originally a project from France and Italy, it soon became a collaboration from five different countries: France, Italy, the Netherlands, Poland, and Hungary. The Virgo observatory is located in the countryside near Pisa, Italy.

Since 2007, LIGO and Virgo share their data in an umbrella collaboration named LVC, the LIGO-Virgo Collaboration.

Even though, LIGO and Virgo have comparable noise levels and detection rates, only the two interferometers in LIGO were operational at the time of the GW150914 event. 
In fact, Virgo was being upgraded to Advanced Virgo, which will have a sensitivity 10 times greater than Initial Virgo. 
When the upgrade is finished, it will join LIGO in the joint collection of GW data. This will improve errors in the parameter estimation, especially localization errors will greatly improve.

LIGO went through similar upgrades along the years.

The initial LIGO detectors were designed to be sensitive to GWs in the frequency band 40-7000 Hz, and capable of detecting a GW strain amplitude as small as $10^{-21}$.

To picture the order of magnitude of these displacements, consider that the change in length of one arm of the interferometer is only about $10^{-18}$ m, a thousand times smaller than the diameter of a proton.

To reach this sensitivity, the detectors need highly stable lasers, multiple layers of vibration isolation and advanced optic techniques.

From the initial period, LIGO had five short science runs (S1 to S5), each one improving over the previous one, culminating with S5 at design sensitivity. 
The S5 run collected a full year of triple-detector coincident data from November 2005 to September 2007.

Between Initial LIGO and Advanced LIGO more science were made under Enhanced LIGO, which provided enhancements that improved sensitivity by a factor of 2. 

But the real revolution came with Advanced LIGO, a set of additions that improved LIGO sensitivity by a factor of 10 over Initial LIGO and widened the frequency range all the way down to 10 Hz (which is known as the seismic wall).


Among the improvements is the upgrading of the laser wattage to 200 W. 
An increase in the test masses to 40 kg in order to reduce radiation pressure noise and to allow larger beam sizes. 
Larger beams and better dielectric mirror coatings combine to reduce the test mass thermal noise by a factor of 5 compared with initial LIGO.
An improvement in vibration isolation, including vertical isolation comparable to the horizontal isolation all almost all stages.
New suspension system based on fused silica rather than steel wires to reduce suspension thermal noise by almost a hundred.
New two-stage active seismic isolation instead of the passive one-stage isolation brought the seismic noise to negligible levels above approximately 10 Hz.

All these improvements combined gave Advanced LIGO a 10-fold increase in sensitivity. 
This increase also means that fainter sources can now be detected, increasing the exploration volume for GW by a factor of a thousand.

Advanced LIGO will have several observations runs, labelled O1, O2, etc. (similar to the scientific runs S1-S5) with gaps between them on which more improvements will be implemented until it reaches design sensitivity.

The projected sensitivity for each of the runs is presented in table \ref{ligoruns}
(`Prospects for Observing and Localizing Gravitational-Wave Transients with Advanced LIGO and Advanced Virgo' - Abbott et al 2016)
(`The First Two Years of Electromagnetic Follow-Up with Advanced LIGO and Virgo' Singer et al )


\begin{table}
\centering
\begin{tabular}{*{9}{|c}|}
  \hline
   & Estimated & 
  \multicolumn{2}{c|}{$E_{GW} = 10^{-2}M_{\odot}c^2$} &
  \multicolumn{2}{c|}{}
  & Number &
  \multicolumn{2}{c|}{\% BNS Localized} \\
   & Run & 
  \multicolumn{2}{c|}{Burst Range (Mpc)}&
  \multicolumn{2}{c|}{BNS Range (Mpc)} &
  of BNS &
  \multicolumn{2}{c|}{within} \\
  Epoch & Duration & 
  LIGO & Virgo &
  LIGO & Virgo &
  Detections &
  5 deg${}^2$ & 20 deg${}^2$ \\ \hline
2015 & 3 months & 40 -- 60 & --- & 40 -- 80 & --- & 0.0004 -- 3 & --- & --- \\ 
2016-17 & 6 months & 60 -- 75 & 20 -- 40 & 80 -- 120 & 20 -- 60 & 0.006 -- 20 & 2 & 5 -- 12 \\
2017-18 & 9 months & 75 -- 90 & 40 -- 50 & 120 -- 170 & 60 -- 85 & 0.04 -- 100 & 1 -- 2 & 10 -- 12 \\
2019$+$ & (per year) & 105 & 40 -- 70 & 200 & 65 -- 130 & 0.2 -- 200 & 3 -- 8 & 8 -- 28 \\
2022$+$ (India) & (per year) & 105 & 80 & 200 & 130 & 0.4 - 400 & 17 & 48 \\ \hline
\end{tabular}
\caption{LVC Observation Runs Projection}
\label{ligoruns}
\end{table}



\begin{comment}
%The successful operation of Advanced LIGO is expected to transform the field from GW detection to GW astrophysics. We illustrate the potential using compact binary coalescences. Detection rate estimates for CBCs can be made using a combination of extrapolations from observed binary pulsars, stellar birth rate estimates and population synthesis models. There are large uncertainties inherent in all of these methods, however, leading to rate estimates that are uncertain by several orders of magnitude. We therefore quote a range of rates, spanning plausible pessimistic and optimistic estimates, as well as a likely rate. For a NS mass of 1.4M⊙ and a BH mass of 10M⊙, these rate estimates for Advanced LIGO are: 0.4– 400 yr−1, with a likely rate of 40 yr−1 for NS–NS binaries; 0.2–300 yr−1 , with a likely rate of 10 yr−1 for NS–BH binaries; 2–4000 yr−1 , with a likely rate of 30 yr−1 for BH–BH binaries.

LIGO was designed so that its data could be searched for GWs from many different sources. The sources can be broadly characterized as either transient or continuous in nature, and for each type, the analysis techniques depend on whether the gravitational waveforms can be accurately modeled or whether only less specific spectral characterizations are possible. We therefore organize the searches into four categories according to source type and analysis technique.
%(i) Transient, modeled waveforms: the compact binary coalescence search. The name follows from the fact that the best understood transient sources are the final stages of binary inspirals [52], where each component of the binary may be a NS or a BH. For these sources the waveform can be calculated with good precision, and matched-filter analysis can be used.
(ii) Transient, unmodeled waveforms: the gravitational-wave bursts search. Transient systems, such as core-collapse supernovae [53], BH mergers and NS quakes, may produce GW bursts that can only be modeled imperfectly, if at all, and more general analysis techniques are needed.
(iii) Continuous, narrow-band waveforms: the continuous wave sources search. An example of a continuous source of GWs with a well-modeled waveform is a spinning NS (e.g. a pulsar) that is not perfectly symmetric about its rotation axis [54].
(iv) Continuous, broadband waveforms: the gravitational-wave background search. Processes operating in the early universe, for example, could have produced a background of GWs that is continuous but stochastic in character [55].
\end{comment}


\section{The TOROS Project}

In 2011, scientists from the Center for Gravitational Wave Astronomy (CGWA) at 
The University of Texas at Brownsville (UTB) 
and the Observatorio Astronomico de Cordoba (OAC) and Instituto de Astronomia Teorica y Experimental (IATE) (the last two in Argentina)
established the TOROS project
to scan likely areas of the localization uncertainty sky map of the GW trigger, looking for possible optical counterparts.
%to respond to the LVC GW triggers with a wide field search for possible optical counterparts in likely areas of the localization uncertainty sky map.

%Motivated by complementing the LIGO observations in the electromagnetic (EM) side, Dr. Mario Diaz from UT RGV founded the TOROS project. TOROS (the Transient Optical Robotic Observatory of the South), has the main goal of scanning the uncertainty region of the LIGO localization map for GW events, with a wide field telescope and a large CCD pixel camera, searching for EM transients candidates for counterparts for such GW event.

TOROS stands for Transient Optical Robotic Observatory of the South,
and its original project consisted in the construction of an observatory site in Cordon Macon,
a mountain top at 4637 m above sea level, in the Andes mountain range in the north of Argentina.

Cordon Macon is a location with high quality seeing and excellent photometric quality.
The mean and median of the seeing measurements obtained at Macon are 0.70'' and 0.55'', respectively (Renzi et al., 2009). 
Several sites within the area around Cordon Macon were considered for the location of the European Extremely Large Telescope, including a site on the Macon Ridge itself.
%It will also link the network of observatories to the sky in the Southern hemisphere.

The proposed TOROS telescope would have a 0.6 m aperture and a 9.85 sq. deg. field of view.
When fully operational it is planned to have three basic modes of operation:
follow up of GW triggers; 
follow up of gamma-ray burst triggers from Fermi, Swift, and other missions; 
and baseline imaging of the entire surveyable area.

It is also included in the full project the inclusion of a data reduction and processing pipeline for transient detection, as well as a database for the dissemination of catalogs and triggers.

While the original project remains on hold awaiting for proper financing of the facility, several other institutions showed interest to participate as well.
This broadened the TOROS project into a wider collaboration of telescopes instead of the single telescope in Argentina that was originally envisioned.

The collaboration has now partners in Mexico, and the Gemini spectrograph as well as the telescopes in Chile and Cordoba previously mentioned. \textcolor{red}{Use this paragraph to include all the participating institutions}

Whether it be the original design or the new dynamic proposed by the enlargement of the collaboration, the interesting targets to TOROS remain to be mergers with at least one Neutron Star (NS), because of its several radiation messengers.

Black Hole-Neutron Star (BH-NS) or Binary Neutron Stars (BNS) mergers are among the expected events detectable by the LVC. 
These highly energetic events will emit GW radiation in the frequency range of LIGO and Virgo sensitivity, strong enough to be detected up to a few hundreds Mpc of distance.
The precise maximum detection distance depends mainly of the masses involved in the merger, as well as other geometrical and spin parameters, but it sits at around 400 Mpc.

BNS and BH-NS mergers have long been proposed as the process leading to short-hard gamma-ray bursts (SGRBs) (Eichler et al., 1989; Narayan et al., 1992), but unfortunately this emission is beamed and thus can only be observed in a small fraction of the events.

Mergers with NS are also predicted to be accompanied by a more isotropic EM counterpart, commonly known as a `Kilonova'. Kilonovae are day to week-long thermal, supernova-like transients, which are powered by the radioactive decay of heavy, neutron-rich elements synthesized in the expanding merger ejecta (Li \& Paczynski 1998). 

%Lattimer \& Schramm in 1974, were the first to propose a BH-NS merger as a suitable environment for the r-process and its radiation emission.

%Needless to say, is that Kilonovas are an ideal counterpart to the LIGO observations.
%We detail the reasons in the following paragraphs.

Kilonova emission is an ideal EM counterpart to the LIGO observations for the reasons we summarize in the following paragraphs.

As previously said, unlike SGRBs, Kilonova emission is isotropic. 
Even though short GRBs emission, is also present during the BNS merger, the r-process emission in the jet and other merger neutron ejecta, is fairly isotropic. The chances to detect the GRB jet are very limited by the collimation of the jet and our line of sight, but r-process radiation is not. 

The Kilonovae are bright.
It was based on their derived peak luminosities, of approximately 1,000 times brighter than a nova, that Metzger et al. (2010) first introduced the term `kilonova' to describe this EM counterparts.

Another fundamental element that the multi-messenger astronomy contributes with,
is the identification of host galaxies.
The inference of the distance to the GW event using redshift of host galaxy will greatly reduce degeneracy in the GW parameter estimation,
especially of the binary inclination with respect to the line of sight.
It will also give a better estimate of the energies involved in the merger.
Other interesting environment properties can be derived from the Kilonova detection, like age of the stellar population and possible displacements due to SN birth kicks.

But most importantly, the merger of a binary system involving a NS is very complex and several kind of factors can effect the radiation pattern and light-curve of the Kilonova as well as the GRW waveform. The optical light curve can serve as a probe into the core and shed light to the intricacies and details of the merge process.

For a more detailed explanation of the Kilonova and a possible detection in the infrared, please refer to chapter \ref{kilonovachapter}.

The next section is dedicated to the LVC O1 campaign and the TOROS participation on it.

\subsection{The O1 LIGO Campaign}

On April 2014, TOROS signed a Memorandum Of Understanding (MOU) with the LIGO Virgo Collaboration, in order to participate in a program to perform follow-up observations of GW candidate events with the benefit of access to LVC proprietary information.

The MOU signing was a very important step that allowed TOROS to participate in the O1 LIGO campaign with positive results.
The MOU allowed TOROS to receive the LIGO alerts, which are private and confidential at the time of O1 and O2, and to be part of a publication as part of a broader collaboration with other institutions for the electromagnetic counterpart search of the first announced GW ever, namely GW150914.

Besides LIGO and Virgo Collaborations, other 24 institutions participated in the GW150914 event along with TOROS. These are detailed in table \ref{gw150914participants}.

\begin{table}
\centering
%\begin{tabular}{|p{14cm}|c|}
\begin{tabular}{|l|c|}
\hline
Institution name & Band \\ \hline
The LIGO Scientific Collaboration & GW \\
The Virgo Collaboration & GW \\
The Australian Square Kilometer Array Pathfinder (ASKAP) Collaboration & \\
The Bootes Collaboration & \\
The Dark Energy Survey And The Dark Energy Camera GW-EM Collaborations & \\
The Fermi GBM Collaboration & $\gamma$-rays \\
The Fermi LAT Collaboration & $\gamma$-rays \\
The Gravitational Wave Inaf Team (Grawita) & \\
The Integral Collaboration & \\
The Intermediate Palomar Transient Factory (iPTF) Collaboration & Optical \\
The Interplanetary Network & \\
The J-Gem Collaboration & \\
The La Silla-Quest Survey & Optical \\
The Liverpool Telescope Collaboration & \\
The Low Frequency Array (LOFAR) Collaboration & Radio \\
The Master Collaboration & \\
The Maxi Collaboration & \\
The Murchison Wide-field Array (MWA) Collaboration & \\
The Pan-STARRS Collaboration & \\
The PESSTO Collaboration & Optical \\
The Pi Of The Sky Collaboration & \\
The SkyMapper Collaboration & \\
The Swift Collaboration & $\gamma$-rays \\
The Tarot, Zadko, Algerian National Observatory, And C2PU Collaboration & \\
The TOROS Collaboration & Optical \\
The Vista Collaboration & \\ 
\hline
\end{tabular}
\caption{Participating Institutions in the GW150914 EM Counterpart Search}
\label{gw150914participants}
\end{table}

Several other institutions showed interest in the project after this achievement and since then the collaboration has grown.

TOROS successfully participated of this campaign, where the first GW was detected, GW150914, and other two events with lower intensity were also detected, one in December ...

Surprisingly, the first of them, GW150914, was detected just before O1 officially began. It was during an engineering run. Despite the premature nature of the alert, TOROS could scan a region of the LIGO localization map and contribute with other XX observatories to the EM search. The search did not show up any transient associated with the event.

On the night of September 16, 2015, an LIGO alert for a GW arrived through the TOROS alert receiver system.

We conducted unfiltered CCD observations (0.35--1$\mu$m) with the 1.5-m telescope at Estacion Astronomica Bosque Alegre (EABA) telescope.

Because of the unexpected timing of the event, the alert was received 2 days later, and our observations started about 2.5 days after the alarm was received by LIGO.

This unexpected detection --observed four days before the first scientific run of the detectors was scheduled to start-- constituted the first detection of the merger of a binary black hole (BBH) system and the first direct detection of gravitational waves. Due to the unexpected timing of the event, LVC provided spatial location information two days later, in the form of probability sky maps via a private GCN circular (Singer 2015, GCN\#18330). TOROS was one of 25 teams that participated in the search for an electromagnetic counterpart search in the southern hemisphere.

On 2015 September 16, the LIGO Virgo Collaboration (LVC) provided two all-sky localization probability maps for the event, based on them.
Both, the coherent Wave Burst (cWB; Klimenko et al. 2016) and the Omicron+LALInference Burst (oLIB; Lynch et al. 2015) search for unmodeled signals. The first one, a rapid localization analysis just searches for coherent power across both detectors while the second one, more refined, assumes a Sine-Gaussian content. The maps provided initial spatial localization of 50\% and 90\% confidence regions encompassing about 200 and 750 square degrees, respectively (Singer 2015, GCN\#18330).

We started our imaging campaign immediately after
receiving, on the night of 2015 September 16, utilizing the
cWB map. Additional observations were obtained the
following night, and a second epoch of imaging was acquired on 2015 December 5 \& 6. We used an Apogee Alta
U9 camera with a field of view (FoV) of 12.'7 $\times$ 8.'5 and
an effective plate scale of 0.''75pix after 3 $\times$ 3 binning. Since we wished to maximize our sensitivity, we conducted unfiltered (``white light'') observations spanning 0.35 < $\lambda/\mu$m < 1. We obtained individual exposures of 60 s with a median seeing (FWHM) of (2.8 $\pm$ 0.6)''. 
We typically obtained 10 images per field, reaching 5$\sigma$ limiting magnitudes of r = 21.7 $\pm$ 0.3 mag.

The LIGO localization regions span several hundred square degrees (see Fig. 1) and vary depending on the algorithm. For instance, the 90\% credible localization area for cWB covers to 310 square degrees while others span up to 750 square degrees (see table 1 in Abbott et al. 2016c). Regardless, all sky maps are consistent with a broad long arc in the Southern hemisphere and a smaller extension in the Northern hemisphere. The algorithm utilized for the CWB estimations produces reasonably accurate maps for BBH signals, but underestimates the extent of high-confidence regions (Essick et al. 2015). As seen in Fig. 1, the adoption of maps from alternative algorithms (not available at the time our observations started) significantly reduces the fraction of the high-confidence region probed by our small FoV.

Despite the little area covered by TOROS, the O1 campaign allowed the collaboration to test the detection and response systems to the alerts.

\begin{table}
\centering
\begin{tabular}{|*{7}{c|}}
  \hline
 Date & GWGC & RA & Dec & $t_{exp}$ & Tile & D \\ 
 (Local Time) & ID & [Deg] & [Deg] & [s] & Number & [Mpc] \\ \hline
2015-09-16 & IC1933 & 51.416101 & -52.78547 & 600 & 1,2,3,4 & 17.45 \\
2015-09-16 & NGC1529 & 61.833301 & -62.89993 & 600 & 5,6,7,8 & 54.76 \\
2015-09-16 & IC2038 & 62.225246 & -55.99074 & 600 & 9,10,11,12 & 7.00 \\
2015-09-16 & IC2039 & 62.259901 & -56.01172 & 600 & 9,10,11,12 & 7.63 \\
2015-09-17 & ESO058-018 & 102.593850 & -71.03123 & 1020 & 13 & 52.23 \\
2015-09-17 & ESO084-015 & 65.550449 & -63.61097 & 1140 & 14 & 14.99 \\
2015-09-17 & ESO119-005 & 72.072451 & -60.29376 & 1080 & 15 & 9.73 \\
2015-09-17 & NGC1559 & 64.398901 & -62.78358 & 900 & 16 & 12.59 \\
2015-09-17 & PGC016318 & 73.728898 & -61.56747 & 1020 & 17 & 9.54 \\
2015-09-17 & PGC269445 & 100.209150 & -71.33026 & 1140 & 18 & 54.83 \\
2015-09-17 & PGC280995 & 96.382499 & -69.15257 & 1140 & 19 & 55.08 \\
2015-09-17 & PGC128075 & 64.859998 & -60.53844 & 720 & 20 & 63.71 \\
2015-09-17 & PGC381152 & 63.584547 & -58.20726 & 1200 & 21 & 13.26 \\
2015-09-17 & PGC075108 & 63.670349 & -58.13199 & 1200 & 21 & 13.29 \\ \hline
\end{tabular}
\caption{Targeted host galaxies}
\label{o1targets}
\end{table}

\begin{figure}[!t]
\centering
\includegraphics[scale=0.5]{pointings}
\caption{cWB, LIB, BYST, LALinf Sky-maps re-scaled regions that mark TOROS targets (red dots).}
\label{fig:pointings}
\end{figure}





\begin{comment}

With LIGO we have just pierced a new window in the purely relativistic universe.

LIGO is sensitive to phenomena in the 10Hz to 7kHz frequency band.

Similarly to the optical case, once other frequencies were explored more science could be done, with other detection methods like PTA (Pulsar Timing Array) and LISA will shed new light to other physical phenomena and physics.

Two things brought about radical changes in Astronomy: Multi-Messenger Astronomy and Time Domain Astronomy.

Sky location is ligo is very poor because it's based mainly on triangulation of detectors.
	Sky error regions are very large (e.g. $\approx$ 850 deg2 for GW150914; Abbott et al. 2016).
	With Virgo and KAGRA and INDIGO will be of of 10-100 square degrees or less (e.g., Fairhurst 2011, Nissanke et al. 2013, Rodriguez et al. 2014). 
	It still greatly exceeds the fields of view of most radio, optical, and X-ray telescopes.

Identifying host galaxies of GW is important. We can know:
	Age of stellar population.
	Displacement due to SN birth kicks.
	Determine distance to GW source. This reduces degeneracies in the GW parameter estimation, especially of the binary inclination with respect to the line of sight.

Based on their derived peak luminosities being approximately one thousand times brighter than a nova, Metzger et al. (2010) first introduced the term `kilonova' to describe the EM counterparts of NS mergers powered by the decay of r-process nuclei

It is known that short GRBs are basically NS-NS mergers.
GRB emission is collimated and jetted, so the chances to detect one along a GW are very low.
NS-NS also emit EM radiation of lower energy or frequency by a different process that occurs in the jet but is nonetheless isotropic. 

Observational (e.g., Fong et al. 2013) and theoretical (e.g. Eichler et al. 1989, Narayan et al. 1992) evidence suggest a relation between merges with at least one NS and the ``short duration'' class of GRBs (Nakar 2007, Berger 2014). 


[literal] Short GRBs are commonly believed to be powered by the accretion of a massive remnant disk onto the compact BH or NS remnant following the merger. This is typically expected to occur within seconds of the GW chirp, making their temporal association with the GWs unambiguous (the gamma-ray sky is otherwise quiet).

For the majority of GW-detected mergers, the jetted GRB emission will be relativistically beamed out of our line of sight.
The off-axis afterglow probably does not provide a promising counterpart for most observers

critical four-way connection between kilonovae, short GRBs, GWs from NS-NS/BH-NS mergers, and the astrophysical origin of the r-process nuclei. Metzger et al. (2010)

NS-NS/BH-NS mergers are also predicted to be accompanied by a more isotropic counterpart, commonly known as a `kilonova'. Kilonovae are day to week-long thermal, supernova-like transients, which are powered by the radioactive decay of heavy, neutron-rich elements synthesized in the expanding merger ejecta (Li \& Paczynski 1998). They provide both a robust EM counterpart to the GW chirp, which is expected to accompany a fraction of BH-NS mergers and essentially all NS-NS mergers, as well as a direct probe of the un- known astrophysical origin of the heaviest elements (e.g., Metzger et al. 2010).

The most significant of those is no doubt, the Kilonova emission produced by rapid capturing of neutrons.
Neutron capture has to be faster than the beta decay rate of the neutron and that's why it has to be rapid.
This capture process is called r-process. R is for rapid.
The r-process physics is quite complicated and involves a bunch of stuff, much of which is modeled to certain confidence, but many other elements are not well known. Several ingredients to the model are not considered fully. 

Blinnikov et al. (1984) and Paczynski (1986) first suggested a connection be- tween NS-NS mergers and GRBs.

Even prior to the discovery of the first binary pulsar (Hulse \& Taylor 1975), Lattimer \& Schramm (1974, 1976) proposed that the merger of compact star binaries --in particular the collision of BH-NS systems-- could give rise to the r-process by the decompression of highly neutron-rich ejecta (e.g. Meyer 1989). 

As compared to the earlier predictions (e.g. Metzger et al. 2010), these higher opacities push the bolometric light curve to peak later in time (1 week instead of a 1 day timescale), and at a lower luminosity (Barnes \& Kasen, 2013). More importantly, the enormous optical opacity caused by line blanketing moved the spectral peak from optical/UV frequencies to the near-infrared (NIR).

BH-BH mergers have no EM counterpart, except perhaps in very specific situations. This is mainly due to lack of baryonic matter.


TOROS, the Transient Optical Robotic Observatory of the South, is a collaboration formed to respond to GW events as detected by the LVC.
Its main purpose then is to detect transient events compatible with expected (or not) signatures of events that can give mutual birth to GW and optical triggers.

Being in its early stages of development, and given the new nature of the multi-messenger astronomy, the TOROS team is building up the tools and infrastructure that will make it capable in the future to promptly respond to this alerts.

Several things can enter in consideration for this task, from software development to forging ties with existing observatories. 

My thesis will focus on several challenges in the development of the software infrastructure needed to process the observatory data. 

At the moment of this writing, TOROS Collaboration consists of several astronomical institutions that showed interest in doing a search and possible follow-ups of candidates to interesting events.

The list of participant institutions is as follows: (should I list them?)

At UT Rio Grande Valley, we developed extensive analysis and web code to allow for the interaction between the institutions (just the broker page, really). 
}

\end{comment}


\chapter{The Kilonova, A Case For Multi-Messenger Astronomy} \label{kilonovachapter}

Short GRBs have long been associated to BNS mergers as their progenitors,
but it was not until the very late 90's that another EM counterpart was found to be expectable from such mergers.

Binary compact mergers where at least one of the merging bodies is a NS, will in most cases produce an EM transient emission, peaking in the NIR and lasting for up to about a week, named Kilonova, based on its derived peak luminosity being approximately one thousand times brighter than a nova. (Metzger et al., 2010)

Li \& Paczynski (1998, LP98) first showed that the radioactive ejecta from a NS-NS or BH-NS merger provides a source for powering transient emission, in analogy with Type Ia SNe. 
Given the low mass and high velocity of the ejecta from a NS-NS/BH-NS merger, they concluded that the ejecta will become transparent to its own radiation quickly, producing emission which peaks on a timescale of about one day, much faster than for normal SNe (which instead peak on a timescale of weeks or longer).

In 1999 Freiburghaus et al. (1999) presented the first explicit calculations showing that the ejecta properties extracted from a hydrodynamical simulation of a NS-NS merger (Rosswog et al. 1999) indeed produced abundance patterns in basic accord with the solar system r-process.

Rapid neutron capture, or r-process for short, is the main driver of the radiation we see during a Kilonova.
In a dense neutron-rich environments, if the neutron capture timescale of lighter seed atom nucleus (like iron) is shorter than the neutron beta-decay timescale, nuclei can capture extra neutrons for themselves.
The newly formed nuclei will later decay emitting radiation.
Burbidge et al. (1957) and Cameron (1957) had already proposed that approximately half of the elements heavier than iron are synthesized in this way.

This EM transient powered by the radioactive heating of the r-process, is what is called a Kilonova.
This radioactive heating occurs through a combination of $\beta$-decays, $\alpha$-decays, and fission of the r-process nuclei (Metzger et al. 2010, Barnes et al. 2016, Hotokezaka et al. 2016)

In the following paragraphs, I provide a more detailed explanation for a Kilonova emission.

During a BNS merger, the neutron-rich ejecta is a favorable environment for rapid neutron capture process.

The BNS ejecta consists mainly of matter ejected either by tidal forces or compression-induced heating at the interface between merging bodies. 
Unbound debris from the merger can also form a disk around the merge and outflows from this disk is another --albeit second in importance-- source of ejecta.

As a first approximation we can imagine the ejecta expanding radially outwards with spherical symmetry, at a certain speed.
Ejecta right after the merger can exceed billions of degrees at the radius of the merger (100 km), but 
during its expansion, the ejecta cools down due to adiabatic expansion.

The thermal radiation cannot initially escape as radiation because of the high optical depth at early times and the correspondingly long photon diffusion timescale through the ejecta. 
Nonetheless, as the ejecta expands, the diffusion time decreases until eventually radiation can escape and the medium becomes transparent to radiation.

Devoid of any other external source of heating, the ejecta would be so cold when it first becomes transparent, that the whole transient would be basically invisible.
The reality is that there are other sources of heating, the most important being the radioactive heating by r-process nuclei decay, that will heat the environment hot enough to be as much as 1,000 times brighter than a Nova.
Other more speculative sources of heating are from within a central engine like a long lived magnetar or an accreting BH. These are discussed separately.

The condition at which ejecta first becomes transparent is crucial since it determines the characteristic timescale at which the light curve peaks.

\begin{comment}

%In physics, GW detection could provide information about strong-field gravitation, the untested domain of strongly curved space where Newtonian gravitation is no longer even a poor approximation. In astrophysics, the sources of GWs that LIGO might detect include binary NSs (like PSR 1913 + 16 but much later in their evolution); binary systems where a black hole (BH) replaces one or both of the NSs; a stellar core collapse which triggers a type II supernova; rapidly rotating, non-axisymmetric NSs; and possibly processes in the early universe that produce a stochastic background of GWs [3].

Based on their derived peak luminosities being approximately one thousand times brighter than a nova, Metzger et al. (2010) 
first introduced the term `kilonova' to describe the EM counterparts of NS mergers powered by the decay of r-process nuclei.

Short GRBs are commonly believed to be powered by the accretion of a massive remnant disk onto the compact BH or NS remnant following the merger. This is typically expected to occur within seconds of the GW chirp, making their temporal association with the GWs unambiguous (the gamma-ray sky is otherwise quiet).

It is known that short GRBs are basically NS-NS mergers.
GRB emission is collimated and jetted, so the chances to detect one along a GW are very low.
NS-NS also emit EM radiation of lower energy or frequency by a different process that occurs in the jet but is nonetheless isotropic. 

Observational (e.g., Fong et al. 2013) and theoretical (e.g. Eichler et al. 1989, Narayan et al. 1992) evidence suggest a relation between merges with at least one NS and the ``short duration'' class of GRBs (Nakar 2007, Berger 2014). 


[literal] Short GRBs are commonly believed to be powered by the accretion of a massive remnant disk onto the compact BH or NS remnant following the merger. This is typically expected to occur within seconds of the GW chirp, making their temporal association with the GWs unambiguous (the gamma-ray sky is otherwise quiet).

For the majority of GW-detected mergers, the jetted GRB emission will be relativistically beamed out of our line of sight.
The off-axis afterglow probably does not provide a promising counterpart for most observers

critical four-way connection between kilonovae, short GRBs, GWs from NS-NS/BH-NS mergers, and the astrophysical origin of the r-process nuclei. Metzger et al. (2010)

NS-NS/BH-NS mergers are also predicted to be accompanied by a more isotropic counterpart, commonly known as a `kilonova'. Kilonovae are day to week-long thermal, supernova-like transients, which are powered by the radioactive decay of heavy, neutron-rich elements synthesized in the expanding merger ejecta (Li \& Paczynski 1998). They provide both a robust EM counterpart to the GW chirp, which is expected to accompany a fraction of BH-NS mergers and essentially all NS-NS mergers, as well as a direct probe of the un- known astrophysical origin of the heaviest elements (e.g., Metzger et al. 2010).

The most significant of those is no doubt, the Kilonova emission produced by rapid capturing of neutrons.
Neutron capture has to be faster than the beta decay rate of the neutron and that's why it has to be rapid.
This capture process is called r-process. R is for rapid.
The r-process physics is quite complicated and involves a bunch of stuff, much of which is modeled to certain confidence, but many other elements are not well known. Several ingredients to the model are not considered fully. 

Blinnikov et al. (1984) and Paczynski (1986) first suggested a connection be- tween NS-NS mergers and GRBs.

Even prior to the discovery of the first binary pulsar (Hulse \& Taylor 1975), Lattimer \& Schramm (1974, 1976) proposed that the merger of compact star binaries --in particular the collision of BH-NS systems-- could give rise to the r-process by the decompression of highly neutron-rich ejecta (e.g. Meyer 1989). 

As compared to the earlier predictions (e.g. Metzger et al. 2010), these higher opacities push the bolometric light curve to peak later in time (1 week instead of a 1 day timescale), and at a lower luminosity (Barnes \& Kasen, 2013). More importantly, the enormous optical opacity caused by line blanketing moved the spectral peak from optical/UV frequencies to the near-infrared (NIR).

BH-BH mergers have no EM counterpart, except perhaps in very specific situations. This is mainly due to lack of baryonic matter.

\section{Physics of the KN}

\subsection{Radioactive radiation in neutron-rich ejecta}

The radioactive decay rate is also largely insensitive to uncertainties in the assumed nuclear masses, cross sections, and fission fragment distribution (although the r-process abundance pattern will be e.g. Eichler et al. 2015; Wu et al. 2016; Mumpower et al. 2016).

Radioactive heating occurs through a combination of $\beta$-decays, $\alpha$-decays, and fission (Metzger et al. 2010, Barnes et al. 2016, Hotokezaka et al. 2016). 

\subsection{isotropic}

Consider the merger ejecta of total mass M, which is expanding at a constant velocity v, such that its radius is R = vt after a time t following the merger. 
We assume spherical symmetry, good first-order approximation

\subsection{r-process}

Burbidge et al. (1957) and Cameron (1957) realized that approximately half of the elements heavier than iron are synthesized via the capture of neutrons onto lighter seed nuclei (e.g., iron) in a dense neutron-rich environment in which the timescale for neutron capture is shorter than the $\beta$-decay timescale.

Freiburghaus et al. (1999) presented the first explicit calculations showing that the ejecta properties extracted from a hydrodynamical simulation of a NS-NS merger (Rosswog et al. 1999) indeed produces abundance patterns in basic accord with the solar system r-process.

In the late 50's Burbidge et al. (1957) and Cameron (1957) had already proposed that approximately half of the elements heavier than iron are synthesized via the capture of neutrons onto lighter seed nuclei (e.g., iron) in a dense neutron-rich environment in which the timescale for neutron capture is shorter than the $\beta$-decay timescale.
Rapid neutron capture process', or r-process for short, 
Despite this mechanism was known for long time, the astrophysical environments in which this happens remained a mystery.
They showed that the radioactive heating rate was relatively insensitive to the precise electron fraction of the ejecta, and they were the first to consider how efficiently the decay products thermalize their energy in the ejecta.

\subsection{ejecta (where? where from?) and opacity}

The ejecta from NS mergers are an astrophysical source of rapid neutron-capture (r-process) 

The type of radiation depends on the EOS for the ejecta and its thermodynamical properties. Opacity, nucleon and particle content, pressure, temperature, MHD state.

Li \& Paczynski (1998, LP98) first showed that the radioactive ejecta from a NS-NS or BH-NS merger provides a source for powering transient emission, in analogy with Type Ia SNe. Given the low mass and high velocity of the ejecta from a NS-NS/BH-NS merger, they concluded that the ejecta will become transparent to its own radiation quickly, producing emission which peaks on a timescale of about one day, much faster than for normal SNe (which instead peak on a timescale of weeks or longer).

Consider the merger ejecta of total mass M, which is expanding at a constant velocity v, such that its radius is R = vt after a time t following the merger. 
We assume spherical symmetry, good first-order approximation
The ejecta is hot immediately after the merger, especially if it originates from the shocked interface between the colliding NS-NS binary. 

This thermal energy cannot, however, initially escape as radiation because of its high optical depth at early times
and the correspondingly long photon diffusion timescale through the ejecta. 
As the ejecta expands, the diffusion time decreases inversely proportional with time, until eventually radiation can escape on the expansion timescale.
This condition determines the characteristic timescale at which the light curve peaks
For values of the opacity $\kappa \approx$ 1 - 100 cm2 g${}^{-1}$ which characterize the range from Lanthanide-free and Lanthanide-rich matter, respectively, the derivation predicts characteristic durations about 1 day to 1 week.

Opacity is crucial since it determines at what time and wavelength the ejecta becomes transparent and the light curve peaks. 

The temperature of matter freshly ejected at the radius of the merger (about 100 km) exceeds billions of degrees. 
However, absent a source of persistent heating, this matter will cool through adiabatic expansion, losing all but a fraction $\approx$ (R0/Rpeak) $\approx$ 1E-9 of its initial thermal energy before reaching the radius Rpeak = vtpeak at which the ejecta becomes transparent.
Such `adiabatic losses' would leave the ejecta so cold as to be effectively invisible.
In reality, the ejecta will be continuously heated by a combination of sources. 
At a minimum, this heating includes contributions from radioactivity due to r-process nuclei (and possibly free neutrons), while, more speculatively, the ejecta can be heated from within by a central engine, such as a long-lived magnetar or accreting BH.

Matter ejected either by tidal forces or due to compression-induced heating at the interface between merging bodies.
Unbound debris can have enough angular momentum to form a disk around the merge.
Outflows from this remnant disk, taking place on longer timescales of up to seconds, provide a second important source of ejecta

In the case of a NS-NS merger, the ejecta properties depend sensitively on the fate of the massive NS remnant which is created by the coalescence event.

\end{comment}

\section{Final fate of NS mergers}

The end product of a NS-NS or BH-NS merger is a central compact remnant, either a BH or a massive NS. 
The last stages of the system will also effect the emission of the kilonova.

The final system depends sensitively on the total mass of the original NS- NS binary (e.g., Shibata \& Uryu 2000; Shibata \& Taniguchi 2006). Above a threshold mass of Mcrit about 2.6 to 3.9 solar masses the remnant collapses to a BH essentially immediately, on the dynamical time of milliseconds or less (Hotokezaka et al. 2011; Bauswein et al. 2013a).

The maximum mass of a NS, though primarily sensitive to the NS EOS, can be increased if the NS is rotating rapidly (e.g., Baumgarte et al. 2000, Ozel et al. 2010, Kaplan et al. 2014). This will result in either a HMSN or a SMNS

\subsection{Prompt BH (here goes all the NS-BH mergers)}

Stellar mass BH-BH binaries are not expected to produce luminous EM emission because there are no baryonic matter.

Besides NS-NS, a NS with a BH is also posible and it will, under some circumstances give raise to a Kilonova. The physics in this case is a bit more complicated and the parameters of the BH play a big role in determining the Kilonova shape or if there's one at all.

\subsection{HMNS -> BH (short duration: ms)}

The mass is supported exclusively by differential rotation it's called a hypermassive NS (HMNS).
This decays into a BH in a few ms.

\subsection{SMNS -> BH (minutes to much longer)}

The mass can be supported by solid body rotation is called a supramassive NS.
This can decay in a BH by a less effective mechanism and can remain stable for minutes or much longer periods.

\subsection{Indefinitely stable NS (see magnetar remnant)}

The merger of a binary with a total mass less than the maximum mass of a non-rotating NS (dependent on the particular EOS but around 2 solar masses), will produce an indefinitely stable remnant, from which a BH never forms (e.g., Metzger et al. 2008; Giacomazzo \& Perna 2013).

\section{Types of Kilonovas}

The variety of sources which contribute to heating the ejecta, particularly on timescales when the ejecta is first becoming transparent.
At a minimum, the ejecta receives heating from the radioactive decay of heavy nuclei synthesized in the ejecta by the r-process. 

\subsection{Red KN}

In the tidal tails in the equatorial plane, or in more spherical outflows from the accretion disk in cases when BH formation is prompt or the HMNS phase is short-lived, the highly neutron-rich matter (Ye < 0.29) will form heavy r-process nuclei.
This r-process will peak in the near infra-red (NIR) at J and K bands (1.2 and 2.2 μm, respectively) on a timescale of several days to a week.

\subsection{Blue KN}

In addition to the highly neutron-rich ejecta (Ye < 0.29), growing evidence suggests that some of the matter which is unbound from a NS-NS merger is less neutron rich (Ye > 0.29; e.g. Wanajo et al. 2014a; Goriely et al. 2015) and thus will be free of Lanthanide group elements (Metzger \& Fernandez 2014). This low-opacity ejecta can reside either in the polar regions, due to dynamical ejection from the NS-NS merger interface, or in more isotropic outflows from the accretion disk in cases when BH formation is significantly delayed.
By assuming a lower opacity appropriate to Lanthanide-free ejecta, the emission now peaks at the visual bands R and I, on a timescale of about 1 day at a level 2-3 magnitudes brighter than the Lanthanide-rich case.

In general, the total kilonova emission from a NS-NS merger will be a combination of `blue' and `red' components, as both high- and low-Ye ejecta components could be visible for viewing angles close to the binary rotation axis (Fig. 4). For equatorial viewing angles, the blue emission is likely to be blocked by the higher opacity of the lanthanide-rich equatorial matter (Kasen et al. 2015). Thus, although the week-long NIR transient is fairly generic, an early blue kilonova will be observed in only a fraction of mergers.

\subsection{Magnetar remnant KN}

As described in §3.1, the type of compact remnant produced by a NS-NS merger (e.g. prompt BH formation, hypermassive NS, supramassive NS, or indefinitely stable NS) depends sensitively on the total mass of the binary relative to the maximum mass of a non-rotating NS, Mmax($\Omega$ = 0). The value of Mmax($\Omega$ = 0) exceeds about 2solar masses (Demorest et al. 2010, Antoniadis et al. 2013) but is otherwise unconstrained13 by observations or theory up to the maximum value about 3solar masses set by the causality limit on the EOS. A `typical' merger of two about 1.3-1.4sm NS results in a remnant mass of about 2.3-2.4sm after accounting for neutrino losses and mass ejection (e.g., Belczynski et al. 2008). If the value of Mmax($\Omega$ = 0) is well below this value (e.g. 2.1-2.2sm), then most mergers will undergo prompt collapse or form hypermassive NSs with very short lifetimes. On the other hand, if the value of Mmax($\Omega$ = 0) is close to or exceeds 2.3-2.4sm, then a order unity fraction of NS-NS mergers could result in long-lived supramassive or indefinitely stable remnants.

If the rotational energy could be extracted in non-GW channels on timescales of hours to years after the merger (e.g., by magnetic dipole radiation), this could substantially enhance the EM emission from NS-NS mergers (e.g. Gao et al. 2013; Metzger \& Piro 2014; Gao et al. 2015; Siegel \& Ciolfi 2016a). However, for NSs of mass Mns   Mmax($\Omega$ = 0), only a fraction of the rotational energy is available to power EM emission, even in principle. This is because the loss of angular momentum that accompanies spin-down results in the NS collapsing into a BH before all of its rotational energy is released.

Nonetheless, there are several mechanisms to extract rotational energy from the indefinitely stable magnetar remnant.
There is plenty literature on the subject that suggests that rotational energy input from a stable magnetar could enhance kilonova emission. The emission is still red in color and peaks on a timescale of 1 to 2 weeks, but the luminosity is greatly enhanced compared to the radioactive case, with peak magnitudes of K $\approx$ 18- 20

\subsection{Enhancing from free neutrons}

In addition to the blue and red components, recent NS-NS merger simulations show that a small fraction of the dynamical ejecta (typically a few percent, or about 1E-4sm) expands sufficiently rapidly that the neutrons do not have time to be captured into nuclei (Bauswein et al., 2013a). This fast expanding matter, which reaches asymptotic velocities v about 0.4-0.5 c, originates from the shock- heated interface between the merging stars and resides on the outermost layers of the polar ejecta. This `neutron skin' can super-heat the outer layers of the ejecta, enhancing the early kilonova emission (Metzger et al. 2015; Lippuner \& Roberts 2015).

\section{Possibly observed KN}

Tanvir \& Metzger

Later that year, Tanvir et al. (2013) and Berger et al. (2013) presented evidence for excess infrared emission following the short GRB 130603B on a timescale of about one week using the Hubble Space Telescope. If confirmed by future observations, this discovery would be the first evidence directly relating NS mergers to short GRBs, and hence to the direct production of r-process nuclei.

\section{Rates}

Rate of GRBs from NS-NS mergers is low, less than once per year all-sky. (e.g. Metzger \& Berger 2012)
We should not expect the first --or even the first several dozen-- GW chirps from NS-NS/BH-NS mergers to be accompanied by a GRB.


Population synthesis models of field binaries predict GW detection rates of NS-NS/BH-NS mergers of about 0.2-300 per year, once Advanced LIGO/Virgo reach their full design sensitivities near the end of this decade (e.g. Abadie et al. 2010, Dominik et al. 2015).
Empirical rates based on observed binary pulsar systems in our galaxy predict a comparable range, with a best bet rate of about 8 NS-NS mergers per year (Kalogera et al. 2004; Kim et al. 2015).



\chapter{Difference Image Analysis} \label{appendix:dia}

Searches for time-varying and position-changing objects are undertaken by comparing a reference image of a particular region of the sky with a second image taken at the moment in which we are interested. Ideally, the reference image is taken with the same telescope using the same filter and CCD. The two images have to be aligned pixel by pixel and then subtracted to reveal any changes in light.

To make a suitable subtraction of two images, one has to match the frames to exactly the same seeing. Image Difference is a technique to find a convolution kernel that best describes (in some minimization sense) the change in point spread function between the images. The idea is to degrade the good seeing image---our reference image---to match the seeing of our second image. Finding the proper kernel can be a delicate operation and there's plenty of literature on the subject. Methods range from PSF modeling through common Gaussian profiles to unmodeled PSF's to Information Theory and Fourier Domain.

The first attempts at image subtraction relied on Fourier decomposition of the images, 
but the technique suffered when noise levels were even moderate and the results were not always good.
\citet{1998ApJ...503..325A} were the first to propose a solution in image space (as opposed to Fourier space). 
They also summarize previous efforts in their introduction and references therein. 
In that paper, they propose an optimization problem that we describe briefly as follows.

We have an image $I$ and a reference image $R$, for which we want to find a convolution kernel $K$ such that

\begin{align}
I(x,y) & \approx (R \mathbin{*} K)(x,y) \\
 & \approx \int \mathrm{d}u \mathrm{d}v {R (u,v) K(x-u,y-v)}
\end{align}
 
The integral symbol has to be understood in practice, as a sum over all the pixels $(x,y)$ on the image. 

The kernel $K$ will try to correct for the PSF difference between the two images.
It's worth noting here, that even though it could be practical to the reader to think so, the kernel $K$ is neither the PSF of the reference nor the PSF of the image.

Usually, the reference image is the one with the best {\em seeing}, 
because it can be done by median-stacking good images, or using Lucky Imaging [add ref] or some other method.

To convert the problem into a linear one, we decompose the kernel into a linear combination of ``{\em basis}'' functions $B$.

\begin{equation} \label{kernel_linear}
K(u,v) = \sum_{i} a_{i} B_{i}(u,v)
\end{equation}

These $B_{i}$ could in principle be any reasonable set of functions. 
%The two most popular choices are modulated Gaussians and the Delta basis, which will be explained in further detail in section \ref{basis}.

Using this linear combination, the convolution will be

\begin{align}
(R \mathbin{*} K)(x,y) & = \sum_{i} a_{i} \left( R \mathbin{*} B_{i} \right)(x,y) \\
 & \equiv \sum_{i} a_{i} C_{i}(x,y)
\end{align}

where the last line defines $C_{i}(x,y)$.

With this decomposition, we can find the set of $a_{i}$ that minimizes the square difference over all pixels.
Define a cost function $Q$:

\begin{align}
Q &= \int \left( I(x,y) - (R \mathbin{*} K)(x,y) \right)^2 \\
 & = \int \left( I(x,y) - \sum_{i} a_{i} C_{i}(x,y) \right)^2,
\end{align}

and let's minimize $Q$ over the set of $a$'s:

\begin{align}
\frac{\partial Q}{\partial a_{i}} = & 2 \int \left( I(x,y) - \sum_{j} a_{j} C_{j}(x,y) \right) C_{i}(x,y) 
\end{align}

Setting the last equation to zero, gives us:

\begin{align}
\sum_{j} a_{j} \int \left( C_{j}(x,y)  C_{i}(x,y) \right) =  \int I(x,y) C_{i}(x,y) 
\end{align}

Which we can write more succinctly as a matrix equation:

\begin{align} \label{matrix_eq}
\sum_{j} M_{ij} a_{j}  =  b_{i}
\end{align}

where

\begin{align} \label{matrix_def}
M_{ij}  &=   \int  C_{i}(x,y)  C_{j}(x,y) \\
b_{i} &=  \int I(x,y) C_{i}(x,y)  \nonumber
\end{align}

We can find the coefficients $a_{i}$ of the optimal kernel for the subtraction by inverting the system \eqref{matrix_eq}.

\section{Different basis functions}

As stated above, any choice of basis in the linearization \eqref{kernel_linear} could work, but two particular choices are the most popular.

The first one was proposed by \citet{1998ApJ...503..325A} and it consist of modulated centered Gaussians:

\begin{equation}
B_{n,d_n^{x}, d_n^{y}}(u,v) = e^{-(u^2+v^2)/2 \sigma_n^2} \times u^{d_n^{x}} v^{d_n^{y}}
\end{equation}

where the exponents in $u$ and $v$ add at most up to $D$, the degree of the modulation polynomial.

This choice of linearization gives enough freedom to approximate the kernel as a sum of modulated Gaussians, which is suitable for many situations.
The parameter $\sigma_n$ in each Gaussian is fixed beforehand by the user.

The number of Gaussians used in the expansion is given by $n$, and for each one of them we have $(D_n + 1)(D_n + 2)/2$ terms in the modulating polynomial.
That gives a total of $n(D_n + 1)(D_n + 2)/2$ unknown $a_i$ coefficients to solve for.

A simpler basis was proposed by \citet{2008MNRAS.386L..77B} and it's effectively a Dirac Delta function for each pixel.

\begin{equation}
B_{i}(x,y) = \delta(x-i,y-j)
\end{equation}

This choice of basis makes every pixel value in the kernel be determined independently by the minimization process.
Obviously, this allows for a greater variety of kernels than in the Gaussian case.
It also comes at a cost: now the number of unknowns to invert for grows quadratically with the kernel side length.
For a kernel of side 11 pixels, we have to create a matrix 121$\times$121 and then invert it. 

As we will see at the end of the chapter, even calculating the elements of the matrix to invert is an expensive operation,
so this absolute freedom in the kernel shape comes at the cost of a much higher numerical complexity.

Despite this complexity, the Delta basis can account for situations that the Gaussian basis can't address.
For example, if the images are very similar in PSF, then the compensating kernel of our problem should be an actual delta at the center 
---the identity kernel---or a very peaky function.

Bramich's method can actually return the correct kernel, while the Gaussian method will have trouble adjusting (potentially broad) Gaussians.

Another issue that Delta basis corrects very well is for tiny misalignments between our reference image and the image we are processing.
In fact, translations are included in the set of convolutions represented by displaced deltas on the kernel.
Convolving with a kernel with a delta displaced $(\Delta x, \Delta y)$ pixels away from the center will effectively translate the image by that same amount,
so misalignments due to small translations (the order of the kernel side) can be completely accounted for in the Delta basis.

\section{Add a varying background}

Background variation can be treated separately or simultaneously with the PSF fitting.

\subsection{Independent background estimation}

An independent background estimation can be done on each separate image before doing the PSF match.

For a stellar field image, one can create an image $I_{B}$, from the image $I$ by excluding all pixels above a certain threshold on the background noise. This image $I_{B}$ will contain pixels belonging to the background only (sources removed).

\begin{equation}
I_{B}(x,y)  = \left\{ I(x,y) : |I(x,y) - \mu| < \sigma \right\}
\end{equation}

On this image $I_{B}$, find the best polynomial fit of degree $d$ to the image using a least square fit.

Minimizing $Q$ over the $b_{ij}$ coefficients

\begin{equation}
Q = \int \left( I_{B}(x,y) - \sum_{i,j}^d b_{ij} x^i y^j \right) ^2
\end{equation}

will give the best polynomial fit to the background.

\subsection{Simultaneous PSF and background estimation}

To do it simultaneously with the PSF matching, we simply add it to our previous $Q$. This will let us remove any remaining variation on our new image $I(x,y)$.

\begin{align}
I(x,y) & \approx (R \mathbin{*} K)(x,y) + B(x,y) 
\end{align}

Note however, that this $B$ is not the sourceless image $I_{B}$ defined in the previous section. This $B$ represents the optimal background approximation that can be expanded as a polynomial with unknown coefficients.

$Q$ is defined now as

\begin{align}
Q &= \int \left( I(x,y) - (R \mathbin{*} K)(x,y) - B(x,y) \right)^2 \\
 & = \int \left( I(x,y) - \sum_{i} a_{i} C_{i}(x,y) - \sum_{i} b_{i} x^i y^j \right)^2
\end{align}

We can pile up the $b$ coefficients onto a larger set of $a$'s and define new $C$'s accordingly.

\begin{equation}
C_{i}(x,y)  = \begin{cases} 
(R \mathbin{*} K)(x,y)  &\mbox{when i,j refer to kernel}  \\ 
x^i y^j & \mbox{when i,j refer to background}  
\end{cases} 
\end{equation}

This leads us to the exact same (but extended) solution for the coefficients $a$, namely:

\begin{align}
\sum_{j} M_{ij} a_{j}  =  b_{i}
\end{align}

where

\begin{align}
M_{ij}  &=   \int  C_{i}(x,y)  C_{j}(x,y) \\
b_{i} &=  \int I(x,y) C_{i}(x,y) 
\end{align}

as before.

\section{How to deal with bad pixels} \label{badpixels}

The astronomical images can have pixel defects due to CCD defects or missing data from alignment. 
Those pixels can, in principle, be in both the reference frame $R$ and the new image $I$. 

Although it is advisable to use a reference image $R$ with few to none bad pixels for the reason that will be explained further on this section.

As before, we now want to approximate the two images as follow

\begin{align}
I(x,y)\bigg|_{\Omega} & \approx (R \mathbin{*} K)(x,y)\bigg|_{\Omega}
\end{align}

But this time, we want to restrict the above only for the good pixels in the images.

Bad pixels in $I$ and $R$ have to be excluded, but we also have to exclude those pixels in $R$ that {\em use} bad pixels in the convolution. This equivalent to dilate the bad pixels mask in $R$ with a kernel the same shape as the convolution kernel $K$. For this reason, it's better to have few bad pixels in $R$.

The union of the bad pixel mask from $I$ and the dilated bad pixel mask from $R$ is the common bad pixel mask. We call $\Omega$ to its complement, so that $\Omega$ is the set of good pixels that will be used in the subtraction. Pixels outside $\Omega$ will be tainted by the bad pixels in either of the two images.

This modification will now makes us define a new $Q$:

\begin{align}
Q = \int_{\Omega} \left( I(x,y) - (R \mathbin{*} K)(x,y) - B(x,y) \right)^2
\end{align}

That only differs from the previous one by the domain of integration (or sum).
The definitions for $M$ and $b$ are similarly derived:

\begin{align}
M_{ij}  &=   \int_{\Omega}  C_{i}(x,y)  C_{j}(x,y) \\
b_{i} &=  \int_{\Omega} I(x,y) C_{i}(x,y) 
\end{align}

\section{Dealing with large fields of view}

When dealing with large field of views, a simple solution suggested by Bramich 2010, is to partition the image into grids, and apply Image Differencing on each grid element.

Another solution is to include in the derivation, a space-varying kernel.
This has the advantage of addressing the issue directly, but we have to quit to the niceties of having convolutions done with FFT.

The derivation follows the usual least squares derivation, except this time we consider each kernel basis element modulated by a polynomial variation across the image. Following Miller (2008):

\begin{align}
K(u, v) &= \sum_n a_n(x,y) B_n(u, v) \\
&= \sum_{n,i,j} a_{nij} x^i y^j B_n(u, v)
\end{align}

The new equations for $M$ and $b$ as in \ref{matrix_def} are the same, except that $C_{i}$ is now defined as:

\begin{equation}
C_I = \left( R \mathbin{*} x^i y^j B_n \right) (x,y)
\end{equation}

and $I$ is here the collective index $\{n,i,j\}$.
Notice that the last equation involves an unusual type of ``convolution'' where the kernel is not constant. This prevents the use of FFT to speed up the calculation.

\section{Cost of building matrix M and b}

Recall that we have to solve the system of equations \eqref{matrix_eq} with definitions in \eqref{matrix_def}.

This implies that each component of the matrix $M$ will involve a convolution and an integration over the whole image.

Therefore, not only the inversion problem is expensive, but even calculating the matrix can be an expensive operation.



\chapter{Machine Learning}

The subtraction techniques are very effective at  modeling PSF differences, nevertheless there are many defects left behind after a subtraction. This is also a well known issue in the difference image analysis. The fictitious (or bogus) sources arise primarily from PSF mismatch and mis-alignments of the images.

These defects can easily confuse algorithms of source detection like SExtractor or similar that relies on excess of flux over the background. For that, it is needed to have a computerized agent capable of discriminating bogus sources from real transients on the subtracted image. 

This is the task of a Machine Learning (ML) agent trained for that purpose. The real bogus classifier, as it is named in the literature relies on --usually morphological-- features to perform the classification.

Machine Learning is such an extensive and intensive area of research with many applications in many fields.

Since the ML field is so wide, to conduct a ML experiment one must make a few decisions. Even most importantly than the particular algorithm, for which there are hundreds and modifications are being published all the time, it's the data that drives the efficiency of the method.

As the saying goes:

\begin{quotation}
Big data will beat a good algorithm \textcolor{red}{Find exact quote and author}
\end{quotation}


For our main test, I decide to train a Random Forest ML algorithm on a training set developed especially for this.

The Random Forest algorithm is an `ensemble' method. Meaning it is a collection of other methods whose results will be averaged over somehow.

In this case, a Random Forest is an `ensemble' of Decision Trees. A Decision Tree is basically a tree of nodes. Each node represent a bifurcation based on one feature. The features and branching threshold on each node are chosen maximizing the expected information gain (or entropy loss) on the training set. That is, how well the bifurcation separates the training data for each class.

Random Forest bootstraps the data so that each subset of data will train a Decision Tree on a random subset of the features. After all Decision Trees are trained this way, the collection of Trees will give the final classification.

A single Decision Tree suffers of great variance, even for big amounts of data. The bifurcations on the greedy algorithm are naturally very dependent on the training set. A small variation on it may create a complete different tree. Random Forest reduces this variance by averaging over many trees. This also creates more realistic and fuzzy class boundaries on the feature space.

On the next section I explain how the training data was prepared.

\section{Training Data}

The main challenge to generate training data is the great unbalance between bogus sources due to imperfect subtraction and actual real transients.

On a typical image, the number of bogus sources can be of a hundred, and the rate of real transients can be only a handful.

A very unbalanced training set can bring about spurious results. As an example, imagine a training set with 90 bogus and 10 reals. A classifier that classifies everything as bogus, would have a success rate for bogus classification of 90\%. An unbalanced training set could also create classifiers with large variance due to the small set of reals.

To generate a balanced training set with comparable amounts of reals and bogus, I generate reals by selecting sources on an image and erasing them on the reference. That way, after a subtraction, those sources will appear as transient events, that is, objects in the image that don't have a corresponding object on the reference.

The image set used for this purpose is CSTAR, but any other set will also do.

The advantage of this method, compared to other methods like injecting fake sources, is that the `transients' obtained this way will have all the particular characteristics of the CCD and instrument from the image set in which we are interested to work. It will also have the characteristics of the subtraction method imperfections, but this is not exclusive to this method.

The `bogus' set is also collected at the same time along with the `real' set.

To have equal representation of sources of different magnitudes, the sources are first binned into 10 bins of \textcolor{red}{magnitude/flux?} and taken from different regions of the CCD.
We partition the image in a 4 by 4 grid and we select one star from each region and we do so for each magnitude bin. \textcolor{red}{explain better!!!}

\subsection{The CSTAR image-set preparation}

The CSTAR image set is a high cadence set of images taken in the winter of 2010 in Antarctica, the South Pole. It comprises 6 months of data with an average cadence of \textcolor{red}{1 min ???? check}.

Since the amount of data is so large, for the purpose of training, we created a subset of 626 images, with a cadence of about an hour during the best seeing month of June (May 31st to June 30th, 2010). The subset was named `cstar\_june\_selection'. The first 10 images of this set (named cstar\_june\_01) is used for training purposes and the rest can be used to search for transients. 

The images are mostly clean, that is they were bias and flat corrected.
Nevertheless, the images suffer from `bleeding' even though they have low exposure time. This bleeding is very significative for bright stars and less so for dimmer stars but still present nonetheless.

The bled stars were covered and a separate mask was created for the covered pixels for future reference. This pixel mask also includes defective pixels (dead lines and columns in the CCD).

The date-time information on the header was also updated according to \textcolor{red}{[ref]}.

For each of these images, we aligned a reference image with it using the package `astroalign' (see section astroalign) and also performed a image subtraction using the delta basis method on a 4 by 4 grid.

\section{The Classifier}

As previously said, we trained a Random Forest classifier based on 7,624 examples of fabricated transients (as described in section data) and 7,624 bogus picked at random from the subtraction images. This totals 15,248 samples for training and validation.

Several scores about the performance of the classifier on the training data are condensed in the scores table (\ref{mlscores}).

\begin{table}
\centering
\begin{tabular}{| >{\itshape}l | l |}
  \hline
  accuracy & 0.9935 \\ \hline
  precision & 0.9956 \\ \hline
  recall & 0.9915 \\ \hline
  F measure & 0.9936 \\ \hline
\end{tabular}
\caption{Scores}
\label{mlscores}
\end{table}

On the training data, the performance is quite good, with all indicators scoring above 99 percent.

In the confusion matrix (table \ref{mlconfusionmatrix}) we see that only 65 out of 7,624 of the reals were mis-classified as bogus, and only 33 bogus out of the 7,624 were mis-classified as reals. These very low numbers explain the very high performance scores of the classifier.

\begin{table}
\centering
\begin{tabular}{ l|c|c| }
\multicolumn{1}{r}{}
 &  \multicolumn{1}{c}{real}
 & \multicolumn{1}{c}{bogus} \\
\cline{2-3}
{\it Classified as} real & 7591 & 33 \\
\cline{2-3}
{\it Classified as} bogus & 65 & 7559 \\
\cline{2-3}
\end{tabular}
\caption{Confusion Matrix}
\label{mlconfusionmatrix}
\end{table}

The ROC curve is presented in figure ??.

\section{Testing the classifier on real data}

Once we have trained our classifier with training data, we wanted to also test it in a more real situation. For that we apply the classifier to the rest of the cstar\_june\_selection data set in search of actual transients of the image.

\begin{comment}
	* Our test-drive with OIS + ML. Results of paper (hopefully)
		* Test data we used (CSTAR)
		* Processing our raw data (selecting images for a dataset, cleaning images, fixing headers, performing subtractions, identifying sources in subtractions, stamps)
		* Getting samples for training (labeling data as RB, winnow, first run of ML)
		* Feature exploration of data (SExtractor features, derived features, morphological features like Zernike, Chebyshev, Fourier, etc.)
		* Random Forest + SMOTE + Cost Matrix (how well did this perform?)
		* Is it reproducible in other data sets?
\end{comment}



		
\chapter{Software Developed}
\section{Pipeline}

Since the ultimate goal of the TOROS Project is the robotization of the telescope,
it is important to have in place a software `pipeline' that accompanies the process
at all stages, from receiving the alert to propose new transient candidates.

The whole operation of TOROS processing can be divided in several stages (Tania ref)
as pictured in figure ??.

The first stage consists on ...

\subsection{Alert Receiver Robot}

The first step of the process starts with the alert receiver robot.

The alert receiver is hosted in a virtual machine server provided by UTRGV IT Services.
It is a Python script that runs uninterruptedly waiting for a signal from the 
GCN service of NASA delivered on a specific port reserved for this purpose.

The Gamma-Ray burst Coordinates Network (GCN) and the Transient Astronomy Network (TAN), collectively called the GCN/TAN network,
is a distribution network of transient events notices dependent on Goddard NASA.
As its name suggest it was primarily intended for the distribution of GRBs but also other transient notices
from Fermi and Swift spacecrafts to the rest of the Astronomy community. 
Most notices are sent in real-time, while the event is still ongoing, 
others are delayed due to telemetry down-link delays.

The subsequent reports of follow-up observations made by ground-based observers, are done by submitting circulars to that same site.
This makes the GCN/TAN network ``a one-stop shopping network for follow-up sites and GRB and transient researchers.''

The LIGO-Virgo Collaboration makes use of the GCN network to disseminate the event
notices to the participating observatories searching for EM counterparts.

The notice comes in the form of a Virtual Observatory Event (VOE) file,
and it is distributed as a message to predefined static IPs and ports,
using the VOEvent Transport Protocol.

\textcolor{blue}{VOEvent is an IVOA Recommendation -- that is, it has been adopted as an international standard. As with many other IVOA standards, VOEvent is based on XML, the Extensible Markup Language [9]. XML is ubiquitous in worldwide web technologies and simply provides a structured way to build a nested hierarchy of elements, to attach attributes to those elements, and to assign values to each. An additional constraint on many IVOA standards, including VOEvent, is a schema to apply rules on the arrangement and numbers of each element and attribute and on their allowed values. XML Schema [10] can be an often-entertaining technology for the designers of a standard. Providing a battle-hardened VOEvent schema [11] is a priority of the VOEvent v2.0 development effort. A well-crafted schema permits the validation of documents (in this case VOEvent messages, also referred to as ``packets'') against the requirements of the standard, and can even be used to automatically create software to parse such messages.}

The VOEvent Transport Protocol is a simple TCP-based protocol for transporting VOEvent messages from authors, through brokers, to subscribers.

VOEvent is the International Virtual Observatory Alliance (IVOA) recommended mechanism for describing astronomical transients.
According to its main website \footnote{\href{http://wiki.ivoa.net/twiki/bin/view/IVOA/IvoaVOEvent}{http://wiki.ivoa.net/twiki/bin/view/IVOA/IvoaVOEvent}},
it is an XML file notice that ``{\em defines the content and meaning of a standard information packet for representing, transmitting, publishing and archiving information about a transient celestial event, with the implication that timely follow-up is of interest.}'' 

In our case, we registered two static IPs with GCN, one in Texas at UT RGV and another one for Cordoba in Argentina at IATE Institute.
Only the one in Texas is functional at the moment, while the one in Cordoba is receiving but has problems sending out emails.

The VOEvent is received by a continuously running script that listens on the specific port and creates an alert email involving the heads and some staff at each TOROS partner.

The listening is done mainly by the PyGCN module developed by Leo Singer, which handles the reception GCN notices from LVC. The rest of email sending is done by usual Python tools for that task.

Right now, the script does a minimum process to deliver the XML body of the VOEvent by email to predetermined recipients.
Future improvements on this side should include the automatic download of the skymaps
to be attached to the email along with the VO Event notice.

It could also pre-process the skymap to extract most likely galaxy hosts for observing targets.

Right now, this is done on a separate script that requires human intervention.

\subsection{Target Selection}

As mentioned before, the target selection is done with a separate script.
This script makes use of the sky-map provided in the alert notice and the 
Gravitational Waves Galaxy Catalog (GWGGC) (A List of Galaxies for Gravitational Wave Searches, Darren J. White et al, 2011)

The White's catalog is a homogeneous list of 53,255 galaxies within 100Mpc.
It is a compilation from 4 different catalogs: an updated version of the Tully Nearby Galaxy Catalog,
the Catalog of Neighboring Galaxies, the V8k catalogue and HyperLEDA.

GWGC contains information on sky position, distance, blue magnitude, major and minor diameters, position angle, and galaxy type.
Also included in the catalog are 150 Milky Way globular clusters.

The authors claim (ref) that GWGC is more complete --within 100 Mpc-- than other catalogs,
due to their use of more up-to-date input catalogs
and the fact that they don't make a blue luminosity cut.

Another catalog along these lines is the Glade GW Catalog. Glade Catalog has xxxx galaxies listed, this many more than CWGC and extends up to XX Mpc.
Nonetheless many distances are inferred by Machine Learning Methods, and completitude of the Catalog... bla bla

The target selection is based on two main criteria: the localization (un)certainty on the target pixel in the all-sky map given by LIGO at the time of the alert,
and several filter cuts on distance, blue luminosity and apparent magnitude.

This marriage between GW and Optical parameters of observability ensures the targets are visible at each telescope site, and that they have some significant probability of being the host.
The list of filter cuts are summarized in table \ref{obsfilters}.

\begin{table}
\centering
\begin{tabular}{|l|c|}
  \hline
 Parameter & Limit value \\ \hline
Observability from location & $30^{\circ} > \delta > -70^{\circ}$ for EABA \\ \hline
Apparent Magnitude & $B \le 21$ mag \\ \hline
Distance  & $D < 60$ Mpc \\ \hline
Absolute Magnitude & $MB \le -21$ mag \\ \hline
\end{tabular}
\caption{Parameter Cuts}
\label{obsfilters}
\end{table}

\subsection{The Broker Website}

Once the list of targets is done, we need to communicate the targets to each telescope.
This is done through a broker website written in Django.
Each telescope representative has assigned a username and password to access a website hosted on UTRGV servers.
This website has a simple interface of tables for each observatory and a list of targets observable from each location, obtained by the target selection method on the previous step.

Each telescope admin then selects targets from the list, to reserve them for his or her observatory. 
The website enforces that the list of targets be unique and that no two observatories are observing the same target.
It does so, by ignoring targets previously selected by others on the query.

Each observatory then carries the observations for the successive nights and file a report of the observations when these are done.

The website can also output a circular draft of all the targets observed by each telescope to be submitted to the GCN/TAN website for the rest of the community.

\subsection{Further Processing}

The previous steps are the normal workflow for the operations on a LIGO campaign.

The pipeline is not fully automatized or optimized for all of the stages, and there are still gaps where human intervention is needed.
Most notably is the target selction or scheduling, and the loading of the observation targets to the broker website.
Making them automatic, would require a REST framework on the Django website and a transfer protocol mechanism,
which are beyond the computational abilities of who writes this thesis.
It is my hope that the software developed so far could be picked up by a web developer that can connect the parts in a seamless way.

The pipeline is written entirely in Python, following the latest trend in Astronomy.
For the web pages developed to support the collaboration, we used the popular Python web framework Django
and a light database on Sqlite 3.

Sqlite provides a lightweight database solution that does not require complex server-client operations, and it stores all data in a simple file.
Thus, the database is easy to browse and modify.
The database handles information for the GW Galaxy Catalog, the user information as well as permissions and login information,
Observatories data.

The following stages of the pipeline deal with the specific analysis to identify candidates to optical counterparts of GW.

The first part deals with the Difference Image Analysis.
For that two images, one taken at the epoch of interest and another one used as a reference. Ideally, the reference is an archive image or a stack of several others to improve signal to noise ratio, cosmic rays and other contaminants on the image.

In the case for our campaign for O1, we had to use a-posteriori references taken a few months after the event.

Before performing the subtraction, both images have to be aligned pixel by pixel.
Software to perform the alignments was done using a method described in section \ref{astroalign_section}.
I describe there a method based on asterism matching inspired in astrometry.net.

The subtraction methods are explained in section \ref{ois_section}.
Several algorithms are explained in the literature with a wide range of methods, from Fourier Transform, to PSF matching to using Information Theory.
Some of them are explained and implemented in a Python module named `ois' developed by me for this project.

\section{Winnow - Human classification of Real/Bogus}

Blue is from Hotwiring the Transient Universe:

\textcolor{blue}{While machines are capable of many types of information processing, they are not so good where something new is present that has not been programmed. Certainly we expect the future flood of events to be mostly handled by machines, with the uninteresting ones never seen by a human expert, but some may be escalated in importance and come to the attention of such experts through a message, or even being urgently awoken in the night.}

\textcolor{blue}{There could be a large number of other people also involved in the enterprise, volunteers recruited from the internet, with some, but by no means expert, ability. All people have excellent image analysis capabilities: they could, for example, look at an image of a star field and determine quickly and accurately if there is an artifact, such as a satellite trail, interference from a nearby bright star, or one of many Earthbound artifacts: from the telescope, camera, or electronics. While many of these types of common artifacts can be detected by machine, there are always new types, or artifacts that are a combination of known types. Since the transient detection software is looking for differences between new and past observations, such artifacts, though rare, will be inevitably found and thus pollute the event stream.}

\textcolor{blue}{This type of `citizen science' has been both popular and extremely useful in GalaxyZoo[6] and CitizenSky[7], and we expect it to be the same with events. A new aspect with events, different from the traditional web-based citizen science, could be that events are `pushed' to the volunteers, so they can respond with their mobile device immediately. Another novel aspect to citizen science could be the recruiting of a cadre of dedicated volunteers to work at a more expert level, looking at light curves or other non-image data; they would need to be sufficiently motivated to study and take a test, to be inducted to this higher level.}

\textcolor{blue}{Given that there will be tens of thousands of transient candidates competing for scarce resources, it is impossible to eyeball even those that survive the myriad of classification steps mentioned earlier. Human neural networks can come to the rescue in such a case. Details about citizen science are provided elsewhere in this book. Here we would like to highlight one particular aspect, the synergy between machine learning and human pattern recognition expertise to improve the machine learning methodologies.}

It is worth mentioning that the web server also hosts a related project `winnow', a web interface to manually classify potential transients
on subtraction images, by voting on `real' or `bogus' categories. 
This voting allows to have good training sets to later train automatic classifiers to do the same task.
The website was tested successfully with many undergraduate and graduate students, 
and there are plans to apply it to high-school students as a way of citizen science project.

\section{Astroalign}

\subsection{The Algorithm in a Nutshell}

The core idea of the algorithm consists on characterizing asterisms (for example triangles or quadrilaterals) by using quantities invariant to translation, rotation or even scaling and flipping. Similar asterisms will have similar invariant tuples in both images so a correspondence can be made between those invariant quantities. 

As an example, the lengths of the sides of a triangle are invariant to translation and rotation. They remain the same whatever position or orientation the triangle may have. Any function of the {\em ratio} of the sides will, in addition, be invariant to scaling.

The idea for the algorithm can be summarized in a few steps

\begin{enumerate}
\item Do for both images \begin{enumerate}
\item Make a catalog of a few brightest sources (but not too few!)
\item Create a 2D tree of the sources to quickly query for close neighbors. (A kd-tree data structure for k=2)
\item For each star, select the 4 nearest neighbors (5 sources including the star itself).
\item Form all the ${5}\choose{3}$ posible triangles from that set of stars.
\item For each triangle in that set, calculate the tuple of invariants that fully characterize the triangle and push the invariant tuple into another kd-tree.
\item There could be many duplicate triangles on the previous list, so it's best to remove them leaving only unique elements.
\end{enumerate}
\item Now do a matching between the two invariant kd-trees to find matches for similar triangles. Two similar triangles will have similar invariant features.
\item Even within a triangle match, one can make a correspondence between individual points by looking at which sides the point belongs to. This way, one can make a point to point correspondence for each triangle correspondence.
\item Pass the invariant matches set to a RANSAC algorithm that will decide which triangles suggest a transformation that fits many other triangles.
\end{enumerate}


\subsection{Selecting Asterisms and Invariant Features}

To find a correspondence we need to fix which figures will we search for in both images. 
The simplest figure is the triangle. A triangle (with all different sides) can determine a unique transformation between two images. 
Another possibility is to search for polygons with more sides. 
The package Astrometry.net uses quadrilaterals for this purpose, but even pentagons or other polygons can be used. 
We will focus on the triangle matching on this note.

For a triangle, knowledge of all its 3 side lengths is enough to fully characterize it, irrespective of position or orientation. 
If we want to characterize it up to a global scaling, then knowing 2 inner angles is enough. 
Equivalently, knowing 2 independent ratios of the side lengths is also enough. 
In fact any function of 2 independent length ratios is enough.

So, for example the tuple $(\frac{L_2}{L_1}, \frac{L_1}{L_0})$ (where $L_2 > L_1 > L_0$) is a valid invariant tuple that fully describe the triangle up to translation, rotation and scaling, and even coordinate flipping.

\subsubsection{Analysis of the invariants}

Let's analyze here the example invariant set given in the previous section.

\begin{align*} 
I_{1}(L_i,L_j,L_k) =& \left( \frac{L_2}{L_1}, \frac{L_1}{L_0} \right)  \numberthis \label{inv01} \\ 
 & \text{where} \left\{
  \begin{array}{lll} 
 L_2 &=& \max\{L\} \\
 L_1&=&\text{middle}\{L\} \\ 
 L_0 &=& \min\{L\}
  \end{array}
\right.
\end{align*}

This choice of invariants maps the positive octant of $\mathbb{R}^3$ of all possible side lengths of a triangle, 
onto a region of the positive quadrant of $\mathbb{R}^2$ in the invariant-features space 

To find out what this region is, we notice that since $L_2 > L_1 > L_0$,

\begin{align*}
x &=  \frac{L_2}{L_1} > 1 \\
y &=  \frac{L_1}{L_0} > 1
\end{align*}

Also, using the triangle inequality:

\begin{align*}
L_2 \leq L_1 + L_0 \implies & x \leq 1 + \frac{1}{y} \\
& y \leq \frac{1}{x-1}
\end{align*}

The curve $y = (x-1)^{-1}$ corresponds to colinear points.

Also, it's worth noting that any equilateral triangle will map to the point $(1,1)$ and an isosceles triangle will map either to $x=1$ or $y=1$ line depending on whether the unequal side is the largest or smallest.

Very peaky triangles will tend to accumulate between the colinear points curve and the $x=1$ line for large values of $y$.

All these observations can be summarized in figure \ref{fig:inv_region}.

\begin{figure}[htbp]
   \centering
   \includegraphics[width = \linewidth]{chapter_astroalign/figures/invariantMap01.pdf}
   \caption{Region for a particular invariant mapping}
   \label{fig:inv_region}
\end{figure}


%Other invariant sets can be constructed, each with a particular mapping from the set of side lengths of triangles to a region in the 2D plane of invariants.

%Some other examples are given in figure \ref{fig:inv_maps}. These have been explored numerically plotting the invariants for a large number of triangles.

%\begin{figure}[htb]
%   \centering
%   \includegraphics[width = \linewidth]{chapter_astroalign/figures/differentInvariantMaps.pdf}
%   \caption{Four examples of invariant mappings}
%   \label{fig:inv_maps}
%\end{figure}

%\lipsum

\subsection{An Ideal Example}

Let's see how the algorithm performs on an ideal example.

For this, we create several stars at random positions and we rotate and translate them as seen in figure \ref{fig:ideal_sources}.

\begin{figure}[htbp]
   \centering
   \includegraphics[width = \linewidth]{chapter_astroalign/figures/idealSources.pdf}
   \caption{Two ideal distribution of sources}
   \label{fig:ideal_sources}
\end{figure}

The invariant features in (\ref{inv01}) from this set of stars is plotted in figure \ref{fig:ideal_inv}.

\begin{figure}
   \centering
   \includegraphics[width = \linewidth]{chapter_astroalign/figures/idealInvariants.pdf}
   \caption{Invariants for the ideal example}
   \label{fig:ideal_inv}
\end{figure}

We note that some invariants belong to collinear points, and the distribution of points is fairly sparse. 
This will help the identification phase when we try to match our triangles.

In the figure, the invariant points for both images were plotted. They appear superimposed in the plot.

We note that every blue invariant point has its corresponding red invariant point on top. 
There are no points without a partner for neither of the images. 
This is because all the sources in one image appear in the other one, there are no missing stars.
In a real situation, some stars will be missing because they are out of the field of view or because they became too faint due to extinction or any other technical reason. 
The algorithm should still work with missing or extra stars in the reference or test image. 

Another issue with real images is that locating the position of a source is not entirely precise, so small errors will appear in the expected position of one source with respect to its partner in the other image.
This error in turn creates an error on the lengths of the sides of the triangle and thus on the invariants calculated from it.
In practice, the invariant points will lay close to each other to a given small tolerance.

Once we have the set of invariants from both images, we query a correspondence to the kd-tree for possible matches within a given tolerance radius. 
Each returned match will be a correspondence between a triangle in one image and another. 
From each, we can make the point to point correspondence, provided all sides are unequal, and this will determine a unique transformation between the images.

Some of the correspondences won't be real ones. It could be that by chance there are two similar triangles in the images that belong to different set of stars.

This is where the RANSAC algorithm comes in.

\subsection{The RANSAC algorithm}

From its Wikipedia page 

{\em ``The Random sample consensus (RANSAC) algorithm is an iterative method to estimate parameters of a mathematical model from a set of observed data which contains outliers.''}

In our case the mathematical model is the similarity transformation between both images and the parameters are those of the transformation, i.e. the rotation, translation and uniform scaling parameters.

RANSAC is capable of choosing a transformation that fits most of the other triangles and is not affected by the rest of spurious outliers. 

This algorithm is also used in the computer vision package OpenCV for a very similar purpose, to ignore outliers when trying to estimate an homography between two images. 

In our case we look for the parameters $t_x$, $t_y$ for the translation in the $x$ and $y$ direction, the rotation angle $\alpha$ and the dilation parameter $\lambda$.

The transformation applied to a point $(x,y)$ will look like this:

\begin{align*}
\left(
 \begin{array}{lll} 
 \lambda \cos \alpha & \lambda \sin \alpha& \lambda t_x \\
 - \lambda \sin \alpha & \lambda \cos \alpha& \lambda t_y \\
 0 & 0 & 1
 \end{array}
\right)
\equiv
\left(
 \begin{array}{lll} 
 a_0 & b_0 & c_0 \\
-b_0 & a_0 & c_1 \\
 0 & 0 & 1
 \end{array}
\right)
\end{align*}

To make our problem linear, we will consider the parameters $a_0, b_0, c_0, c_1$ as if they were independent.

Two data points pairs are sufficient to determine uniquely a transformation for this 4 parameters. 
More can be used if we use a linear square minimization.

%This candidate transformation $T$ will be tested against all the other triangle matches in a RANSAC algorithm.

\subsection{Error propagation}

Doing a simple propagation of errors we see that for nearly equilateral triangles $L_{1} \approx L_{2} \approx L_{3} \approx L$, 
the errors in the invariants go like $\Delta I_{i} \sim \frac{\Delta L}{L}$.
This means that, for invariants near $(I_{1}, I_{2}) = (1, 1)$ errors in the determination of the lengths of the triangle side will not magnify errors in the invariants.

%This is done by \citet{1995PASP..107.1119V}.
 \label{astroalign_section}

\section{Optimal Image Subtraction} \label{ois_section}
	
OIS (Optimal Image Subtraction) is a Python module that implements several Difference Image Analysis methods as described in section X.

It works on Numpy arrays, so that way is agnostic on the origin of the image. 

Bad pixels that need to be ignored in the image are marked using Numpy's masked arrays (True on bad pixel).

The interface to the user has two entry points: the module methods optimal\_system and subtract\_on\_grid. The latter is just a convenient method to partition the image in a certain grid and perform the former on each grid, taking into account pixels outside the grid when necessary.

optimal\_system will return ...

OIS has documentation on the popular documentation site readthedocs.io.
OIS is released under MIT Licence and has a GitHub page on ...


    
\chapter{Conclusions}
* The future of LIGO, TOROS and multi messenger astronomy


\appendix
\chapter{Derivation of the Gravitational Wave Equations} \label{gwderivation}

Gravitational waves are a particular kind of solution to the Einstein's field equations of General Relativity.

In many situations we can consider, we are in a flat background situation, in which our metric does not differ much locally from the Minkowski metric.
Suppose we are far away, removed from any strong source in an asymptotically flat spacetime.

In such situation we can assume that locally our metric is the Minkowski metric $\eta$ plus some small deviation $h$. We want to study the dynamics of such small perturbation in a linearized Einstein field equation. 


To do that, we consider the metric $g = \eta + h$ in the linearized Einstein's field equations. Linearized here means that quadratic and higher factors of $h$ will be simply ignored, as they are assumed much smaller than the flat metric.

Let's find out what condition the Einstein field equation $G_{\mu \nu}(\eta + h) = 8\pi T_{\mu \nu} = 0$ imposes on this small perturbation $h$.

%\begin{align}
%G_{\mu \nu}(\eta + h) = 8\pi T_{\mu \nu} = 0
%\end{align}

%It is worth noting here that Einstein equations are not linear in nature, but they are second order on the metric derivatives and furthermore linear on the second order terms. This will allow us to derive from the linearized Einstein equations, a second order wave equation on the small perturbation $h$ that decouples from $\eta$.

%Another issue to point out is that the vague definition of 'small perturbation' has to be of physical nature and not an artifact on the choice of the coordinate system. We will not worry about this issue on the present work, but it is worth keeping it in mind. Other concerns about coordinate system effects were historically raised and settled in time, particularly the many gauge choices done during the derivation.

The Christoffel symbols to first order in $h$ are:

\begin{align}
\Gamma_{abc} &= \frac{1}{2} \left( g_{ca,b} + g_{cb,a} - g_{ab,c} \right) \\
  &= \frac{1}{2} \left( (\eta + h)_{ca,b} + (\eta + h)_{cb,a} - (\eta + h)_{ab,c} \right) \\
 &=\frac{1}{2} \left( h_{ca,b} + h_{cb,a} - h_{ab,c} \right) 
\end{align}

At this point we must note that we will still call $g^{ab}$ the inverse of $g_{ab}$, which is $g^{ab} = \eta^{ab} - h^{ab}$ to first order in $h$. 

And thus,
\begin{align}
\Gamma^{a}_{bc} &= g^{cd} \Gamma_{abd} \\
 &=(\eta - h)^{cd} \Gamma_{abd} \\
 &= \frac{1}{2} \eta^{cd} \left( h_{ca,b} + h_{cb,a} - h_{ab,c} \right) + \order(h^2)
\end{align}

Similarly, the Ricci tensor to first order in $h$ will be:

\begin{align}
R_{ab} &= \Gamma^{c}_{ab,c} - \Gamma^{c}_{cb,a} \\
&= \frac{1}{2} \left( h_{a}{}^{c}{}_{,bc} + h_{b}{}^{c}{}_{,ac} - h_{ab,c}{}^{c} - h_{c}{}^{c}{}_{,ab} \right)
\end{align}

Which will finally lead us to the Einstein's tensor $G$:

\begin{align}
G_{ab} &= R_{ab} - \frac{1}{2}g_{ab}R \\
&= R_{ab} - \frac{1}{2}\eta_{ab}R + \order(h^2) \\
&= \frac{1}{2} \left( h_{ac,b}{}^{c} + h_{bc,a}{}^{c} - h_{ab,c}{}^{c} - h_{c}{}^{c}{}_{,ab} -\eta_{ab} \left( h_{cd,}{}^{cd} - h_{c}{}^{c}{}_{,d}{}^{d} \right) \right)
\end{align}

This can be brought to a shorter form defining $\bar{h}{}_{ab} = h_{ab} - \eta_{ab} h$

Where $h = h_{c}{}^{c}$ is the trace of the metric tensor $h$.
With this definition, the Einstein's field equation becomes:

\begin{align}
\bar{h}{}_{ab,c}{}^{c} + \bar{h}{}_{ac,b}{}^{c} + \bar{h}{}_{bc,a}{}^{c} -\eta_{ab} \bar{h}{}_{cd,}{}^{cd} = 0
\end{align}

\section{Gauge choices}

There is a gauge freedom in General Relativity corresponding to the group of diffeomorphisms, that can be used to simplify the equations even more. Just like the Gauge freedom in electrodynamics $A \rightarrow A + \partial \chi$ we can chose $h$ to satisfy $\bar{h}{}_{ab,}{}^{b} = 0$

In this ``Lorentz Gauge'', the linearized Einstein field equations, reduce to the usual wave equation for each component of $\bar{h}$:

\begin{align}
\bar{h}_{ab,c}{}^{c} = \Box{\bar{h}{}_{ab}} = 0
\end{align}

Further gauge choices can let us choose the trace of $\bar{h}$ to be zero and each $h_{0\mu}$ component to be zero for $\mu = 0,1,2,3$. Notice that once the trace of $\bar{h}$ is set to zero, it implies that $\bar{h}{}_{ab} = h_{ab}$

\begin{align} \label{gaugecond}
& \Box{h_{ab}} = 0 \\
& h_{ab,}{}^{b} = 0 \\
& h_{a}{}^{a} = 0 \\
& h_{0 \mu} = 0; \; \mu=0,1,2,3
\end{align}

This is called the transverse traceless (TT) gauge. The reader can refer the full derivation for the gauge choices in Misner or Wald.

With the conditions in (\ref{gaugecond}), we seek solutions in the form of plane waves of the form $h_{ab} = H_{ab}e^{\pm \imath k_{\mu}x^{\mu}}$.

Our gauge conditions impose similar conditions on $k$ and $H$:

\begin{align}
& k^{\mu} H_{\mu \nu} = 0 \\
& H_{\mu}{}^{\mu} = 0 \\
& H_{0 \mu} = 0; \; \mu=0,1,2,3
\end{align}

This leaves $H$ with only two independent components. If we consider a wave propagating in the z direction, the most general form for a transverse traceless metric will be:

\begin{equation}
\begin{bmatrix}
0 & 0 & 0 & 0 \\
0 & h_{+} & h_{\times} & 0 \\
0 & h_{\times} & -h_{+} & 0 \\
0 & 0 & 0 & 0 \\
\end{bmatrix}
e^{\pm \imath k_{\mu}x^{\mu}}
\end{equation}

and the total metric $g$

\begin{equation}
\begin{bmatrix}
-1 & 0 & 0 & 0 \\
0 & 1 + h_{+}e^{\pm \imath k_{\mu}x^{\mu}} & h_{\times} e^{\pm \imath k_{\mu}x^{\mu}} & 0 \\
0 & h_{\times} e^{\pm \imath k_{\mu}x^{\mu}} & 1 - h_{+}e^{\pm \imath k_{\mu}x^{\mu}} & 0 \\
0 & 0 & 0 & 1 \\
\end{bmatrix}
\end{equation}

The two independent polarizations are called ``h plus'' ($h_+$) and ``h cross'' ($h_{\times}$).

\chapter{Notations}

Here we show the use of multiple appendixes.

\section{Math Notations}

Each appendix can have sub-sections as a regular chapter.

\section{Additional Notations}

\pagebreak{}

\bibliographystyle{plain}
\nocite{*}
\bibliography{sampleThesis}

\begin{vita}
\section*{Education}

\begin{itemize}
  \item B.S. Physics, National Cordoba University (Argentina), 2007.
  \item M.S. Physics, University of Texas at Brownsville, 2010.
\end{itemize}

\section*{Publications}

\begin{itemize}
\item \href{http://adsabs.harvard.edu/abs/2016ApJ...828L..16D}{\bf GW150914: First search for the electromagnetic counterpart of a gravitational-wave event by the TOROS collaboration} (2016). Mario C. Diaz et al; {\it The Astrophysical Journal Letters}, Volume 828, Issue 2, article id. L16, 6 pp.
\item \href{http://adsabs.harvard.edu/abs/2016ApJ...826L..13A}{\bf Localization and broadband follow-up of the gravitational-wave transient GW150914} (2016). B. P. Abbott et al. {\it The Astrophysical Journal Letters}, Volume 826, Issue 1, article id. L13, pp
\item \href{http://www.math.unm.edu/~lau/DMS1216866/tauFluids_BHLP.pdf}{\bf Multidomain, sparse, spectral-tau method for helically symmetric flow} (2014). M Beroiz, T Hagstrom, SR Lau, RH Price; {\it Computers and Fluids}  (2014), pp. 250-265.
\item \href{http://iopscience.iop.org/1538-3881/147/5/100}{\bf Bright microwave pulses from PSR B0531+21 observed with a prototype transient survey receiver} April 2014. J. Andrew O'Dea, F. A. Jenet, Tsan-Huei Cheng, Chau M. Buu, Martin Beroiz, Sami W. Asmar, and J. W. Armstrong; {\it Astronomical Journal}, 147, 100.
\item \href{http://ieeexplore.ieee.org/xpls/abs_all.jsp?arnumber=5555950}{\bf A Prototype Radio Transient Survey Instrument for Piggyback Deep Space Network Tracking}, May 2011. Chau M. Buu, Fredrick A. Jenet, John W. Armstrong, Sami W. Asmar, Martin Beroiz, Tsan-Huei Cheng, and J. Andrew O'Dea; {\it Proceedings of the IEEE} Volume 99, Number 5.
\item \href{http://journals.aps.org/prd/abstract/10.1103/PhysRevD.76.024012}{\bf Gravitational instability of static spherically symmetric Einstein-Gauss-Bonnet black holes in five and six dimensions}, 2007. M. Beroiz, G. Dotti y R.J. Gleiser, {\it Physical Review D} 76, 024012 hep-th/0703074.
\end{itemize}


\section*{Scientific Meetings and Conferences}

\begin{itemize}
\item \href{http://adsabs.harvard.edu/abs/2016APS..APRR14005B}{Speaker at the APS April Meeting 2016}, abstract R14.005. "Results of optical follow-up observations of advanced LIGO triggers from O1 in the southern hemisphere". South Lake City, April 2016.
\item \href{https://conference.scipy.org/scipy2014/schedule/presentation/1730/}{SciPy Conference 2014. Speaker at Mini Symposium in Astronomy}. "Transient detection and image analysis pipeline for TOROS project". Austin July 2014.
\item Advances and Challenges in Computational General Relativity, Brown University, Providence, RI, May 20-22, 2011.
\item Poster presentation 215th American Astronomical Society (AAS) Meeting, Washington DC, January 3-7, 2010 ``The X-ray Emission of SN1978K: Still Here After All These Years''  - Eric M. Schlegel, M. Beroiz.
\item Poster presentation 215th American Astronomical Society (AAS) Meeting, Washington DC, January 3-7, 2010 ``Current Results at PALFA Pulsar Survey at Arecibo Observatory'' - M. Beroiz, K. Stovall, F. Jenet, J. Cordes, D. Lorimer, D. Backer, PALFALFA Consortium.
\item Summer Provost Research Program at UTSA, 2009. Research on X-ray spectrum of Super Nova 1978K and Poster Presentation.
\item UTB Summer School in Gravitational Wave Astronomy at South Padre Island TX, June 2008.
\item Grav07, La Falda (Cordoba), Argentina. November 5-7, 2007.
\item Poster Presentation at 92nd AFA (Physics Association Argentina) Annual Meeting, Salta, Argentina, September 24-27, 2007.
\end{itemize}

\end{vita}

\end{document}
