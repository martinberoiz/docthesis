%% LyX 2.1.4 created this file.  For more info, see http://www.lyx.org/.
%% Do not edit unless you really know what you are doing.
\documentclass[12pt,english]{report}
\usepackage[T1]{fontenc}
\usepackage[latin9]{inputenc}
\usepackage{babel}
\usepackage{longtable}
\usepackage{float}
\usepackage{calc}
\usepackage{amsmath}
\usepackage{amsthm}
\usepackage{setspace}
\usepackage[unicode=true,pdfusetitle,
 bookmarks=true,bookmarksnumbered=false,bookmarksopen=false,
 breaklinks=false,pdfborder={0 0 1},backref=false,colorlinks=false]
 {hyperref}
\usepackage{comment}

\makeatletter

%%%%%%%%%%%%%%%%%%%%%%%%%%%%%% LyX specific LaTeX commands.
\providecommand{\LyX}{\texorpdfstring%
  {L\kern-.1667em\lower.25em\hbox{Y}\kern-.125emX\@}
  {LyX}}
%% Because html converters don't know tabularnewline
\providecommand{\tabularnewline}{\\}
\floatstyle{ruled}
\newfloat{algorithm}{tbp}{loa}[chapter]
\providecommand{\algorithmname}{Algorithm}
\floatname{algorithm}{\protect\algorithmname}

%%%%%%%%%%%%%%%%%%%%%%%%%%%%%% Textclass specific LaTeX commands.
\usepackage{UTSAthesis}      
\usepackage{times}            
\usepackage{latexsym}

%Bibliography packages
\usepackage[square]{natbib} % defines citet, citep, ...
\bibpunct{(}{)}{;}{a}{}{,} % to follow the A&A style - 
\newcommand{\aj}{AJ}
\newcommand{\apj}{ApJ}
\newcommand{\apjl}{ApJ}
\newcommand{\apjs}{ApJS}
\newcommand{\aap}{A\&A}
\newcommand{\aaps}{A\&AS}
\newcommand{\mnras}{MNRAS}
\newcommand{\nat}{Nature}
\newcommand{\araa}{ARAA}
\newcommand{\prd}{Phys. Rev. D}
\newcommand{\pasj}{PASJ}
\newcommand{\ETC}{et al.}
\newcommand{\physrep}{Physics Report}
\newcommand{\gca}{GCA}
\newcommand{\pasa}{PASA}
\newcommand{\pasp}{PASP}
\newcommand{\aapr}{A\&A~Rev.}
\newcommand{\apss}{Ap\&SS}
%End of bibliography packages 

%Added by me
\newcommand\numberthis{\addtocounter{equation}{1}\tag{\theequation}}
\newcommand{\order}{\ensuremath{\mathcal{O}}}
%End of added by me

\newenvironment{ruledcenter}{%
  \begin{center}
  \rule{\textwidth}{1mm} } {%
  \rule{\textwidth}{1mm} 
  \end{center}}%


  \theoremstyle{definition}
  \newtheorem{defn}{\protect\definitionname}
\theoremstyle{plain}
\newtheorem{thm}{\protect\theoremname}

\@ifundefined{showcaptionsetup}{}{%
 \PassOptionsToPackage{caption=false}{subfig}}
\usepackage{subfig}
\makeatother

\providecommand{\definitionname}{Definition}
\providecommand{\theoremname}{Theorem}

\begin{document}

% Committee Members
\supervisor{Mario Diaz, Ph.D.}
\committeeB{Lucas Macri, Ph.D.}
\committeeC{Matthew Benacquista, Ph.D.}
\committeeD{Eric Schlegel, Ph.D.}
\committeeE{Ricardo Lopez Mobilia, Ph.D.}

\informationitems{Doctor of Philosophy in Physics}{Ph.D.}{M.Sc.}{Department of Physics And Astronomy}{College of Sciences}{May}{ 2017 }

\thesiscopyright{Copyright 2017 Martin Beroiz \\
All rights reserved. }

\dedication{\emph{I would like to dedicate this thesis to ??????.}}


\title{\textbf{OPTICAL COUNTERPARTS TO GRAVITATIONAL WAVES}}


\author{Martin Beroiz}
\maketitle
\begin{acknowledgements}
Thanks y'all.

(Notice: If any part of the thesis/dissertation has been published
before, the following two paragraphs should be included without alteration).

\begin{singlespace}
\emph{This Masters Thesis/Recital Document or Doctoral Dissertation
was produced in accordance with guidelines which permit the inclusion
as part of the Masters Thesis/Recital Document or Doctoral Dissertation
the text of an original paper, or papers, submitted for publication.
The Masters Thesis/Recital Document or Doctoral Dissertation must
still conform to all other requirements explained in the Guide for
the Preparation of a Masters Thesis/Recital Document or Doctoral Dissertation
at The University of Texas at San Antonio. It must include a comprehensive
abstract, a full introduction and literature review, and a final overall
conclusion. Additional material (procedural and design data as well
as descriptions of equipment) must be provided in sufficient detail
to allow a clear and precise judgment to be made of the importance
and originality of the research reported. }

\emph{It is acceptable for this Masters Thesis/Recital Document or
Doctoral Dissertation to include as chapters authentic copies of papers
already published, provided these meet type size, margin, and legibility
requirements. In such cases, connecting texts, which provide logical
bridges between different manuscripts, are mandatory. Where the student
is not the sole author of a manuscript, the student is required to
make an explicit statement in the introductory material to that manuscript
describing the students contribution to the work and acknowledging
the contribution of the other author(s). The signatures of the Supervising
Committee which precede all other material in the Masters Thesis/Recital
Document or Doctoral Dissertation attest to the accuracy of this statement.}\end{singlespace}
\end{acknowledgements}
\begin{abstract}
The upcoming research in the Gravitational Wave domain has introduced new challenges to Astronomy and has opened a new window to the universe.

Our work here looks for a way to harness the power of multi-messenger astronomy to enhance and complement the incredible capabilities of the new generation of GW detectors, with LVC leading the way.

It will hopefully help newcomers to the field to understand the details and challenges of this new discipline and encourage research in the area. \textcolor{red}{[This is probably too pedantic]}

In the first chapter, I introduce the concepts and frame on which my thesis developed.

In the second chapter, I discuss the two main elements of a modern transient search. That is, Image Difference and Machine Learning classification.

\end{abstract}

\pageone{}

\chapter{Introduction}

In February 2016, celebrating the centenary anniversary of Einstein's first paper on gravitational waves(``Approximate integration of the field equations of gravitation''), the LVC collaboration announced the first ever direct detection of a gravitational wave, labeled GW150914.
With this, a new window of the universe for the purely relativistic astronomical phenomena was opened. 

The history of GW was not without controversies. From the discussion of whether GW would carry energy to the experiments of Joseph Weber, Gravitational Waves made its way from the theoretical realm into a stronghold position in Astrophysics.

The detection of GW150914 firmly established the base for a new kind of Astronomy.
Years to come will turn GW detection, into a full field GW Astronomy.

There is a new messenger in the universe that is purely relativistic in nature. The graviton and its wave counterpart will bring us information about high gravitational fields, compact massive objects, relativistic speed phenomena and more.
It will let us probe into the physics of the extreme gravitational field astronomical bodies.

The GW information will complete, refine and expand our understanding of our universe.
It will complement our telescopes in the radio, visible and every other frequency band. 
Optical Astronomy has become an even more important partner in this search.
Together they will uncover even more details of the inner mechanisms of the celestial bodies.

In the following sections, I offer a brief introduction of GWs and their relation to Optical Astronomy.

\section{Gravitational Waves}

\begin{comment}

%Quadrupole radiation is the lowest allowed form and is thus usually the dominant form. In this case, the GW field strength is proportional to the second time derivative of the quadrupole moment of the source, and it falls off in amplitude inversely with distance from the source.

%As with electromagnetic waves, GWs travel at the speed of light and are transverse in character, i.e. the strain oscillations occur in directions orthogonal to the direction in which the wave is propagating.
\end{comment}

General Relativity predicts that very massive or energetic events will create traveling perturbations of the underlying spacetime metric in the form of gravitational waves.
These are linear disturbances to a flat background metric, that are propagated outwards with the speed of light fading with the inverse of the distance.
These tensorial transversal waves will modify the local metric as it travels through space.

Putting it in more technical terms, in an asymptotically flat spacetime, removed from any strong source so that one can assume an almost flat local solution, one can study weak perturbations to the flat Minkowski metric.
In this regime, the linearized Einstein field equations admit wave solutions.
These wave solutions are called Gravitational Waves. \textcolor{red}{Add who coined the term for the first time, it's in the review.}

The GW metric has the form: (we refer the reader to Appendix \ref{gwderivation} for a more complete derivation of the metric)

\begin{equation}
h_{\mu \nu} = 
\begin{bmatrix}
0 & 0 & 0 & 0 \\
0 & h_{+}e^{\pm \imath k_{\mu}x^{\mu}} & h_{\times} e^{\pm \imath k_{\mu}x^{\mu}} & 0 \\
0 & h_{\times} e^{\pm \imath k_{\mu}x^{\mu}} & -h_{+}e^{\pm \imath k_{\mu}x^{\mu}} & 0 \\
0 & 0 & 0 & 0 \\
\end{bmatrix}
\end{equation}

\textcolor{red}{Talk about the two polarizations}

Since a spacetime metric is a measure of distance between pair of events, the metric perturbation wave will modify the relative distance between points in space as it passes by. 
It is in fact an effective oscillating strain or tidal force between free test masses.
One could then devise an instrument set to detect these relative differences in distance for such test masses.

But because of the very `perturbative' nature of GW, these differences are very small. 
To have any chance of detection, one must look into the extreme side of gravity: very massive compact astrophysical objects with big gravitational fields, moving at relativistic velocities.
Even the most powerful astronomical events, like a black hole merger will produce disturbances of $10^{-23}$ m at a few hundreds Mpc of distance.

Perhaps the first experimental evidence for the existence of GWs came 50 years after their prediction with the work of Husel and Taylor. For decades, they studied the pulses we receive from one Neutron Star (NS) in a Neutron Star Binary (NSB).
They discovered that the slow rate of period decrease of the pulses precisely matched General Relativity predictions. \textcolor{red}{To Add: It was a Nobel Prize work. It settled the debate on whether GW carry energy or not. The decrease in period rate is proof that the energy is radiated away into GW.}

Nonetheless, a direct detection of GW wouldn't come for another 50 years.
In September 14, 2015, a GW detection labeled GW150914 was finally made by the LIGO Collaboration.

Direct detection {\em in situ} of GW requires a transducer of GW in some other form measurable by common instruments.

One pioneer work along this line was the work done by Joseph Weber in the late 1960s and early 70s. He proposed using a 2 meters in length and 1 meter in diameter aluminium cylinder, that would resonate with passing GW at 1660 Hz.
These tiny resonant oscillations could in principle be measured. Even though ingenious, the carefully devised experiment could not yield positive reproducible results that convinced the scientific community. \textcolor{red}{Maybe add that there were still attempts at cryogenic resonators into the 90s?}

A whole other category of these instruments are the ones based in the interferometry of lasers running on two long arms. Several of this kind are built on different parts of the planet. A few are planned to be built in the future. Most notably is the case of Lisa, a proposed GW interferometer to be set in space trailing the Earth's orbit.

Interferometer GW detectors transduce a GW warp in space to the shrinking and stretching of relative distances of masses put far apart in two or more long interferometers. 
In the case of the Earth based observatories, each interferometer is one arm several kilometers long, of an L shape facility.
The masses are the end mirrors that reflect a laser beam pointed in each arm direction. 
On normal circumstances, the beams from two different arms can be set to be (`locked') on a dark or bright fringe of the interferometer diffraction pattern.
When a GW passes by, it will alter the relative distances of the mirror masses, thus changing the optical path of the laser beams. This in turn, will translate into a shift in the diffraction pattern consistent with the deformation of space.

In the following section I offer a more detailed account on the the operation of the first of these Earth based interferometers, the LIGO observatory.


\section{The LIGO-Virgo Collaboration}

\begin{comment}
The following text is mine, but too convoluted

The test masses for LIGO are two massive mirrors at the end of two perpendicular long arms (4 km long.) Each mirror weighs about 40 kg and each one of the arms can be thought, for simplicity, as an independent laser interferometer.

When a GW passes at an angle to this set-up, the arms will stretch and shorten in opposite directions (one arm will stretch as the other shrinks), while the laser keeps traveling the arms unaffected. This causes minuscule differences in optical paths for the light and this in turn will result in the lasers being out of phase from each other. This difference in phase from the `lock' position indicates a change in the relative positions of the mirrors.

Many other mundane situations can also produce this same effect, the most simple one being seismic activity of almost any strength. Additionally, noise from the instrument itself, from the laser beam and electronics, as well as many intrinsic vibration modes of the system complicates the output signal that will be analyzed.
\end{comment}

LIGO --the Laser Interferometer Gravitational-Wave Observatory-- (Abramovici et al. 1992) is an American observatory set to detect the GW predicted by General Relativity.

It actually consists of two separate observatories, one in the state of Washington and another one in Louisiana.

Co-founded in 1992 by Kip Thorne and Ronald Drever of Caltech and Rainer Weiss of MIT, LIGO is a joint project between scientists at MIT, Caltech, and many other colleges and universities.

Virgo is a similar observatory in Europe. Originally a project from France and Italy, it soon became a collaboration from five different countries: France, Italy, the Netherlands, Poland, and Hungary. The Virgo observatory is located in the countryside near Pisa, Italy.

Since 2007, LIGO and Virgo share their data in an umbrella collaboration named LVC, the LIGO-Virgo Collaboration.

Even though, LIGO and Virgo have comparable noise levels and detection rates, only the two interferometers in LIGO were operational at the time of the GW150914 event. 
In fact, Virgo was being upgraded to Advanced Virgo, which will have a sensitivity 10 times greater than Initial Virgo. 
When the upgrade is finished, it will join LIGO in the joint collection of GW data. This will improve errors in the parameter estimation, especially localization errors will greatly improve.

LIGO went through similar upgrades along the years.

The initial LIGO detectors were designed to be sensitive to GWs in the frequency band 40-7000 Hz, and capable of detecting a GW strain amplitude as small as $10^{-21}$.

To picture the order of magnitude of these displacements, consider that the change in length of one arm of the interferometer is only about $10^{-18}$ m, a thousand times smaller than the diameter of a proton.

To reach this sensitivity, the detectors need highly stable lasers, multiple layers of vibration isolation and advanced optic techniques.

From the initial period, LIGO had five short science runs (S1 to S5), each one improving over the previous one, culminating with S5 at design sensitivity. 
The S5 run collected a full year of triple-detector coincident data from November 2005 to September 2007.

Between Initial LIGO and Advanced LIGO more science were made under Enhanced LIGO, which provided enhancements that improved sensitivity by a factor of 2. 

But the real revolution came with Advanced LIGO, a set of additions that improved LIGO sensitivity by a factor of 10 over Initial LIGO and widened the frequency range all the way down to 10 Hz (which is known as the seismic wall).


Among the improvements is the upgrading of the laser wattage to 200 W. 
An increase in the test masses to 40 kg in order to reduce radiation pressure noise and to allow larger beam sizes. 
Larger beams and better dielectric mirror coatings combine to reduce the test mass thermal noise by a factor of 5 compared with initial LIGO.
An improvement in vibration isolation, including vertical isolation comparable to the horizontal isolation all almost all stages.
New suspension system based on fused silica rather than steel wires to reduce suspension thermal noise by almost a hundred.
New two-stage active seismic isolation instead of the passive one-stage isolation brought the seismic noise to negligible levels above approximately 10 Hz.

All these improvements combined gave Advanced LIGO a 10-fold increase in sensitivity. 
This increase also means that fainter sources can now be detected, increasing the exploration volume for GW by a factor of a thousand.

Advanced LIGO will have several observations runs, labelled O1 through OX, (similar to the scientific runs S1-S5) with gaps between them on which more improvements will be implemented until it reaches design sensitivity in OX. 

The projected sensitivity for each of the runs is presented in table X

\begin{comment}
%The successful operation of Advanced LIGO is expected to transform the field from GW detection to GW astrophysics. We illustrate the potential using compact binary coalescences. Detection rate estimates for CBCs can be made using a combination of extrapolations from observed binary pulsars, stellar birth rate estimates and population synthesis models. There are large uncertainties inherent in all of these methods, however, leading to rate estimates that are uncertain by several orders of magnitude. We therefore quote a range of rates, spanning plausible pessimistic and optimistic estimates, as well as a likely rate. For a NS mass of 1.4M⊙ and a BH mass of 10M⊙, these rate estimates for Advanced LIGO are: 0.4– 400 yr−1, with a likely rate of 40 yr−1 for NS–NS binaries; 0.2–300 yr−1 , with a likely rate of 10 yr−1 for NS–BH binaries; 2–4000 yr−1 , with a likely rate of 30 yr−1 for BH–BH binaries.

LIGO was designed so that its data could be searched for GWs from many different sources. The sources can be broadly characterized as either transient or continuous in nature, and for each type, the analysis techniques depend on whether the gravitational waveforms can be accurately modeled or whether only less specific spectral characterizations are possible. We therefore organize the searches into four categories according to source type and analysis technique.
%(i) Transient, modeled waveforms: the compact binary coalescence search. The name follows from the fact that the best understood transient sources are the final stages of binary inspirals [52], where each component of the binary may be a NS or a BH. For these sources the waveform can be calculated with good precision, and matched-filter analysis can be used.
(ii) Transient, unmodeled waveforms: the gravitational-wave bursts search. Transient systems, such as core-collapse supernovae [53], BH mergers and NS quakes, may produce GW bursts that can only be modeled imperfectly, if at all, and more general analysis techniques are needed.
(iii) Continuous, narrow-band waveforms: the continuous wave sources search. An example of a continuous source of GWs with a well-modeled waveform is a spinning NS (e.g. a pulsar) that is not perfectly symmetric about its rotation axis [54].
(iv) Continuous, broadband waveforms: the gravitational-wave background search. Processes operating in the early universe, for example, could have produced a background of GWs that is continuous but stochastic in character [55].
\end{comment}

\section{The Kilonova, A Case For Multi-Messenger Astronomy}

%In physics, GW detection could provide information about strong-field gravitation, the untested domain of strongly curved space where Newtonian gravitation is no longer even a poor approximation. In astrophysics, the sources of GWs that LIGO might detect include binary NSs (like PSR 1913 + 16 but much later in their evolution); binary systems where a black hole (BH) replaces one or both of the NSs; a stellar core collapse which triggers a type II supernova; rapidly rotating, non-axisymmetric NSs; and possibly processes in the early universe that produce a stochastic background of GWs [3].

With LIGO we have just pierced a new window in the purely relativistic universe.

LIGO is sensitive to phenomena in the 10Hz to 7kHz frequency band.

Similarly to the optical case, once other frequencies were explored more science could be done, with other detection methods like PTA (Pulsar Timing Array) and LISA will shed new light to other physical phenomena and physics.

Two things brought about radical changes in Astronomy: Multi-Messenger Astronomy and Time Domain Astronomy.


It is known that short GRBs are basically NS-NS mergers.
GRB emission is collimated and jetted, so the chances to detect one along a GW are very low.
NS-NS also emit EM radiation of lower energy or frequency by a different process that occurs in the jet but is nonetheless isotropic. 
Actually several other mechanisms of neutron and particle interactions create radiation in various wavelengths and durations.
The most significant of those is no doubt, the Kilonova emission produced by rapid capturing of neutrons.
Neutron capture has to be faster than the beta decay rate of the neutron and that's why it has to be rapid.
This capture process is called r-process. R is for rapid.
The r-process physics is quite complicated and involves a bunch of stuff, much of which is modeled to certain confidence, but many other elements are not well known. Several ingredients to the model are not considered fully. 
There are other neutron physics, like interaction of the neutron rich jet with the medium that gives raise to other EM radiation in the radio, blue and other bands. But these are second order radiations or they enhance the Kilonova emission.

In the late 50's Burbidge et al. (1957) and Cameron (1957) had already proposed that approximately half of the elements heavier than iron are synthesized via the capture of neutrons onto lighter seed nuclei (e.g., iron) in a dense neutron-rich environment in which the timescale for neutron capture is shorter than the $\beta$ decay timescale.
Rapid neutron capture process', or r-process for short, 
Despite this mechanism was known for long time, the astrophysical environments in which this happens remained a mystery.

Core collapse SNe have long been considered promising r-process sources.

The r-process efficiency or cross section depends on a number of factors.
electron fraction $Y_{e}$ less than but about 0.5
They showed that the radioactive heating rate was relatively insensitive to the precise electron fraction of the ejecta, and they were the first to consider how efficiently the decay products thermalize their energy in the ejecta.

Even prior to the discovery of the first binary pulsar (Hulse \& Taylor 1975), Lattimer and Schramm (1974, 1976) proposed that the merger of compact star binaries in particular the collision of BH-NS systems could give rise to the r-process by the decompression of highly neutron-rich ejecta (e.g. Meyer 1989). Blinnikov et al. (1984) and Paczynski (1986) first suggested a connection be- tween NS-NS mergers and GRBs.

Freiburghaus et al. (1999) presented the first explicit calculations showing that the ejecta properties extracted from a hydrodynamical simulation of a NS-NS merger (Rosswog et al. 1999) indeed produces abundance patterns in basic accord with the solar system r-process.

Li \& Paczynski (1998, LP98) first showed that the radioactive ejecta from a NS-NS or BH-NS merger provides a source for powering transient emission, in analogy with Type Ia SNe. Given the low mass and high velocity of the ejecta from a NS-NS/BH-NS merger, they concluded that the ejecta will become transparent to its own radiation quickly, producing emission which peaks on a timescale of about one day, much faster than for normal SNe (which instead peak on a timescale of weeks or longer).

Metzger et al. (2010) also highlighted the critical four-way connection between kilonovae, short GRBs, GWs from NS-NS/BH-NS mergers, and the astrophysical origin of the r-process nuclei.

Once the radioactive heating rate was determined, attention turned to the even thornier issue of the ejecta opacity. The latter is crucial since it determines at what time and wavelength the ejecta becomes transparent and the light curve peaks. 

As compared to the earlier predictions (e.g. Metzger et al. 2010), these higher opacities push the bolometric light curve to peak later in time (about 1 week instead of a 1 day timescale), and at a lower luminosity (Barnes \& Kasen, 2013). More importantly, the enormous optical opacity caused by line blanketing moved the spectral peak from optical/UV frequencies to the near-infrared (NIR).

Later that year, Tanvir et al. (2013) and Berger et al. (2013) presented evidence for excess infrared emission following the short GRB 130603B on a timescale of about one week using the Hubble Space Telescope. If confirmed by future observations, this discovery would be the first evidence directly relating NS mergers to short GRBs, and hence to the direct production of r-process nuclei.

Based on their de- rived peak luminosities being approximately one thousand times brighter than a nova, Metzger et al. (2010) first introduced the term `kilonova' to describe the EM counterparts of NS mergers powered by the decay of r-process nuclei

\subsection{The Physics of the Kilonova}

Consider the merger ejecta of total mass M, which is expanding at a constant velocity v, such that its radius is $R \approx vt$ after a time $t$ following the merger. 
We assume spherical symmetry, good first-order approximation
The ejecta is hot immediately after the merger, especially if it originates from the shocked interface between the colliding NS-NS binary. 
This thermal energy cannot, however, initially escape as radiation because of its high optical depth at early times
and the correspondingly long photon diffusion timescale through the ejecta. 
As the ejecta expands, the diffusion time decreases inversely proportional with time, until eventually radiation can escape on the expansion timescale.
This condition determines the characteristic timescale at which the light curve peaks
For values of the opacity $\kappa \approx $1-100 $\textmd{cm}^{-2}\textmd{g}^{-1}$ which characterize the range from Lanthanide-free and Lanthanide-rich matter, respectively, the derivation predicts characteristic durations from about 1 day to 1 week.

The temperature of matter freshly ejected at the radius of the merger (about 100 km) exceeds billions of degrees. 
However, absent a source of persistent heating, this matter will cool through adiabatic expansion, losing all but a fraction about (R0/Rpeak) about 1E-9 of its initial thermal energy before reaching the radius $R_{peak} = v t_{peak}$ at which the ejecta becomes transparent.
Such `adibatic losses' would leave the ejecta so cold as to be effectively invisible.
In reality, the ejecta will be continuously heated by a combination of sources. 
At a minimum, this heating includes contributions from radioactivity due to r-process nuclei (and possibly free neutrons), while, more speculatively, the ejecta can be heated from within by a central engine, such as a long-lived magnetar or accreting BH.

So, Kilonova emission have 3 main ingredients:

* The mass and velocity of the ejecta from NS-NS/BH-NS mergers.
* The opacity $\kappa$ of expanding neutron-rich matter.
* The variety of sources which contribute to heating the ejecta, particularly on timescales when the ejecta is first becoming transparent.

Regarding the sources heating the ejecta, is the r-process radioactive heating the main one.
At a minimum, the ejecta receives heating from the radioactive decay of heavy nuclei synthesized in the ejecta by the r-process. 

The radioactive decay rate is also largely insensitive to uncertainties in the assumed nuclear masses, cross sections, and fission fragment distribution (although the r-process abundance pattern will be e.g. Eichler et al. 2015; Wu et al. 2016; Mumpower et al. 2016).

Radioactive heating occurs through a combination of $\beta$-decays, $\alpha$-decays, and fission (Metzger et al. 2010, Barnes et al. 2016, Hotokezaka et al. 2016). 

\subsection{The red and blue kilonovas}

There are red emission and blue emissions.
Depending on the type of ejecta, the emission can be in the NIR or bluer, in the visual bands R and I.

In the tidal tails in the equatorial plane, or in more spherical outflows from the accretion disk in cases when BH formation is prompt or the HMNS phase is short-lived, the highly neutron-rich matter (Ye < 0.29) will form heavy r-process nuclei.
This r-process will peak in the near infra-red (NIR) at J and K bands (1.2 and 2.2 μm, respectively) on a timescale of several days to a week.

In addition to the highly neutron-rich ejecta (Ye < 0.29), growing evidence suggests that some of the matter which is unbound from a NS-NS merger is less neutron rich (Ye > 0.29; e.g. Wanajo et al. 2014a; Goriely et al. 2015) and thus will be free of Lanthanide group elements (Metzger \& Fernandez 2014). This low-opacity ejecta can reside either in the polar regions, due to dynamical ejection from the NS-NS merger interface, or in more isotropic outflows from the accretion disk in cases when BH formation is significantly delayed.
By assuming a lower opacity appropriate to Lanthanide-free ejecta, the emission now peaks at the visual bands R and I, on a timescale of about 1 day at a level 2-3 magnitudes brighter than the Lanthanide-rich case.

In general, the total kilonova emission from a NS-NS merger will be a combination of `blue' and `red' components, as both high- and low-Ye ejecta components could be visible for viewing angles close to the binary rotation axis (Fig. 4). For equatorial viewing angles, the blue emission is likely to be blocked by the higher opacity of the lanthanide-rich equatorial matter (Kasen et al. 2015). Thus, although the week-long NIR transient is fairly generic, an early blue kilonova will be observed in only a fraction of mergers.

In addition to the blue and red components, recent NS-NS merger simulations show that a small fraction of the dynamical ejecta (typically a few percent, or about 1E-4 solar masses) expands sufficiently rapidly that the neutrons do not have time to be captured into nuclei (Bauswein et al., 2013a). This fast expanding matter, which reaches asymptotic velocities v about 0.4-0.5 c, originates from the shock- heated interface between the merging stars and resides on the outermost layers of the polar ejecta. This `neutron skin' can super-heat the outer layers of the ejecta, enhancing the early kilonova emission (Metzger et al. 2015; Lippuner \& Roberts 2015).

\subsection{Kilonova from a magnetar remnant}

As described in 3.1, the type of compact remnant produced by a NS-NS merger (e.g. prompt BH formation, hypermassive NS, supramassive NS, or indefinitely stable NS) depends sensitively on the total mass of the binary relative to the maximum mass of a non-rotating NS, Mmax($\Omega$ = 0). The value of Mmax($\Omega$ = 0) exceeds about 2solar masses (Demorest et al. 2010, Antoniadis et al. 2013) but is otherwise unconstrained13 by observations or theory up to the maximum value of about 3 solar masses set by the causality limit on the EOS. A `typical' merger of two 1.3 to 1.4 solar masses NS results in a remnant mass of 2.3 to 2.4 solar masses after accounting for neutrino losses and mass ejection (e.g., Belczynski et al. 2008). If the value of Mmax($\Omega$ = 0) is well below this value (e.g. 2.1-2.2 solar masses), then most mergers will undergo prompt collapse or form hypermassive NSs with very short lifetimes. On the other hand, if the value of Mmax($\Omega$ = 0) is close to or exceeds 2.3-2.4 solar masses, then a order unity fraction of NS-NS mergers could result in long-lived supramassive or indefinitely stable remnants.

If the rotational energy could be extracted in non-GW channels on timescales of hours to years after the merger (e.g., by magnetic dipole radiation), this could substantially enhance the EM emission from NS-NS mergers (e.g. Gao et al. 2013; Metzger \& Piro 2014; Gao et al. 2015; Siegel \& Ciolfi 2016a). However, for NSs of mass Mns   Mmax(Ω = 0), only a fraction of the rotational energy is available to power EM emission, even in principle. This is because the loss of angular momentum that accompanies spin-down results in the NS collapsing into a BH before all of its rotational energy is released.

Nonetheless, there are several mechanisms to extract rotational energy from the indefinitely stable magnetar remnant.
There is plenty literature on the subject that suggests that rotational energy input from a stable magnetar could enhance kilonova emission. The emission is still red in color and peaks on a timescale of 1 to 2 weeks, but the luminosity is greatly enhanced compared to the radioactive case, with peak magnitudes of $K \approx$ 18-20.

\subsection{Sources of ejecta}

The type of radiation depends on the EOS for the ejecta and its thermodynamical properties. Opacity, nucleon and particle content, pressure, temperature, MHD state.

Matter ejected either by tidal forces or due to compression-induced heating at the interface between merging bodies.
Unbound debris can have enough angular momentum to form a disk around the merge.
Outflows from this remnant disk, taking place on longer timescales of up to seconds, provide a second important source of ejecta

In the case of a NS-NS merger, the ejecta properties depend sensitively on the fate of the massive NS remnant which is created by the coalescence event.

\subsection{Fate of the merged system}

The end product of a NS-NS or BH-NS merger is a central compact remnant, either a BH or a massive NS. 

The last stages of the system will also effect the emission of the kilonova.
The system can form a temporary merged NS supported by rotation but will eventually decay into a BH after a time that can go from ms to minutes or much longer. [Explain here all 3 cases, HMSN SMSN and BH]

The final system depends sensitively on the total mass of the original NS- NS binary (e.g., Shibata \& Uryu 2000; Shibata \& Taniguchi 2006). Above a threshold mass of Mcrit $\approx$ 2.6-3.9 solar masses the remnant collapses to a BH essentially immediately, on the dynamical time of milliseconds or less (Hotokezaka et al. 2011; Bauswein et al. 2013a).

The maximum mass of a NS, though primarily sensitive to the NS EOS, can be increased if the NS is rotating rapidly (e.g., Baumgarte et al. 2000, Ozel et al. 2010, Kaplan et al. 2014).
If the mass is supported exclusively by differential rotation it's called a hypermassive NS (HMNS).
This decays into a BH in a few ms.
If the mass can be supported by solid body rotation is called a supramassive NS.
This can decay in a BH by a less effective mechanism and can remain stable for minutes or much longer periods.
The distinction between these two is important because energy input from long lived remnants could substantially enhance the kilonova emission.

The merger of a binary with a total mass less than the maximum mass of a non-rotating NS (dependent on the particular EOS but around 2 solar masses), will produce an indefinitely stable remnant, from which a BH never forms (e.g., Metzger et al. 2008; Giacomazzo \& Perna 2013).



------------------------------

Besides NS-NS, a NS with a BH is also posible and it will, under some circumstances give raise to a Kilonova. The physics in this case is a bit more complicated and the parameters of the BH play a big role in determining the Kilonova shape or if there's one at all.

BH-BH mergers have no EM counterpart, except perhaps in very specific situations. This is mainly due to lack of baryonic matter.

Back to NS-NS mergers, the emission is isotropic and peaks in the infrared under normal circumstances.


Burbidge et al. (1957) and Cameron (1957) realized that approximately half of the elements heavier than iron are synthesized via the capture of neutrons onto lighter seed nuclei (e.g., iron) in a dense neutron-rich environment in which the timescale for neutron capture is shorter than the $\beta$-decay timescale.

The ejecta from NS mergers are an astrophysical source of rapid neutron-capture (r-process) 

Identifying host galaxies of GW is important. We can know:
	Age of stellar population.
	Displacement due to SN birth kicks.
	Determine distance to GW source. This reduces degeneracies in the GW parameter estimation, especially of the binary inclination with respect to the line of sight.

Stellar mass BH-BH binaries are not expected to produce luminous EM emission because there are no baryonic matter.
Either NS-NS or BH-NS is needed to have a full synthesis of GW and EM skies.
Population synthesis models of field binaries predict GW detection rates of NS-NS/BH-NS mergers of $\approx$ 0.2-300 per year, once Advanced LIGO/Virgo reach their full design sensitivities near the end of this decade (e.g. Abadie et al. 2010, Dominik et al. 2015).
Empirical rates based on observed binary pulsar systems in our galaxy predict a comparable range, with a best bet rate of $\approx$ 8 NS-NS mergers per year (Kalogera et al. 2004; Kim et al. 2015).

Sky location is ligo is very poor because it's based mainly on triangulation of detectors.
	Sky error regions are very large (e.g. $\approx$ 850 deg2 for GW150914; Abbott et al. 2016).
	With Virgo and KAGRA and INDIGO will be of of 10-100 square degrees or less (e.g., Fairhurst 2011, Nissanke et al. 2013, Rodriguez et al. 2014).
	It still greatly exceeds the fields of view of most radio, optical, and X-ray telescopes.
	
Observational (e.g., Fong et al. 2013) and theoretical (e.g. Eichler et al. 1989, Narayan et al. 1992) evidence suggest a relation between merges with at least one NS and the ``short duration'' class of GRBs (Nakar 2007, Berger 2014). 

[literal] Short GRBs are commonly believed to be powered by the accretion of a massive remnant disk onto the compact BH or NS remnant following the merger. This is typically expected to occur within seconds of the GW chirp, making their temporal association with the GWs unambiguous (the gamma-ray sky is otherwise quiet).

GRB are the cleanest EM counterpart (why? don't know)
Rate of GRBs from NS-NS mergers is low, less than once per year all-sky. (e.g. Metzger \& Berger 2012)
We should not expect the first --or even the first several dozen-- GW chirps from NS-NS/BH-NS mergers to be accompanied by a GRB.

For the majority of GW-detected mergers, the jetted GRB emission will be relativistically beamed out of our line of sight.
The off-axis afterglow probably does not provide a promising counterpart for most observers


NS-NS/BH-NS mergers are also predicted to be accompanied by a more isotropic counterpart, commonly known as a `kilonova'. Kilonovae are day to week-long thermal, supernova-like transients, which are powered by the radioactive decay of heavy, neutron-rich elements synthesized in the expanding merger ejecta (Li \& Paczynski 1998). They provide both a robust EM coun- terpart to the GW chirp, which is expected to accompany a fraction of BH-NS mergers and essentially all NS-NS mergers, as well as a direct probe of the unknown astrophysical origin of the heaviest elements (e.g., Metzger et al. 2010).



\section{The TOROS Collaboration \textcolor{red}{(or project?)}}
* Lengthy TOROS Description (whatever it evolved into at this point)

TOROS, the Transient Optical Robotic Observatory of the South, is a collaboration formed to respond to GW events as detected by the LVC.
Its main purpose then is to detect transient events compatible with expected (or not) signatures of events that can give mutual birth to GW and optical triggers.

Being in its early stages of development, and given the new nature of the multi-messenger astronomy, the TOROS team is building up the tools and infrastructure that will make it capable in the future to promptly respond to this alerts.

Several things can enter in consideration for this task, from software development to forging ties with existing observatories. 

My thesis will focus on several challenges in the development of the software infrastructure needed to process the observatory data. 

At the moment of this writing, TOROS Collaboration consists of several astronomical institutions that showed interest in doing a search and possible follow-ups of candidates to interesting events.

The list of participant institutions is as follows: (should I list them?)

At UT Rio Grande Valley, we developed extensive analysis and web code to allow for the interaction between the institutions (just the broker page, really). 


\section{The Transient World}

\section{The Time-Domain Revolution in Astronomy}
* GW20150914 and our contribution

* The transient world
	* The Time Domain revolution in Astronomy
	* What are transients and which ones are we interested in

	* How can we catch transients
		* Previous work: Catalog Matching: Marica's pipeline and Shuang's work on the transient identification and comparison to my OIS code.
		* Previous work: OIS + ML (the iPTF experience)
		* Subsection on ML (Supervised vs. Unsupervised, overall view and some popular algorithms)
		* Discussion of both methods


\chapter{Difference Image Analysis} \label{appendix:dia}

Searches for time-varying and position-changing objects are undertaken by comparing a reference image of a particular region of the sky with a second image taken at the moment in which we are interested. Ideally, the reference image is taken with the same telescope using the same filter and CCD. The two images have to be aligned pixel by pixel and then subtracted to reveal any changes in light.

To make a suitable subtraction of two images, one has to match the frames to exactly the same seeing. Image Difference is a technique to find a convolution kernel that best describes (in some minimization sense) the change in point spread function between the images. The idea is to degrade the good seeing image---our reference image---to match the seeing of our second image. Finding the proper kernel can be a delicate operation and there's plenty of literature on the subject. Methods range from PSF modeling through common Gaussian profiles to unmodeled PSF's to Information Theory and Fourier Domain.

The first attempts at image subtraction relied on Fourier decomposition of the images, 
but the technique suffered when noise levels were even moderate and the results were not always good.
\citet{1998ApJ...503..325A} were the first to propose a solution in image space (as opposed to Fourier space). 
They also summarize previous efforts in their introduction and references therein. 
In that paper, they propose an optimization problem that we describe briefly as follows.

We have an image $I$ and a reference image $R$, for which we want to find a convolution kernel $K$ such that

\begin{align}
I(x,y) & \approx (R \mathbin{*} K)(x,y) \\
 & \approx \int \mathrm{d}u \mathrm{d}v {R (u,v) K(x-u,y-v)}
\end{align}
 
The integral symbol has to be understood in practice, as a sum over all the pixels $(x,y)$ on the image. 

The kernel $K$ will try to correct for the PSF difference between the two images.
It's worth noting here, that even though it could be practical to the reader to think so, the kernel $K$ is neither the PSF of the reference nor the PSF of the image.

Usually, the reference image is the one with the best {\em seeing}, 
because it can be done by median-stacking good images, or using Lucky Imaging [add ref] or some other method.

To convert the problem into a linear one, we decompose the kernel into a linear combination of ``{\em basis}'' functions $B$.

\begin{equation} \label{kernel_linear}
K(u,v) = \sum_{i} a_{i} B_{i}(u,v)
\end{equation}

These $B_{i}$ could in principle be any reasonable set of functions. 
%The two most popular choices are modulated Gaussians and the Delta basis, which will be explained in further detail in section \ref{basis}.

Using this linear combination, the convolution will be

\begin{align}
(R \mathbin{*} K)(x,y) & = \sum_{i} a_{i} \left( R \mathbin{*} B_{i} \right)(x,y) \\
 & \equiv \sum_{i} a_{i} C_{i}(x,y)
\end{align}

where the last line defines $C_{i}(x,y)$.

With this decomposition, we can find the set of $a_{i}$ that minimizes the square difference over all pixels.
Define a cost function $Q$:

\begin{align}
Q &= \int \left( I(x,y) - (R \mathbin{*} K)(x,y) \right)^2 \\
 & = \int \left( I(x,y) - \sum_{i} a_{i} C_{i}(x,y) \right)^2,
\end{align}

and let's minimize $Q$ over the set of $a$'s:

\begin{align}
\frac{\partial Q}{\partial a_{i}} = & 2 \int \left( I(x,y) - \sum_{j} a_{j} C_{j}(x,y) \right) C_{i}(x,y) 
\end{align}

Setting the last equation to zero, gives us:

\begin{align}
\sum_{j} a_{j} \int \left( C_{j}(x,y)  C_{i}(x,y) \right) =  \int I(x,y) C_{i}(x,y) 
\end{align}

Which we can write more succinctly as a matrix equation:

\begin{align} \label{matrix_eq}
\sum_{j} M_{ij} a_{j}  =  b_{i}
\end{align}

where

\begin{align} \label{matrix_def}
M_{ij}  &=   \int  C_{i}(x,y)  C_{j}(x,y) \\
b_{i} &=  \int I(x,y) C_{i}(x,y)  \nonumber
\end{align}

We can find the coefficients $a_{i}$ of the optimal kernel for the subtraction by inverting the system \eqref{matrix_eq}.

\section{Different basis functions}

As stated above, any choice of basis in the linearization \eqref{kernel_linear} could work, but two particular choices are the most popular.

The first one was proposed by \citet{1998ApJ...503..325A} and it consist of modulated centered Gaussians:

\begin{equation}
B_{n,d_n^{x}, d_n^{y}}(u,v) = e^{-(u^2+v^2)/2 \sigma_n^2} \times u^{d_n^{x}} v^{d_n^{y}}
\end{equation}

where the exponents in $u$ and $v$ add at most up to $D$, the degree of the modulation polynomial.

This choice of linearization gives enough freedom to approximate the kernel as a sum of modulated Gaussians, which is suitable for many situations.
The parameter $\sigma_n$ in each Gaussian is fixed beforehand by the user.

The number of Gaussians used in the expansion is given by $n$, and for each one of them we have $(D_n + 1)(D_n + 2)/2$ terms in the modulating polynomial.
That gives a total of $n(D_n + 1)(D_n + 2)/2$ unknown $a_i$ coefficients to solve for.

A simpler basis was proposed by \citet{2008MNRAS.386L..77B} and it's effectively a Dirac Delta function for each pixel.

\begin{equation}
B_{i}(x,y) = \delta(x-i,y-j)
\end{equation}

This choice of basis makes every pixel value in the kernel be determined independently by the minimization process.
Obviously, this allows for a greater variety of kernels than in the Gaussian case.
It also comes at a cost: now the number of unknowns to invert for grows quadratically with the kernel side length.
For a kernel of side 11 pixels, we have to create a matrix 121$\times$121 and then invert it. 

As we will see at the end of the chapter, even calculating the elements of the matrix to invert is an expensive operation,
so this absolute freedom in the kernel shape comes at the cost of a much higher numerical complexity.

Despite this complexity, the Delta basis can account for situations that the Gaussian basis can't address.
For example, if the images are very similar in PSF, then the compensating kernel of our problem should be an actual delta at the center 
---the identity kernel---or a very peaky function.

Bramich's method can actually return the correct kernel, while the Gaussian method will have trouble adjusting (potentially broad) Gaussians.

Another issue that Delta basis corrects very well is for tiny misalignments between our reference image and the image we are processing.
In fact, translations are included in the set of convolutions represented by displaced deltas on the kernel.
Convolving with a kernel with a delta displaced $(\Delta x, \Delta y)$ pixels away from the center will effectively translate the image by that same amount,
so misalignments due to small translations (the order of the kernel side) can be completely accounted for in the Delta basis.

\section{Add a varying background}

Background variation can be treated separately or simultaneously with the PSF fitting.

\subsection{Independent background estimation}

An independent background estimation can be done on each separate image before doing the PSF match.

For a stellar field image, one can create an image $I_{B}$, from the image $I$ by excluding all pixels above a certain threshold on the background noise. This image $I_{B}$ will contain pixels belonging to the background only (sources removed).

\begin{equation}
I_{B}(x,y)  = \left\{ I(x,y) : |I(x,y) - \mu| < \sigma \right\}
\end{equation}

On this image $I_{B}$, find the best polynomial fit of degree $d$ to the image using a least square fit.

Minimizing $Q$ over the $b_{ij}$ coefficients

\begin{equation}
Q = \int \left( I_{B}(x,y) - \sum_{i,j}^d b_{ij} x^i y^j \right) ^2
\end{equation}

will give the best polynomial fit to the background.

\subsection{Simultaneous PSF and background estimation}

To do it simultaneously with the PSF matching, we simply add it to our previous $Q$. This will let us remove any remaining variation on our new image $I(x,y)$.

\begin{align}
I(x,y) & \approx (R \mathbin{*} K)(x,y) + B(x,y) 
\end{align}

Note however, that this $B$ is not the sourceless image $I_{B}$ defined in the previous section. This $B$ represents the optimal background approximation that can be expanded as a polynomial with unknown coefficients.

$Q$ is defined now as

\begin{align}
Q &= \int \left( I(x,y) - (R \mathbin{*} K)(x,y) - B(x,y) \right)^2 \\
 & = \int \left( I(x,y) - \sum_{i} a_{i} C_{i}(x,y) - \sum_{i} b_{i} x^i y^j \right)^2
\end{align}

We can pile up the $b$ coefficients onto a larger set of $a$'s and define new $C$'s accordingly.

\begin{equation}
C_{i}(x,y)  = \begin{cases} 
(R \mathbin{*} K)(x,y)  &\mbox{when i,j refer to kernel}  \\ 
x^i y^j & \mbox{when i,j refer to background}  
\end{cases} 
\end{equation}

This leads us to the exact same (but extended) solution for the coefficients $a$, namely:

\begin{align}
\sum_{j} M_{ij} a_{j}  =  b_{i}
\end{align}

where

\begin{align}
M_{ij}  &=   \int  C_{i}(x,y)  C_{j}(x,y) \\
b_{i} &=  \int I(x,y) C_{i}(x,y) 
\end{align}

as before.

\section{How to deal with bad pixels} \label{badpixels}

The astronomical images can have pixel defects due to CCD defects or missing data from alignment. 
Those pixels can, in principle, be in both the reference frame $R$ and the new image $I$. 

Although it is advisable to use a reference image $R$ with few to none bad pixels for the reason that will be explained further on this section.

As before, we now want to approximate the two images as follow

\begin{align}
I(x,y)\bigg|_{\Omega} & \approx (R \mathbin{*} K)(x,y)\bigg|_{\Omega}
\end{align}

But this time, we want to restrict the above only for the good pixels in the images.

Bad pixels in $I$ and $R$ have to be excluded, but we also have to exclude those pixels in $R$ that {\em use} bad pixels in the convolution. This equivalent to dilate the bad pixels mask in $R$ with a kernel the same shape as the convolution kernel $K$. For this reason, it's better to have few bad pixels in $R$.

The union of the bad pixel mask from $I$ and the dilated bad pixel mask from $R$ is the common bad pixel mask. We call $\Omega$ to its complement, so that $\Omega$ is the set of good pixels that will be used in the subtraction. Pixels outside $\Omega$ will be tainted by the bad pixels in either of the two images.

This modification will now makes us define a new $Q$:

\begin{align}
Q = \int_{\Omega} \left( I(x,y) - (R \mathbin{*} K)(x,y) - B(x,y) \right)^2
\end{align}

That only differs from the previous one by the domain of integration (or sum).
The definitions for $M$ and $b$ are similarly derived:

\begin{align}
M_{ij}  &=   \int_{\Omega}  C_{i}(x,y)  C_{j}(x,y) \\
b_{i} &=  \int_{\Omega} I(x,y) C_{i}(x,y) 
\end{align}

\section{Dealing with large fields of view}

When dealing with large field of views, a simple solution suggested by Bramich 2010, is to partition the image into grids, and apply Image Differencing on each grid element.

Another solution is to include in the derivation, a space-varying kernel.
This has the advantage of addressing the issue directly, but we have to quit to the niceties of having convolutions done with FFT.

The derivation follows the usual least squares derivation, except this time we consider each kernel basis element modulated by a polynomial variation across the image. Following Miller (2008):

\begin{align}
K(u, v) &= \sum_n a_n(x,y) B_n(u, v) \\
&= \sum_{n,i,j} a_{nij} x^i y^j B_n(u, v)
\end{align}

The new equations for $M$ and $b$ as in \ref{matrix_def} are the same, except that $C_{i}$ is now defined as:

\begin{equation}
C_I = \left( R \mathbin{*} x^i y^j B_n \right) (x,y)
\end{equation}

and $I$ is here the collective index $\{n,i,j\}$.
Notice that the last equation involves an unusual type of ``convolution'' where the kernel is not constant. This prevents the use of FFT to speed up the calculation.

\section{Cost of building matrix M and b}

Recall that we have to solve the system of equations \eqref{matrix_eq} with definitions in \eqref{matrix_def}.

This implies that each component of the matrix $M$ will involve a convolution and an integration over the whole image.

Therefore, not only the inversion problem is expensive, but even calculating the matrix can be an expensive operation.



\chapter{Machine Learning}

The subtraction techniques are very effective at  modeling PSF differences, nevertheless there are many defects left behind after a subtraction. This is also a well known issue in the difference image analysis. The fictitious (or bogus) sources arise primarily from PSF mismatch and mis-alignments of the images.

These defects can easily confuse algorithms of source detection like SExtractor or similar that relies on excess of flux over the background. For that, it is needed to have a computerized agent capable of discriminating bogus sources from real transients on the subtracted image. 

This is the task of a Machine Learning (ML) agent trained for that purpose. The real bogus classifier, as it is named in the literature relies on --usually morphological-- features to perform the classification.

Machine Learning is such an extensive and intensive area of research with many applications in many fields.

Since the ML field is so wide, to conduct a ML experiment one must make a few decisions. Even most importantly than the particular algorithm, for which there are hundreds and modifications are being published all the time, it's the data that drives the efficiency of the method.

As the saying goes:

\begin{quotation}
Big data will beat a good algorithm \textcolor{red}{Find exact quote and author}
\end{quotation}


For our main test, I decide to train a Random Forest ML algorithm on a training set developed especially for this.

The Random Forest algorithm is an `ensemble' method. Meaning it is a collection of other methods whose results will be averaged over somehow.

In this case, a Random Forest is an `ensemble' of Decision Trees. A Decision Tree is basically a tree of nodes. Each node represent a bifurcation based on one feature. The features and branching threshold on each node are chosen maximizing the expected information gain (or entropy loss) on the training set. That is, how well the bifurcation separates the training data for each class.

Random Forest bootstraps the data so that each subset of data will train a Decision Tree on a random subset of the features. After all Decision Trees are trained this way, the collection of Trees will give the final classification.

A single Decision Tree suffers of great variance, even for big amounts of data. The bifurcations on the greedy algorithm are naturally very dependent on the training set. A small variation on it may create a complete different tree. Random Forest reduces this variance by averaging over many trees. This also creates more realistic and fuzzy class boundaries on the feature space.

On the next section I explain how the training data was prepared.

\section{Training Data}

The main challenge to generate training data is the great unbalance between bogus sources due to imperfect subtraction and actual real transients.

On a typical image, the number of bogus sources can be of a hundred, and the rate of real transients can be only a handful.

A very unbalanced training set can bring about spurious results. As an example, imagine a training set with 90 bogus and 10 reals. A classifier that classifies everything as bogus, would have a success rate for bogus classification of 90\%. An unbalanced training set could also create classifiers with large variance due to the small set of reals.

To generate a balanced training set with comparable amounts of reals and bogus, I generate reals by selecting sources on an image and erasing them on the reference. That way, after a subtraction, those sources will appear as transient events, that is, objects in the image that don't have a corresponding object on the reference.

The image set used for this purpose is CSTAR, but any other set will also do.

The advantage of this method, compared to other methods like injecting fake sources, is that the `transients' obtained this way will have all the particular characteristics of the CCD and instrument from the image set in which we are interested to work. It will also have the characteristics of the subtraction method imperfections, but this is not exclusive to this method.

The `bogus' set is also collected at the same time along with the `real' set.

To have equal representation of sources of different magnitudes, the sources are first binned into 10 bins of \textcolor{red}{magnitude/flux?} and taken from different regions of the CCD.
We partition the image in a 4 by 4 grid and we select one star from each region and we do so for each magnitude bin. \textcolor{red}{explain better!!!}

\subsection{The CSTAR image-set preparation}

The CSTAR image set is a high cadence set of images taken in the winter of 2010 in Antarctica, the South Pole. It comprises 6 months of data with an average cadence of \textcolor{red}{1 min ???? check}.

Since the amount of data is so large, for the purpose of training, we created a subset of 626 images, with a cadence of about an hour during the best seeing month of June (May 31st to June 30th, 2010). The subset was named `cstar\_june\_selection'. The first 10 images of this set (named cstar\_june\_01) is used for training purposes and the rest can be used to search for transients. 

The images are mostly clean, that is they were bias and flat corrected.
Nevertheless, the images suffer from `bleeding' even though they have low exposure time. This bleeding is very significative for bright stars and less so for dimmer stars but still present nonetheless.

The bled stars were covered and a separate mask was created for the covered pixels for future reference. This pixel mask also includes defective pixels (dead lines and columns in the CCD).

The date-time information on the header was also updated according to \textcolor{red}{[ref]}.

For each of these images, we aligned a reference image with it using the package `astroalign' (see section astroalign) and also performed a image subtraction using the delta basis method on a 4 by 4 grid.

\section{The Classifier}

As previously said, we trained a Random Forest classifier based on 7,624 examples of fabricated transients (as described in section data) and 7,624 bogus picked at random from the subtraction images. This totals 15,248 samples for training and validation.

Several scores about the performance of the classifier on the training data are condensed in the scores table (\ref{mlscores}).

\begin{table}
\centering
\begin{tabular}{| >{\itshape}l | l |}
  \hline
  accuracy & 0.9935 \\ \hline
  precision & 0.9956 \\ \hline
  recall & 0.9915 \\ \hline
  F measure & 0.9936 \\ \hline
\end{tabular}
\caption{Scores}
\label{mlscores}
\end{table}

On the training data, the performance is quite good, with all indicators scoring above 99 percent.

In the confusion matrix (table \ref{mlconfusionmatrix}) we see that only 65 out of 7,624 of the reals were mis-classified as bogus, and only 33 bogus out of the 7,624 were mis-classified as reals. These very low numbers explain the very high performance scores of the classifier.

\begin{table}
\centering
\begin{tabular}{ l|c|c| }
\multicolumn{1}{r}{}
 &  \multicolumn{1}{c}{real}
 & \multicolumn{1}{c}{bogus} \\
\cline{2-3}
{\it Classified as} real & 7591 & 33 \\
\cline{2-3}
{\it Classified as} bogus & 65 & 7559 \\
\cline{2-3}
\end{tabular}
\caption{Confusion Matrix}
\label{mlconfusionmatrix}
\end{table}

The ROC curve is presented in figure ??.

\section{Testing the classifier on real data}

Once we have trained our classifier with training data, we wanted to also test it in a more real situation. For that we apply the classifier to the rest of the cstar\_june\_selection data set in search of actual transients of the image.

\begin{comment}
	* Our test-drive with OIS + ML. Results of paper (hopefully)
		* Test data we used (CSTAR)
		* Processing our raw data (selecting images for a dataset, cleaning images, fixing headers, performing subtractions, identifying sources in subtractions, stamps)
		* Getting samples for training (labeling data as RB, winnow, first run of ML)
		* Feature exploration of data (SExtractor features, derived features, morphological features like Zernike, Chebyshev, Fourier, etc.)
		* Random Forest + SMOTE + Cost Matrix (how well did this perform?)
		* Is it reproducible in other data sets?
\end{comment}



		
\chapter{Software Developed}
\section{Pipeline}

The pipeline is written entirely in Python, following the latest trend in Astronomy.

For the web pages developed to support the collaboration, we used the popular Python web framework Django.



\section{Astroalign}

\subsection{The Algorithm in a Nutshell}

The core idea of the algorithm consists on characterizing asterisms (for example triangles or quadrilaterals) by using quantities invariant to translation, rotation or even scaling and flipping. Similar asterisms will have similar invariant tuples in both images so a correspondence can be made between those invariant quantities. 

As an example, the lengths of the sides of a triangle are invariant to translation and rotation. They remain the same whatever position or orientation the triangle may have. Any function of the {\em ratio} of the sides will, in addition, be invariant to scaling.

The idea for the algorithm can be summarized in a few steps

\begin{enumerate}
\item Do for both images \begin{enumerate}
\item Make a catalog of a few brightest sources (but not too few!)
\item Create a 2D tree of the sources to quickly query for close neighbors. (A kd-tree data structure for k=2)
\item For each star, select the 4 nearest neighbors (5 sources including the star itself).
\item Form all the ${5}\choose{3}$ posible triangles from that set of stars.
\item For each triangle in that set, calculate the tuple of invariants that fully characterize the triangle and push the invariant tuple into another kd-tree.
\item There could be many duplicate triangles on the previous list, so it's best to remove them leaving only unique elements.
\end{enumerate}
\item Now do a matching between the two invariant kd-trees to find matches for similar triangles. Two similar triangles will have similar invariant features.
\item Even within a triangle match, one can make a correspondence between individual points by looking at which sides the point belongs to. This way, one can make a point to point correspondence for each triangle correspondence.
\item Pass the invariant matches set to a RANSAC algorithm that will decide which triangles suggest a transformation that fits many other triangles.
\end{enumerate}


\subsection{Selecting Asterisms and Invariant Features}

To find a correspondence we need to fix which figures will we search for in both images. 
The simplest figure is the triangle. A triangle (with all different sides) can determine a unique transformation between two images. 
Another possibility is to search for polygons with more sides. 
The package Astrometry.net uses quadrilaterals for this purpose, but even pentagons or other polygons can be used. 
We will focus on the triangle matching on this note.

For a triangle, knowledge of all its 3 side lengths is enough to fully characterize it, irrespective of position or orientation. 
If we want to characterize it up to a global scaling, then knowing 2 inner angles is enough. 
Equivalently, knowing 2 independent ratios of the side lengths is also enough. 
In fact any function of 2 independent length ratios is enough.

So, for example the tuple $(\frac{L_2}{L_1}, \frac{L_1}{L_0})$ (where $L_2 > L_1 > L_0$) is a valid invariant tuple that fully describe the triangle up to translation, rotation and scaling, and even coordinate flipping.

\subsubsection{Analysis of the invariants}

Let's analyze here the example invariant set given in the previous section.

\begin{align*} 
I_{1}(L_i,L_j,L_k) =& \left( \frac{L_2}{L_1}, \frac{L_1}{L_0} \right)  \numberthis \label{inv01} \\ 
 & \text{where} \left\{
  \begin{array}{lll} 
 L_2 &=& \max\{L\} \\
 L_1&=&\text{middle}\{L\} \\ 
 L_0 &=& \min\{L\}
  \end{array}
\right.
\end{align*}

This choice of invariants maps the positive octant of $\mathbb{R}^3$ of all possible side lengths of a triangle, 
onto a region of the positive quadrant of $\mathbb{R}^2$ in the invariant-features space 

To find out what this region is, we notice that since $L_2 > L_1 > L_0$,

\begin{align*}
x &=  \frac{L_2}{L_1} > 1 \\
y &=  \frac{L_1}{L_0} > 1
\end{align*}

Also, using the triangle inequality:

\begin{align*}
L_2 \leq L_1 + L_0 \implies & x \leq 1 + \frac{1}{y} \\
& y \leq \frac{1}{x-1}
\end{align*}

The curve $y = (x-1)^{-1}$ corresponds to colinear points.

Also, it's worth noting that any equilateral triangle will map to the point $(1,1)$ and an isosceles triangle will map either to $x=1$ or $y=1$ line depending on whether the unequal side is the largest or smallest.

Very peaky triangles will tend to accumulate between the colinear points curve and the $x=1$ line for large values of $y$.

All these observations can be summarized in figure \ref{fig:inv_region}.

\begin{figure}[htbp]
   \centering
   \includegraphics[width = \linewidth]{chapter_astroalign/figures/invariantMap01.pdf}
   \caption{Region for a particular invariant mapping}
   \label{fig:inv_region}
\end{figure}


%Other invariant sets can be constructed, each with a particular mapping from the set of side lengths of triangles to a region in the 2D plane of invariants.

%Some other examples are given in figure \ref{fig:inv_maps}. These have been explored numerically plotting the invariants for a large number of triangles.

%\begin{figure}[htb]
%   \centering
%   \includegraphics[width = \linewidth]{chapter_astroalign/figures/differentInvariantMaps.pdf}
%   \caption{Four examples of invariant mappings}
%   \label{fig:inv_maps}
%\end{figure}

%\lipsum

\subsection{An Ideal Example}

Let's see how the algorithm performs on an ideal example.

For this, we create several stars at random positions and we rotate and translate them as seen in figure \ref{fig:ideal_sources}.

\begin{figure}[htbp]
   \centering
   \includegraphics[width = \linewidth]{chapter_astroalign/figures/idealSources.pdf}
   \caption{Two ideal distribution of sources}
   \label{fig:ideal_sources}
\end{figure}

The invariant features in (\ref{inv01}) from this set of stars is plotted in figure \ref{fig:ideal_inv}.

\begin{figure}
   \centering
   \includegraphics[width = \linewidth]{chapter_astroalign/figures/idealInvariants.pdf}
   \caption{Invariants for the ideal example}
   \label{fig:ideal_inv}
\end{figure}

We note that some invariants belong to collinear points, and the distribution of points is fairly sparse. 
This will help the identification phase when we try to match our triangles.

In the figure, the invariant points for both images were plotted. They appear superimposed in the plot.

We note that every blue invariant point has its corresponding red invariant point on top. 
There are no points without a partner for neither of the images. 
This is because all the sources in one image appear in the other one, there are no missing stars.
In a real situation, some stars will be missing because they are out of the field of view or because they became too faint due to extinction or any other technical reason. 
The algorithm should still work with missing or extra stars in the reference or test image. 

Another issue with real images is that locating the position of a source is not entirely precise, so small errors will appear in the expected position of one source with respect to its partner in the other image.
This error in turn creates an error on the lengths of the sides of the triangle and thus on the invariants calculated from it.
In practice, the invariant points will lay close to each other to a given small tolerance.

Once we have the set of invariants from both images, we query a correspondence to the kd-tree for possible matches within a given tolerance radius. 
Each returned match will be a correspondence between a triangle in one image and another. 
From each, we can make the point to point correspondence, provided all sides are unequal, and this will determine a unique transformation between the images.

Some of the correspondences won't be real ones. It could be that by chance there are two similar triangles in the images that belong to different set of stars.

This is where the RANSAC algorithm comes in.

\subsection{The RANSAC algorithm}

From its Wikipedia page 

{\em ``The Random sample consensus (RANSAC) algorithm is an iterative method to estimate parameters of a mathematical model from a set of observed data which contains outliers.''}

In our case the mathematical model is the similarity transformation between both images and the parameters are those of the transformation, i.e. the rotation, translation and uniform scaling parameters.

RANSAC is capable of choosing a transformation that fits most of the other triangles and is not affected by the rest of spurious outliers. 

This algorithm is also used in the computer vision package OpenCV for a very similar purpose, to ignore outliers when trying to estimate an homography between two images. 

In our case we look for the parameters $t_x$, $t_y$ for the translation in the $x$ and $y$ direction, the rotation angle $\alpha$ and the dilation parameter $\lambda$.

The transformation applied to a point $(x,y)$ will look like this:

\begin{align*}
\left(
 \begin{array}{lll} 
 \lambda \cos \alpha & \lambda \sin \alpha& \lambda t_x \\
 - \lambda \sin \alpha & \lambda \cos \alpha& \lambda t_y \\
 0 & 0 & 1
 \end{array}
\right)
\equiv
\left(
 \begin{array}{lll} 
 a_0 & b_0 & c_0 \\
-b_0 & a_0 & c_1 \\
 0 & 0 & 1
 \end{array}
\right)
\end{align*}

To make our problem linear, we will consider the parameters $a_0, b_0, c_0, c_1$ as if they were independent.

Two data points pairs are sufficient to determine uniquely a transformation for this 4 parameters. 
More can be used if we use a linear square minimization.

%This candidate transformation $T$ will be tested against all the other triangle matches in a RANSAC algorithm.

\subsection{Error propagation}

Doing a simple propagation of errors we see that for nearly equilateral triangles $L_{1} \approx L_{2} \approx L_{3} \approx L$, 
the errors in the invariants go like $\Delta I_{i} \sim \frac{\Delta L}{L}$.
This means that, for invariants near $(I_{1}, I_{2}) = (1, 1)$ errors in the determination of the lengths of the triangle side will not magnify errors in the invariants.

%This is done by \citet{1995PASP..107.1119V}.


\section{Optimal Image Subtraction}
	
OIS (Optimal Image Subtraction) is a Python module that implements several Difference Image Analysis methods as described in section X.

It works on Numpy arrays, so that way is agnostic on what is the source of the image. 

Bad pixels that need to be ignored in the image are marked using Numpy's masked arrays (True on bad pixel).

The interface to the user has two entry points: the module methods optimal\_system and subtract\_on\_grid. The latter is just a convenient method to partition the image in a certain grid and perform the former on each grid, taking into account pixels outside the grid when necessary.

optimal\_system will return ...

OIS has documentation on the popular documentation site readthedocs.io.
OIS is released under MIT Licence and has a GitHub page on ...


    
\chapter{Conclusions}
* The future of LIGO, TOROS and multi messenger astronomy


\appendix
\chapter{Derivation of the Gravitational Wave Equations} \label{gwderivation}

Gravitational waves are a particular kind of solution to the Einstein's field equations of General Relativity.

In many situations we can consider, we are in a flat background situation, in which our metric does not differ much locally from the Minkowski metric.
Suppose we are far away, removed from any strong source in an asymptotically flat spacetime.

In such situation we can assume that locally our metric is the Minkowski metric $\eta$ plus some small deviation $h$. We want to study the dynamics of such small perturbation in a linearized Einstein field equation. 


To do that, we consider the metric $g = \eta + h$ in the linearized Einstein's field equations. Linearized here means that quadratic and higher factors of $h$ will be simply ignored, as they are assumed much smaller than the flat metric.

Let's find out what condition the Einstein field equation $G_{\mu \nu}(\eta + h) = 8\pi T_{\mu \nu} = 0$ imposes on this small perturbation $h$.

%\begin{align}
%G_{\mu \nu}(\eta + h) = 8\pi T_{\mu \nu} = 0
%\end{align}

%It is worth noting here that Einstein equations are not linear in nature, but they are second order on the metric derivatives and furthermore linear on the second order terms. This will allow us to derive from the linearized Einstein equations, a second order wave equation on the small perturbation $h$ that decouples from $\eta$.

%Another issue to point out is that the vague definition of 'small perturbation' has to be of physical nature and not an artifact on the choice of the coordinate system. We will not worry about this issue on the present work, but it is worth keeping it in mind. Other concerns about coordinate system effects were historically raised and settled in time, particularly the many gauge choices done during the derivation.

The Christoffel symbols to first order in $h$ are:

\begin{align}
\Gamma_{abc} &= \frac{1}{2} \left( g_{ca,b} + g_{cb,a} - g_{ab,c} \right) \\
  &= \frac{1}{2} \left( (\eta + h)_{ca,b} + (\eta + h)_{cb,a} - (\eta + h)_{ab,c} \right) \\
 &=\frac{1}{2} \left( h_{ca,b} + h_{cb,a} - h_{ab,c} \right) 
\end{align}

At this point we must note that we will still call $g^{ab}$ the inverse of $g_{ab}$, which is $g^{ab} = \eta^{ab} - h^{ab}$ to first order in $h$. 

And thus,
\begin{align}
\Gamma^{a}_{bc} &= g^{cd} \Gamma_{abd} \\
 &=(\eta - h)^{cd} \Gamma_{abd} \\
 &= \frac{1}{2} \eta^{cd} \left( h_{ca,b} + h_{cb,a} - h_{ab,c} \right) + \order(h^2)
\end{align}

Similarly, the Ricci tensor to first order in $h$ will be:

\begin{align}
R_{ab} &= \Gamma^{c}_{ab,c} - \Gamma^{c}_{cb,a} \\
&= \frac{1}{2} \left( h_{a}{}^{c}{}_{,bc} + h_{b}{}^{c}{}_{,ac} - h_{ab,c}{}^{c} - h_{c}{}^{c}{}_{,ab} \right)
\end{align}

Which will finally lead us to the Einstein's tensor $G$:

\begin{align}
G_{ab} &= R_{ab} - \frac{1}{2}g_{ab}R \\
&= R_{ab} - \frac{1}{2}\eta_{ab}R + \order(h^2) \\
&= \frac{1}{2} \left( h_{ac,b}{}^{c} + h_{bc,a}{}^{c} - h_{ab,c}{}^{c} - h_{c}{}^{c}{}_{,ab} -\eta_{ab} \left( h_{cd,}{}^{cd} - h_{c}{}^{c}{}_{,d}{}^{d} \right) \right)
\end{align}

This can be brought to a shorter form defining $\bar{h}{}_{ab} = h_{ab} - \eta_{ab} h$

Where $h = h_{c}{}^{c}$ is the trace of the metric tensor $h$.
With this definition, the Einstein's field equation becomes:

\begin{align}
\bar{h}{}_{ab,c}{}^{c} + \bar{h}{}_{ac,b}{}^{c} + \bar{h}{}_{bc,a}{}^{c} -\eta_{ab} \bar{h}{}_{cd,}{}^{cd} = 0
\end{align}

\section{Gauge choices}

There is a gauge freedom in General Relativity corresponding to the group of diffeomorphisms, that can be used to simplify the equations even more. Just like the Gauge freedom in electrodynamics $A \rightarrow A + \partial \chi$ we can chose $h$ to satisfy $\bar{h}{}_{ab,}{}^{b} = 0$

In this ``Lorentz Gauge'', the linearized Einstein field equations, reduce to the usual wave equation for each component of $\bar{h}$:

\begin{align}
\bar{h}_{ab,c}{}^{c} = \Box{\bar{h}{}_{ab}} = 0
\end{align}

Further gauge choices can let us choose the trace of $\bar{h}$ to be zero and each $h_{0\mu}$ component to be zero for $\mu = 0,1,2,3$. Notice that once the trace of $\bar{h}$ is set to zero, it implies that $\bar{h}{}_{ab} = h_{ab}$

\begin{align} \label{gaugecond}
& \Box{h_{ab}} = 0 \\
& h_{ab,}{}^{b} = 0 \\
& h_{a}{}^{a} = 0 \\
& h_{0 \mu} = 0; \; \mu=0,1,2,3
\end{align}

This is called the transverse traceless (TT) gauge. The reader can refer the full derivation for the gauge choices in Misner or Wald.

With the conditions in (\ref{gaugecond}), we seek solutions in the form of plane waves of the form $h_{ab} = H_{ab}e^{\pm \imath k_{\mu}x^{\mu}}$.

Our gauge conditions impose similar conditions on $k$ and $H$:

\begin{align}
& k^{\mu} H_{\mu \nu} = 0 \\
& H_{\mu}{}^{\mu} = 0 \\
& H_{0 \mu} = 0; \; \mu=0,1,2,3
\end{align}

This leaves $H$ with only two independent components. If we consider a wave propagating in the z direction, the most general form for a transverse traceless metric will be:

\begin{equation}
\begin{bmatrix}
0 & 0 & 0 & 0 \\
0 & h_{+} & h_{\times} & 0 \\
0 & h_{\times} & -h_{+} & 0 \\
0 & 0 & 0 & 0 \\
\end{bmatrix}
e^{\pm \imath k_{\mu}x^{\mu}}
\end{equation}

and the total metric $g$

\begin{equation}
\begin{bmatrix}
-1 & 0 & 0 & 0 \\
0 & 1 + h_{+}e^{\pm \imath k_{\mu}x^{\mu}} & h_{\times} e^{\pm \imath k_{\mu}x^{\mu}} & 0 \\
0 & h_{\times} e^{\pm \imath k_{\mu}x^{\mu}} & 1 - h_{+}e^{\pm \imath k_{\mu}x^{\mu}} & 0 \\
0 & 0 & 0 & 1 \\
\end{bmatrix}
\end{equation}

The two independent polarizations are called ``h plus'' ($h_+$) and ``h cross'' ($h_{\times}$).

\chapter{Notations}

Here we show the use of multiple appendixes.

\section{Math Notations}

Each appendix can have sub-sections as a regular chapter.

\section{Additional Notations}

\pagebreak{}

\bibliographystyle{plain}
\nocite{*}
\bibliography{sampleThesis}

\begin{vita}
\section*{Education}

\begin{itemize}
  \item B.S. Physics, National Cordoba University (Argentina), 2007.
  \item M.S. Physics, University of Texas at Brownsville, 2010.
\end{itemize}

\section*{Publications}

\begin{itemize}
\item \href{http://adsabs.harvard.edu/abs/2016ApJ...828L..16D}{\bf GW150914: First search for the electromagnetic counterpart of a gravitational-wave event by the TOROS collaboration} (2016). Mario C. Diaz et al; {\it The Astrophysical Journal Letters}, Volume 828, Issue 2, article id. L16, 6 pp.
\item \href{http://adsabs.harvard.edu/abs/2016ApJ...826L..13A}{\bf Localization and broadband follow-up of the gravitational-wave transient GW150914} (2016). B. P. Abbott et al. {\it The Astrophysical Journal Letters}, Volume 826, Issue 1, article id. L13, pp
\item \href{http://www.math.unm.edu/~lau/DMS1216866/tauFluids_BHLP.pdf}{\bf Multidomain, sparse, spectral-tau method for helically symmetric flow} (2014). M Beroiz, T Hagstrom, SR Lau, RH Price; {\it Computers and Fluids}  (2014), pp. 250-265.
\item \href{http://iopscience.iop.org/1538-3881/147/5/100}{\bf Bright microwave pulses from PSR B0531+21 observed with a prototype transient survey receiver} April 2014. J. Andrew O'Dea, F. A. Jenet, Tsan-Huei Cheng, Chau M. Buu, Martin Beroiz, Sami W. Asmar, and J. W. Armstrong; {\it Astronomical Journal}, 147, 100.
\item \href{http://ieeexplore.ieee.org/xpls/abs_all.jsp?arnumber=5555950}{\bf A Prototype Radio Transient Survey Instrument for Piggyback Deep Space Network Tracking}, May 2011. Chau M. Buu, Fredrick A. Jenet, John W. Armstrong, Sami W. Asmar, Martin Beroiz, Tsan-Huei Cheng, and J. Andrew O'Dea; {\it Proceedings of the IEEE} Volume 99, Number 5.
\item \href{http://journals.aps.org/prd/abstract/10.1103/PhysRevD.76.024012}{\bf Gravitational instability of static spherically symmetric Einstein-Gauss-Bonnet black holes in five and six dimensions}, 2007. M. Beroiz, G. Dotti y R.J. Gleiser, {\it Physical Review D} 76, 024012 hep-th/0703074.
\end{itemize}


\section*{Scientific Meetings and Conferences}

\begin{itemize}
\item \href{http://adsabs.harvard.edu/abs/2016APS..APRR14005B}{Speaker at the APS April Meeting 2016}, abstract R14.005. "Results of optical follow-up observations of advanced LIGO triggers from O1 in the southern hemisphere". South Lake City, April 2016.
\item \href{https://conference.scipy.org/scipy2014/schedule/presentation/1730/}{SciPy Conference 2014. Speaker at Mini Symposium in Astronomy}. "Transient detection and image analysis pipeline for TOROS project". Austin July 2014.
\item Advances and Challenges in Computational General Relativity, Brown University, Providence, RI, May 20-22, 2011.
\item Poster presentation 215th American Astronomical Society (AAS) Meeting, Washington DC, January 3-7, 2010 ``The X-ray Emission of SN1978K: Still Here After All These Years''  - Eric M. Schlegel, M. Beroiz.
\item Poster presentation 215th American Astronomical Society (AAS) Meeting, Washington DC, January 3-7, 2010 ``Current Results at PALFA Pulsar Survey at Arecibo Observatory'' - M. Beroiz, K. Stovall, F. Jenet, J. Cordes, D. Lorimer, D. Backer, PALFALFA Consortium.
\item Summer Provost Research Program at UTSA, 2009. Research on X-ray spectrum of Super Nova 1978K and Poster Presentation.
\item UTB Summer School in Gravitational Wave Astronomy at South Padre Island TX, June 2008.
\item Grav07, La Falda (Cordoba), Argentina. November 5-7, 2007.
\item Poster Presentation at 92nd AFA (Physics Association Argentina) Annual Meeting, Salta, Argentina, September 24-27, 2007.
\end{itemize}

\end{vita}

\end{document}
